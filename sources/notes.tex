\documentclass[11pt,a4paper]{article}
\usepackage[a4paper,margin=1cm,footskip=.5cm]{geometry}
\usepackage{fontenc}
\usepackage{inputenc}
\usepackage{authblk}
\usepackage[version=3]{mhchem} % Formula subscripts using \ce{}
\usepackage{todonotes}
\usepackage{amssymb}
\usepackage{amsmath}
\usepackage{graphicx}
\usepackage{dcolumn}
\usepackage{pdflscape}
\usepackage[toc,page]{appendix}
\usepackage{afterpage}
\usepackage{multirow}
\usepackage{capt-of}% or use the larger `caption` package
% \usepackage[round]{natbib}
% \usepackage[LGRgreek]{mathastext}
\usepackage[strict]{changepage}
\usepackage{color}
\usepackage{lscape}
\usepackage{listings}
\usepackage{hyperref}
\usepackage{wrapfig}
\usepackage{siunitx}
\usepackage{marvosym}
\usepackage{indentfirst}
\usepackage{titlesec}
\usepackage[round, sort, numbers, authoryear]{natbib}
\usepackage{tabularx}

\def\mean#1{\left< #1 \right>}


\graphicspath{ {./img/} }

\renewcommand{\arraystretch}{1.5}

% settings for hyperlinks
\hypersetup{
    % bookmarks=true,         % show bookmarks bar?
    % unicode=false,          % non-Latin characters in Acrobat’s bookmarks
    % pdftoolbar=true,        % show Acrobat’s toolbar?
    % pdfmenubar=true,        % show Acrobat’s menu?
    % pdffitwindow=false,     % window fit to page when opened
    % pdfstartview={FitH},    % fits the width of the page to the window
    % pdftitle={My title},    % title
    % pdfauthor={Author},     % author
    % pdfsubject={Subject},   % subject of the document
    % pdfcreator={Creator},   % creator of the document
    % pdfproducer={Producer}, % producer of the document
    % pdfkeywords={keyword1} {key2} {key3}, % list of keywords
    % pdfnewwindow=true,      % links in new PDF window
    colorlinks=true,       % false: boxed links; true: colored links
    % linkcolor=red,          % color of internal links (change box color with linkbordercolor)
    citecolor=blue,        % color of links to bibliography
    filecolor=black,      % color of file links
    urlcolor=blue,           % color of external links
    linkcolor=blue
}

\setcounter{secnumdepth}{3}


\title{Notes}

\author[1]{Igor Markelov}
\affil[1]{University of Waterloo, Canada}

\begin{document}
    \maketitle

    \tableofcontents

    \section{Introduction}



    \section{Evolution. Birth of Complexity} % (fold)
    \label{sec:birth_of_complexity}

    \subsection{What is life?} % (fold)
    \label{sub:what_is_life_}

    Some scientists believe that life is the process, not the structure, and define it as the process of keeping non-equlibrium conditions by organic system with extraction of energy from environment. \cite{Markov2010}

    Others emphasize mandatory discretizing of living facilities and believe that the concept of \"life\" is inseparable from the concept of \"organism\". Third emphasize the informational nature of life and define it as the ability of some fragment of information ("replicator") to copy itself using the resources of the environment. \cite{Markov2010}

    Properties of life \cite{Markov2010}:

    \begin{itemize}
        \item having the genetic information (DNA);
        \item active implementation of the functions of self-maintenance and reproduction, as well as to obtain the energy required to perform all this work (Proteins).
    \end{itemize}

    \subsection{What was the first: RNA, DNA or proteins?} % (fold)
    \label{sub:what_was_first_rna_dna_or_proteins_}


    RNA - serve as intermediaries between DNA and proteins, providing a readout of genetic information. \cite{Markov2010}

    RNA world theory: according to which the first living creatures were RNA organisms without proteins and DNA. A first prototype of the future RNA-organism could be the autocatalytic loop formed by replicating RNA molecules - ribozymes, which are capable of catalyzing the synthesis of copies of themselves. \cite{Markov2010}

    Many ribozymes work best at low temperatures, and sometimes even below the freezing point of water - ice in tiny cavities which have high reactant concentrations. Some consider this an indication that life originated at low temperatures. \cite{Markov2010}

    A, G, C, U - the standard nucleotides: adenosine, guanosine, cytidine and uridine other letters marked nonstandard (modified) nucleotides, including - inosine.\cite{Markov2010}

    \subsection{Ancient Ocean and Metals} % (fold)
    \label{sub:ancient_ocean_and_metals}

    % subsection ancient_ocean_and_metals (end)

    It is possible that the matter is necessary cofactors - ions of metals, including rare. It is known that many protein enzymes use metal ions as its essential components. Such proteins are referred metalloprotein. \cite{Markov2010}

    It is known that in the ancient ocean was much more than now, the different ions of heavy metals, including quite exotic, such as wolfram, molybdenum or vanadium. \cite{Markov2010}

    The important and not yet fully understood the role of metals in the life of primitive cells show the results of the study of unusual microbe discovered in 2000 in a bioreactor ray Metallurgical plant in Tula. Unlike plant source of energy for organic synthesis from CO2 ferroplasma do not use the sunlight, but the chemical reaction of oxidation of ferrous iron \ce{Fe^{2+}} to \ce{Fe^{3+}}. \cite{Golyshina2000}.

    \cite{Ferrer2007} assumed that the situation is observed in ferroplasma is accidentally preserved remnants of ancient life stages of development. Life could have arisen in the microcavities crystals of pyrite, in conditions very similar to those that now inhabits ferroplasma.

    At first, the earliest forms of life actively used simple inorganic catalysts to make the necessary chemical reactions, especially iron compounds, and sulfur. Gradually, these catalysts were replaced by more efficient organic - that is, proteins, and it is natural to assume that the first proteins include the iron atoms as essential structural and functional components. \cite{Markov2010}

    \subsection{Biogeochemical cycles and the origin of life} % (fold)
    \label{sub:in_the_begining_there_was_society}

    \paragraph{Hydrothermal vents.} Recently, a German chemist, managed to prove the possibility of abiotic synthesis of organic substances in the environment, which still exist on the ocean floor. It was found that in submarine hot volcanic sources chemical reactions could occur: using inorganic compounds such as carbon monoxide (CO) and hydrogen cyanide (H CN), and variety of organic molecules could be formed. The catalyst for these reactions(iron and nickel) are present in the hydrothermal solids waters. Reactions are proceeds well at temperatures of 80-120 degrees. The conditions under which the experiments were conducted, were as close as possible to reality. According to the researchers, such conditions (including all components of the reaction mixture) could exist in underwater volcanic vents in the early stages of the Earth \citep{Johnson2008}.


    Not the one place of start. Some scientists proceed from the fact that the stable existence of the biosphere is only possible if the there is relative close (use of all) of biogeochemical cycles - otherwise creatures quickly use up all the resources or poison themselves products of their own activity.\cite{Markov2010}

    Organism capable alone to close the cycle, just is not possible, as a perpetual motion machine. \cite{Markov2010}

    Theoretically, it is still possible to imagine a single type of micro-organisms on the planet for a very long time. For example, if the `food' is coming little by little from the earth, and the waste is either recycled in the geochemical cycle without the participation of living organisms or, for example, simply disposed of in the Earth crust. Thus, this hypothetical organism is simply built into the already-established geochemical cycles, only catalyze it.\cite{Markov2010}

    \subsection{The history of Earth} % (fold)
    \label{sub:the_history_of_earth}

    Earth formed 4.5-4.6 billion years ago, but from the first 700-800 million years of its existence in the crust left too few traces. \cite{Markov2010}.

    There is evidence that the hydrosphere - water shell of our planet - appeared very early. The evidence is the crystals of zircon age of 4.4 billion years old, discovered in Western Australia. Structure and isotopic composition of these crystals suggest that they were formed in the presence of water. \cite{Markov2010}

    The earliest evidence of life is a lightweight carbon isotope composition of graphite inclusions in apatite crystals found in Greenland in the sediments of 3.8 billion years old. These inclusions have the increased percentage of light carbon isotope 12C, which may be the result of vital activity of autotrophs - organisms capable of synthesizing organic matter from \ce{CO2}. \cite{Markov2010}

    Geological time scale table by \cite{Gradstein2004,Gradstein2004a}.

    \subsection{Microorganisms are the cause of sedimentary formation of Early Earth.} % (fold)
    \label{sub:microorganisms_are_the_cause_of_sedimentary_formation_of_early_earth_}

    One of the most surprising discoveries in geology over the past decade is that, as it turned out, almost all the geological processes that formed the sedimentary cover of the planet, have actively participated (and still participate) microorganisms. It is proved that many ore deposits - not only iron, but gold, manganese and many others - have a biological origin. \cite{Markov2010}

    Zinc deposits arose due to bacteria. \cite{Moreau2006}

    \subsection{The most important cycle of life} % (fold)
    \label{sub:the_most_important_cycle_of_life}

    According to the results of comparative genetic studies, and the logically. First, at least one of the first could, be chemoautotrophs - Archaea-methanogens. In the simplest case, they gain energy by reduction of carbon dioxide to methane using molecular hydrogen \citep{Markov2010}:

    \begin{equation}
        \ce{H2 + CO2 -> CH4 + H2O}
    \end{equation}

    Life on Earth is based on the ability of autotrophic organisms produce organic matter from carbon dioxide (\ce{CO2}). \cite{Markov2010}

    Most organisms are used to lock the \ce{CO2} cyclic sequence of chemical reactions known as Calvin cycle. \cite{Markov2010}

    \subsection{OM limitting environments. The invention of anoxic photosynthesis} % (fold)
    \label{sub:organic_limitting_environments}

    % subsection organic_limitting_environments (end)

    Therefore, until recently, it was unclear how Archaea take RuBP, substance absolutely necessary for fixing \ce{CO2}. It turned out that RuBP formed in Archean not from other phosphorylated sugars in the Calvin cycle, and of molecules that are building blocks of RNA and performing many other functions in the living cell. There is every reason to believe that this way of fixing \ce{CO2} is evolutionarily older than the Calvin cycle. It is no accident, he found it in archaea - organisms, which many experts consider the most archaic forms of life. \cite{Sato2007}

    With a lack of dissolved organic matter marine bacteria can be feeded by sunlight. Recent discoveries have shown that in addition to the green plants and cyanobacteria carry out photosynthesis using chlorophyll a, and previously known phototrophic bacteria that do the same thing with bacteriochlorophylls, powered by sunlight are many other microbes that have light-sensitive proteins - proteorhodopsins. \cite{Sabehi2005}

    They found proteorhodopsins in Dokdonia bacterium living in the Mediterranean. Bacteria able to cultivate in the laboratory. It was found that in natural seawater bacterium grows on light much better than in the dark. \cite{Gomez-Consarnau2007}

    Thus, proteorhodopsins improve viability of marine bacteria only at low (but not too low) concentrations of dissolved organics. Perhaps, the negative results obtained previously with Pelagibacter could be explained by irrational amount of organics in test environments. \cite{Markov2010}

    \subsection{Bacterial mats} % (fold)
    \label{sub:bacterial_mats}


    The invention anoxic photosynthesis was a big step forward. Living beings who have mastered the secret of photosynthesis, have access to inexhaustible energy source - sunlight. However, they rely on scarce chemicals slowly coming out of the ground. The fact that the single light for photosynthesis is not enough, they should have any substance, where the electron can be torn off (this is called photo-oxidation). In the simplest case hydrogen sulfide asts as an electron donor in photosynthesis. As a result of anoxic photosynthetic hydrogen sulfide is converted to sulfur (\ce{S}) or sulfate (\ce{SO4^{2-}}). These communities have survived to this day in their respective habitats - where enough methane and sulfate (for example, in the vicinity of underwater mud volcanoes). \cite{Markov2010}

    Around the same period (over 3.2 billion years ago), apparently were the first heterotrophs - fermenters that get energy from the anoxic fermentation of organic matter, produced by autotrophs. As wastes fermenters tend to secrete molecular hydrogen, for which in ancient biosphere were hunters: firstly, archaea, methanogens, secondly, bacteria - sulfate reducers (they readily used molecular hydrogen as a reducing agent for the reduction of sulfates). \cite{Markov2010}

    Clearly, at this stage most microorganisms could not do without each other. Even today, many fermenters flatly refuse to grow in the absence of microbes utilizing hydrogen (sulfate reducers or methanogens). \cite{Markov2010}

    Ancient bacterial mats, probably consisting of two layers. Phototrophs lived In the upper anoxic. They synthesized organic matter from carbon dioxide, consumed hydrogen sulfide and secrete sulfates. Fermenters lived in the lower layer (they consumed organic matter produced by phototrophs and release hydrogen), sulfate reducers (consumed sulfate and produced hydrogen sulfide), and, possibly, with methanogens with methanotrophs. Beneath  this community gradually, layer by layer, sediment were accumulated  - layered formation, known as stromatolites. Calcium carbonate - the main building material of stromatolite - partly precipitated from sea water was produced in part by the microbes.\cite{Markov2010}.

    \paragraph{Layers.} In the beginning there was the community. It was a three-layered bacterial mat, almost the same as modern bacterial mats, with the difference that the top layer is not formed by oxygen (oxygenic) but anoxic photosynthetic bacteria . These were the ancestors of cyanobacteria, which has not learned yet how to use water as an electron donor. They consumed the old hydrogen sulfide and sulfur or sulfates were isolated \citep{Markov2010}.

    The second layer is formed by another anoxic photosynthetic bacteria, including alpha proteobacteria - the ancestors of the purple bacteria. This pink creation and today live in bacterial mats beneath the cyanobacteria, because they consume longer wavelength light, which easily passes through the top layer of the green community \citep{Markov2010}.

    In the third layer were many any trifles. There were fermenting bacteria that fermented excess organics produced by photosynthetic in the upper layers. They isolated the molecular hydrogen, which is used for the reduction of sulfates by sulfate reducers. As a result of their activities in the community to replenish stocks of hydrogen sulfide required by two upper layers. Here was also methanogenetic bacteria, who are willing to use hydrogen produced fermenters for reduction of carbon dioxide and methane synthesis. Methanogens today live almost everywhere, where there is no oxygen and fermenters - for example, we have in the intestine \citep{Markov2010}.

    \subsection{Oxygen Toxicity} % (fold)
     \label{sub:oxygen_toxicity}

    The community was quite stable and could easily exist in a form of hundreds of millions of years (it, apparently do). But then the cyanobacteria `invented' oxygenic photosynthesis and began to produce oxygen the peaceful life came to an end. For all the ancient forms of life on earth - and all without exception, members of our ancient community - oxygen was a dangerous poison. Even cyanobacteria itself was not very comfortable to live in a poisoned - in their view - the environment. But the opportunity to finally get rid of the `hydrogen sulfide addiction' outweighed all other considerations. Of course, the cyanobacteria came extremely selfish - for their own independence, they have poisoned nearly every living thing on the planet, but in the end they turned out to be useful for the selfishness of the biosphere. After all, without it earth still would remain `the planet of microbes'. Fortunately for cyanobacteria, they quickly found a way to neutralize the toxic products of its own life. And the same way - and even with greater efficiency - inhabitants of the second layer used to protect themselves against poison, purple bacteria \citep{Markov2010}.

    \subsection{Learn how to respire the oxygen.} % (fold)
    \label{sub:learn_to_respire_the_oxygen_}

    Simply put, during the process of photosynthesis the quantum of light knocks electrons out of the chlorophyll molecule. This `excited' electrons are then transferred by the chain of proteins and gradually losing its energy, which is the synthesis of ATP. Finally, the electron returns to the place that is transmitted chlorophyll molecule - the same or different. `The general idea' of respiring oxygen is not the electron is taken from chlorophyll but from some other organic molecules (for example, pyruvate), and then similarly transmitted through the chain of proteins and finally to the oxygen! Having obtained the required number of electrons, toxic oxygen immediately takes over the corresponding number of protons (\ce{H+}) and converted into harmless water. So tricky cell kills two birds: neutralizes dangerous poison and stores energy. After all, the system of ATP synthesis by the energy transmitted `from hand to hand' was inherited by respiratory `molecular machine' from the unit of photosynthesis in cyanobacteria. The respiration is carried out by the same protein complexes, so that between the respiration and photosynthesis processes, there is even a kind of competition for the `right to use' proteins - electron carriers \citep{Markov2010}.

    It is known that about 2.4-2.5 billion years ago, there was a rapid increase in the concentration of oxygen in the atmosphere and hydrosphere. This is called `the great oxidation event'.


    \subsection{Symbiosis} % (fold)
    \label{sub:hydrothermal_vents}

    Bacteria Ruthia magnifica, intracellular symbiont, living in the tissues of the bivalve Calyptogena magnifica. This symbiotic `superorganism' dwells on the seabed in deep water near the hydrothermal vents. I must say that the ocean floor at a depth of several kilometers are usually quite deserted, and not because that living organisms can not withstand high pressure. The limiting factor is the food: here light does not penetrate, photosynthesis is impossible and benthic animals may eat only those meager crumbs that fall from above, from the illuminated layers of water, where life is much richer \citep{Newton2007}.

    The genome analysis confirmed that the bacterium, as expected, fix inorganic carbon using the Calvin cycle - though with some `non-traditional' features. Bacteria gains the energy by the oxidation of sulfur compounds. When there is hydrogen sulfide in excess bacteria oxidize it to sulfur, which is stored in granules in bacterial cells for `bad days'. These stocks can then be used for further oxidation (to sulfite and then to sulfate, which is derived from the cell by specific protein transporters) \citep{Newton2007}.

    However, where the hydrogen sulfide or methane is leaking from the seabed,oases of life are blooming. They exist due to chemoautotrophic  bacteria which oxidize H2S and CH4 using oxygen (rarely using sulfates or nitrates as the oxidant). The energy released during this chemical reaction is used for the synthesis of organic compounds from carbon dioxide. All animals living in hydrothermal oases (worms, clams, shrimp) feed this bacteria by filtering them out of the water or take them for the symbiosis on the surface of his body, or even inside it \citep{Markov2010}.

    A huge role in the biosphere also play symbioses of autotrophs and heterotrophs - cooperation organisms that synthesize organic matter from carbon dioxide, with consumers ready organics \citep{Markov2010}.

    The greatest perfection of the system is achieved in three-lichen, which include in addition to the fungus host specialized on photosynthesis green algae and specializing in nitrogen fixation cyanobacteria \citep{Markov2010}.

    Special and very extraordinary case the connection between the food chain and biochemical symbiosis is clam Elysia viridis, feed on algae. This mollusc manages to relocate the plastids of eaten algae in their own cells and keep them for a long time living there, thus acquiring the ability to photosynthesize \citep{provorov2005metabolic}.

    \subsection{Inevitable Degradation after Symbiosis} % (fold)
    \label{sub:inevitable_degradation_after_symbiosis}

    The main feature of Ruthia magnifica, of course, is her amazing biochemical independence. This complete set of genes is required for the basic biochemical processes of chemoautotrophs, weren't found in the other intracellular bacteria. Apparently, the symbiotic system Calyptogena magnifica - Ruthia magnifica in the early stages of formation and evolutionary symbiont had not yet been very far along the path of inevitable degradation \citep{Newton2007}.

    The opposite example showing how far can the microbe in the transformation into an organelle, were given by bacteria Carsonella. Carsonella is living in psyllites (Psyllidae) - small, aphid-like insects that feed exclusively on plant sap. Like other insects adhering to this modest diet (e.g., bugs and aphids), psylla acquired bacterial assistants, which are necessary for the synthesized material missing from the vegetable juice, especially amino acids. The successful symbiosis proved to be the decisive factor which allowed psylla (and other insects) go on food only with vegetable sap. \citep{Nakabachi2006}.

    \subsection{Three-Way Symbiosis} % (fold)
    \label{sub:three_way_symbiosis}

    \cite{Marquez2007}


    \subsection{Abiogenic organic matter} % (fold)
    \label{sub:abiogenesis_}

    One of such unique community has recently been found deep underground in South Africa. It all started with the fact that the miners in the South African gold mine Mponeng  began to drill another well, and at a depth of 2.8 km stumbled upon the aquifer. Deep water, lost among the basalts age of 2.7 billion years old, were under high pressure, have an alkaline reaction and were saturated with all kinds of chemical species: various salts, among which the sulfate dissolved gases such as hydrogen, methane, carbon dioxide and others, and simple organic compounds (hydrocarbons, formate, acetate). Most of the organic matter, according to the isotopic composition has abiogenic origin, that is not generated by living organisms but by geological processes. The temperature of the underground water - slightly above 60 degrees. Microbiota found in the water of the well of the South African, was the first proven case of an autonomous long existence of living organisms in the bowels of the earth, without any connection with the `big biosphere' \citep{Lin2006}. More than 88\% of these bacteria are one species of sulfate-reducing bacteria. These bacteria get energy reducing sulfate (\ce{SO4^{2-}}) with molecular hydrogen. Furthermore, besides the sulfate reducers, other microbes in small amounts were found  - only about 25 species, including four species of methanogenic archaea. Calculations have shown that under such conditions sulphate reduction is the most advantageous of all possible types of energy metabolism \cite{Lin2006}.


    \subsection{Single-Species Ecosystem from Mponeng} % (fold)
    \label{sub:single_specie_ecosystem}

    To start an underground bacterium was called Desulforudis audaxviator. ``Audax viator'' - the words of the mysterious Latin phrases to target hero of Jules Verne novel `Journey to the Center of the Earth'. Translated it means `a brave wanderer'. Well, the name is quite appropriate. According to researchers, the microbe made his courageous journey into the bowels of the earth and has adapted to life in solitude at least 20 million years ago \citep{Chivian2008}.

    Since the `brave wanderer' alone performs all the functions to be performed by living beings in an ecosystem, the authors expected and found out that the genome must contain a full set of livelihood in extreme conditions, including biochemical mechanisms for energy, nitrogen and carbon fixation and synthesis of all the necessary substances \citep{Chivian2008}.

    But what  dint have at all, even in a rudimentary form, it is the proteins that allow to use oxygen, or at least to defend itself against its toxic effects. This means that `wanderer' haven't dealt with the oxygen for a long time \citep{Chivian2008}.


    \subsection{Altruists} % (fold)
    \label{sub:first_altruists}

    Apparently, at the earliest stages of the development prokaryotic biosphere microbes had to cooperate with each other, unite in complex groups and jointly solve their biochemical `tasks'. Efficiency and stability of microbial communities is enhanced by the development of communication between microbes. Developing systems of chemical `dialogue'. Secreting into the environment a variety of substances, microorganisms reports to the `neighbors' about their condition and influence their behavior. Then the altruism was born - the ability to sacrifice their own interests for the good of the community \citep{Markov2010}.


    if necessary Bacillus subtilis is able to grow the flagella and acquire mobility; to collect in the `pack' in which the movement of microbes become consistent; take the `decisions' based on chemical signals from the relatives. It uses a special `quorum sensing' - a kind of chemical voting, when a certain critical number of votes changes the behavior of bacteria. Moreover, B. subtilis is able to assemble into multicellular aggregates where the complexity of its structure approaching multicellular organisms \citep{Markov2010,Ellermeier2006}.

    \subsection{Hunger and Cannibalism} % (fold)
    \label{sub:cann}

    In an emergency (for example, during prolonged fasting) Bacillus subtilis are transformed into spores that are resistant to the adverse effects to wait for better times. But transforming to the spores for the B. subtilis is an expensive process that requires activation of about 500 genes, and this measure is reserved for the most extreme case. Well, as the penultimate measures famine microbe resorted to murder and cannibalism of their relatives. Unless, of course, relatives around quite a lot, that is, the population density is high. If not, then do nothing, they have to turn into a spores on an `empty stomach' \citep{Markov2010,Ellermeier2006}.

    Activation SpooA leads to a cascade of reactions, including the production of a cell toxin SdpC, killing those bacilli which `switch' is turned off. However, the trick is that fasting activates SpooA only in half of microbes. Killed cells release organic matter which is consumed by `killers' \citep{Markov2010,Ellermeier2006}.

    \subsection{Blue-green revolutionaries} % (fold)
    \label{sub:blue_green_revolutionaries}

    The most important turning point in the development of life was the invention of oxygenic, or oxygen, photosynthesis, whereby began to accumulate in the atmosphere and oxygen became possible existence of higher organisms. This great event occurred around  2.5-2.7 billion years ago (although some scientists believe in more earlier appearance of photosynthetic oxygen). `Inventor' of oxygenic photosynthesis was the cyanobacteria, or, as they used to be called blue-green algae \citep{Markov2010}.


    How the transition from anoxic photosynthesis (in which the electron donor is hydrogen sulfide) to oxygen, in which the electron donor is water happened? Back in 1970, has been proposed a theoretical model according to which the transition was realized through an intermediate stage as an electron donor serving nitrogen compounds \citep{Olson1970}.

    Only in 2007 the nitrogen photosynthesis (the intermediate stage on the way to the oxygen photosynthesis) has finally been discovered. The discovery was made during the study of microbes that live in fresh water and sewage lagoons. Microbiologists from the University of Konstanz (Germany) grew bacteria in anoxic conditions in the light in a medium with a small amount of nitrite (\ce{NO2-}). After a few weeks in 10 of 14 samples the pink color became noticeable which is the characteristic of bacteria practicing anoxic photosynthesis, and it was recodered the oxidation of nitrite to nitrates (\ce{NO3-}) \citep{Griffin2007}.

    Some biologists say, using metaphorical language, that the plants are just comfortable `houses' for living of cyanobacteria \citep{Markov2010}.

    \subsection{Nitrogen Fixation} % (fold)
    \label{sub:nitrogen_fixation}

    The main problem faced by nitrogen-fixing cyanobacteria is that the key enzymes of nitrogen fixation - nitrogenase - can not work in the presence of oxygen, which is released during photosynthesis \citep{Markov2010}.

    Until recently, scientists believed that combine photosynthesis and nitrogen fixation in the same cell is not possible. However, recent studies have shown that we are greatly underestimating the metabolic ability of cyanobacteria \citep{Markov2010}.

    Cyanobacterium Synechococcus manage to combine in his single cell photosynthesis and nitrogen fixation, separating them in time. During the day they photosynthesize, but at night, when there is no light, photosynthesis stops and the concentration of oxygen in the cyanobacterial mat abruptly falls, they switch to nitrogen fixation. Thus it was possible to find out where cyanobacteria takes nitrogen at temperatures that are not suitable for the growth of nitrogen-fixing cyanobacteria \citep{Steunou2006}.

    One of the most important are the so-called nitrogen-fixing symbioses - cooperation with plant organisms capable to transfer nitrogen from the atmosphere, or buried in the soil organic matter in plant-available form (ammonium, NH4 +). Two component symbiotic nitrogen-fixing - a terrestrial plant (here, any suitable type of plants) and any bacteria capable of fixing nitrogen. The role of the latter may act cyanobacteria actinobacteria alpha proteobacteria. Because a lack of available nitrogen - the main limiting factor limiting plant growth. Removing this restriction, it would be possible to achieve an enormous increase in productivity \citep{Markov2010}.

    \subsection{From Prokaryotes to Eukaryotes} % (fold)
    \label{sub:from_proka}

    But not all the biochemical processes that could be useful for cell, can be done in a single `common pot', which is the cytoplasm of prokaryotes. Imagine how would complicate the work of the chemist, If he has only one single tube! To overcome this drawback Prokaryotes, of course, tried in their own way. If you look at prokaryotic cell  closely, you will notice that at the disposal of the bacteria is not really one `test tube', but two. In the role of a second is so-called periplasmic space, that is the area outside of the cell membrane - the multi layered cell wall. In the interior of the cell wall chemical processes can occur that are not compatible with those that are in the cytoplasm. But the two tubes it is certainly not enough for a good chemical laboratory! The internal environment of the eukaryotic cell is divided by double and single membranes on a wide variety of sections - `compartments' (nucleus, mitochondria, plastids, endoplasmic reticulum, and so on. D.) \citep{Markov2010}.

    To overcome these limitations, prokaryotic cells needed to make another step - it is quite natural and logical - towards the further strengthening of integration, solidarity community. They were the really merge into a single body, to abandon their cellular individuality and combine their personal chromosome into one big overall genome. That's what happened at the beginning of the Proterozoic eon (probably around 2.0-2.2 billion years ago) \citep{Markov2010}.

    The researchers found microorganisms in the sediments at a depth of 3283 m in the Arctic Ocean  from superkingdom archaea, which are closer to eukaryotes than any other prokaryotes. Based on a set of genes, a new group of microbes, dubbed lokiarhey has many important features of eukaryotes, including actin cytoskeleton and the ability to phagocytosis \citep{Spang2015}.

    \subsection{Climate temperature} % (fold)
    \label{sub:climate_tempre}

    % subsection climate_tempre (end)

    The earliest common ancestors who lived in the time of Early Archean (3.5 billion years ago) were adjusted to a temperature of about 60-70C. The youngest of the late Proterozoic 50 million years ago), preferred a much more cool climate - 37-35C \citep{Gaucher2008}.


    \section{Microorganisms - the main catalyst of reactions} % (fold)
    \label{sec:microorganisms_the_main_catalizator_of_reactions}

        \emph{However, it is becoming increasingly apparent that even in ancient, relatively nondynamic subsurface environments, simplified nonbiological models do not accurately describe or predict important geochemical processes.For example, attempts to describe the distribution of redox reactions in groundwater with equi- librium thermodynamic models are generally unsuc- cessful [Fish, 1993; Hostettler, 1984; Lindberg and Runnells, 1984; Lovley et al., 1994, and references therein]. This is because most of these redox reactions do not take place spontaneouslybut require microor- ganisms to catalyze them. Microorganisms catalyze the redox reactions in order to gain energy for main- tenance and growth. This requirement to conserve energy places biochemical constraints on the path- ways by which these redox reactions are carried out and limits the rate and extent of the microbially cata- lyzed processes.This is why typical subsurfaceenvi- ronments often approach steady state conditions but are generally far removed from redox equilibrium.} from \cite{Lovley1995}

        \emph{Reducers may be organotrophic, using carbon compounds, such as lactate and pyruvate as electron donors, or lithotrophic, and use hydrogen gas (H2) as an electron donor.} from internet.

        \subsection{Why microorganism can live there} % (fold)
        \label{sub:why_microorganism_matter_there}

            \emph{Microorganisms are the only life forms that can inhabit most deep subsurface environments because the typical pore spaces are too small for other types of life [Ghiorse and Wilson, 1988]} \cite{Lovley1995}
        % subsection why_microorganism_matter_there (end)


        \subsection{How microorganisms get energy without light?} % (fold)
        \label{sub:how_we_can_get_energy_without_light_}

            \emph{Light is not available in the deep subsurface,so photosynthesis is not possible.} \cite{Lovley1995}

            \emph{Therefore microbial life is dependent upon energy sources that have been buried in the sediment or enter as dissolved components in the recharge water.} \cite{Lovley1995}

            \emph{These energy sources are organic matter and other reduced compounds(Mn(II), Fe(II), ammonia, sulfide) that can be oxidized with the release of energy} \cite{Lovley1995}


            \emph{Each of these types of metabolism involves a complex series of electron transfer reactions within the microorganisms. The environmentally significant electron transfer is the final electron transfer to an electron acceptor from the environment. This is referred to as the terminal electron accepting process (TEAP). The most common terminal electron acceptors for organic matter oxidation are expected to be 02, nitrate,Mn(IV), Fe(lII), sulfate, and carbon dioxide. Thus the predominant TEAPs in sedimentary environments are 02 reduction (aerobic respiration), nitrate reduction (denitrification and dissimilatory nitrate reduction), Mn(IV) reduction, Fe(III) reduction, sulfate reduction, and carbon dioxide reduction (methane production} \cite{Lovley1995}

            \emph{Thus, at equilibrium, the ADP concentration would be expected to be more than $10^7$ times that of ATP, but in fact, the ATP concentrations in most cells are 10 to 100 times that of ADP. This is a crucial point because it’s the relative concentrations of ATP and ADP that matter.} \cite{Karp2008}

            Shewanella’s outer membrane is full of tiny chemical wires, enabled by specialized proteins, that let it move electricity out of the cell. The wires make direct contact with the manganese oxide, which is how it can deposit electrons and “breathe” a solid substance.

    \subsection{Cannibalism and Spore formation} % (fold)
    \label{sub:cannibalism}

    \emph{Under certain conditions, certain microbes respond to stress in a more elaborate manner, an extreme example of which is endospore formation in Bacillus subtilis. Endospore (or more simply, spore) forma- tion, which is triggered by nutrient limitation, is a complex de- velopmental process that involves the expression of more than 500 genes over the course of 6 to 8 hr (Britton et al., 2002; Eichenberger et al., 2004; Fujita et al., 2005; Molle et al., 2003; Steil et al., 2003). The process culminates in the formation of a resting cell that is capable of resisting en- vironmental extremes and remaining dormant for long pe- riods of time. It might be expected that the decision to com- mit to spore formation is not taken lightly since converting a cell into a spore requires a large investment of time and en- ergy and because the process becomes irreversible after about 2 hr (Dworkin and Losick, 2005; Parker et al., 1996). Indeed, evidence indicates that, under conditions of high cell population density (colonies), the bacterium forestalls committing itself for as long as possible by a process of cannibalism (Gonzalez-Pastor et al., 2003), a central aspect of which is the subject of this report.} \cite{Ellermeier2006}

    \emph{Upon amino-acid deprivation, M. xanthus cells aggregate into local groups of ,100,000 individuals that exchange intercellular signals to con- struct spore-bearing fruiting bodies. Importantly, only a minority of cells survive development after differentiation into stress-resistant spores, and thus proficiency at competition for limited sporulation ‘slots’ can be a major fitness component for distinct genotypes undergoing co-development in chimaeric fruiting bodies} \cite{Fiegna2006}


    \subsection{Symbiosis vs Parasitism} % (fold)
    \label{sub:symbiosis_vs_parasitism}

    \emph{Obligate relationships are common in biology. Higher eukaryotes host a vast array of obligate parasites and mutualists that include both microbes and larger organisms. For example, numerous bacteria are obligate pathogens or symbionts, many bird species are obligate brood parasites9 and a variety of insects are social parasites that rely on workers of another species to raise their offspring.} \cite{Fiegna2006}

    \emph{Myxococcus xanthus is the best characterized species of the soil- dwelling myxobacteria, which are distinguished by cooperative pre- dation on other microbes and social development into awide variety of fruiting body morphologies. Predation is accomplished by swarming packs of cells that secrete toxic and lytic metabolites that kill and degrade prey organisms and thereby generate an extra- cellular, public pool of resources.} \cite{Fiegna2006}


    \subsection{Self-photosensitization of nonphotosynthetic bacteria} % (fold)
    \label{sub:non_phosynthetic_organism_can_become_photosynthetic_in_appropriate_environment}

    Нефотосинтезирующую бактерию можно обучить фотосинтезу, поместив ее в подходящую среду

    Американские химики и биоинженеры создали «гибридную» фотосинтезирующую систему, в которой светособирающую функцию выполняют наночастицы сульфида кадмия, а живая нефотосинтезирующая бактерия Moorella thermoacetica забирает у них возбужденные светом электроны, которые она затем использует для восстановления CO2 и синтеза органики. Ожидается, что подобные биотехнические устройства, преобразующие энергию солнечного света в нужные человеку органические вещества, в дальнейшем превзойдут по эффективности и удобству использования обычные фотосинтезирующие организмы.

    Химики из Калифорнийского университета в Беркли \citep{Zhang:2015ih} сообщили в первом выпуске журнала Science за 2016 год о создании удивительной гибридной фотосинтезирующей системы, состоящей из нефотосинтезирующей бактерии и неорганических наночастиц — светоуловителей, которые бактерия сама осаждает на своей поверхности из раствора.

    Биологическим компонентом гибридной системы является бактерия Moorella thermoacetica. Этот микроб был выбран по нескольким причинам. Во-первых, у него очень пластичный метаболизм. Он может расти и как гетеротроф, питаясь готовой органикой (например, глюкозой), и как хемоавтотроф, самостоятельно производя органику из углекислого газа, причем энергию и электроны, необходимые для фиксации CO2, он может получать путем окисления широкого круга разных органических и неорганических субстратов (см.: H. L. Drake, S. L. Daniel, 2004. Physiology of the thermophilic acetogen Moorella thermoacetica). Эту бактерию можно даже «кормить» электронами прямо с электрода, и она будет их поглощать, использовать для восстановления CO2 и за счет этого расти и размножаться. Микробов, обладающих такой способностью, называют «электротрофами» (см.: D. R. Lovley, 2010. Powering microbes with electricity: direct electron transfer from electrodes to microbes).


    % subsection non_phosynthetic_organism_can_become_photosynthetic_in_appropriate_environment (end)

    \section{Philippe Course}
    \label{sec:Philippe Course}

    \subsection{Biomineralisation}
    \label{sub:Biomineralisation}


    \subsubsection{What are the major groups of biominerals?} % (fold)
    \label{sec:what_are_the_major_groups_of_biominerals_}

    Carbonates and phosphates are the major groups of biominerals.

    % subsubsection what_are_the_major_groups_of_biominerals_ (end)



    \subsubsection{List three reasons why organisms produce biominerals.} % (fold)
    \label{sec:list_three_reasons_why_organisms_produce_biominerals_}

    \begin{itemize}
      \item Support of the body (formation of skeletons, bones);
      \item Protection of the vital organs, defense against predators (production of shells);
      \item Feeding and predation (formation of teeth and attacking limbs).
    \end{itemize}

    % subsubsection list_three_reasons_why_organisms_produce_biominerals_ (end)



    \subsubsection{What are some distinguishing characteristics of biominerals?} % (fold)
    \label{sec:what_are_some_distinguishing_characteristics_of_biominerals_}

    \begin{itemize}
      \item Unusual external morphology;
      \item Crystal domains are delimited by organic layers.
    \end{itemize}

    % subsubsection what_are_some_distinguishing_characteristics_of_biominerals_ (end)



    \subsubsection{What are the main processes involved in biologically controlled biomineralization?} % (fold)
    \label{sec:what_are_the_main_processes_involved_in_biologically_controlled_biomineralization_}

    \begin{itemize}
      \item Nucleation;
      \item Growth;
      \item Creating the shape (morphology);
      \item Allocation.
    \end{itemize}

    % subsubsection what_are_the_main_processes_involved_in_biologically_controlled_biomineralization_ (end)



    \subsubsection{How can biominerals tell us something about past environmental conditions?} % (fold)
    \label{sec:how_can_biominerals_tell_us_something_about_past_environmental_conditions_}

    \begin{itemize}
      \item Oxygen isotopic composition as a proxy for T;
      \item The shape of a crystal reflects the nature of the environment;

    \end{itemize}

    Not mentioned in the paper:

    \begin{itemize}
      \item Sr/Ca as a proxy fer El Niño and similar rapid environmental change events~\citep{PerezHuerta:2013bu};
      \item Ba/Ca as a proxy for dissolved barium in water~\citep{PerezHuerta:2013bu};
      \item Mg/Ca ratio as proxy for temperature~\citep{Nehrke:2013cf};
      \item Nano-structure as a T and P proxies~\citep{Olson:2012hh};
      \item thickness of shell as T proxy (H = 0.322 + 0.012 T) (Pupa Gilbert, not published yet).
    \end{itemize}

    % subsubsection how_can_biominerals_tell_us_something_about_past_environmental_conditions_ (end)



    \subsubsection{What is the vital effect?} % (fold)
    \label{sec:what_is_the_vital_effect_}

    Biological processes, which cause chemical and isotopic compositions of biominerals (or other compounds in general case) to be significantly different from those formed inorganically under the same environmental conditions.


    % section what_is_the_vital_effect_ (end)

    \subsection{Early life}
    \label{sub:Early life}


    \subsubsection{Venus is a “dehydrated” planet, while oceans cover Earth – explain?} % (fold)
    \label{sec:venus_is_a_dehydrated_planet_while_oceans_cover_earth_explain_}

    At some point in the history of the solar system the Venus did lose its all water. The evidence of dehydration of Venus come from isotopic composition of elements in the atmosphere compared to the Earth. If we consider N, C and O they are quite similar; it means that initially planets had quite the same chemical composition; given that they also are so close to each other this is no surprise. But when we look at hydrogen isotopic composition D:H then we can notice great difference; Venus has higher ratio of D:H. This means that at some point Venus lost much more of light isotope of H than the Earth; it is suggested that the hydrogen left Venus due to \textit{runaway greenhouse effect} which can be described by following steps:

    \begin{itemize}
      \item The temperature of Venus surface is higher than 75C; water was evaporated into the atmosphere;
    \begin{equation}
      \ce{H2O(l) -> H2O(g)}
    \end{equation}


      \item Water is a strong greenhouse gas. It catalyzed evaporation even more;

      \item Strong UV radiation break down the water:

    \begin{equation}
      \ce{2H2O(g) ->[h\nu] 2H2(g) + O2(g)}
    \end{equation}

      \item Lightest Hydrogen isotope escaped the atmosphere of Venus.
    \end{itemize}

    Also, the top of the today\'s Earth atmosphere is cold and water falls back to the surface. This effect helps to retain the hydrogen in the atmosphere of the Earth and losses are negligible.


    \subsubsection{Summarize the main consequences of impacts on the early evolution of the inner planets and the emergence of life.} % (fold)
    \label{sec:summarize_the_main_consequences_of_impacts_on_the_early_evolution_of_the_inner_planets_and_the_emergence_of_life_}

    Collision between planetesimals produced the inner planets. Venus, Earth and Mars all received water vapour and carbon, perhaps with early oceans on all three. Having completely different conditions all three planets went their own way.

    Early Mars may have been safer than the early Earth, and Mars had water free on the surface and was possibly habitable during Hadean. There is a chance that Life may have been brought from Mars.

    There is an explanation of the origin of Moon that at some point in history the Earth was struck by the inner planet about the size of the Mars or bigger. This enormous impact formed the Moon and gave the Earth day-night cycle and seasons which are crucial for emergence of Life on Earth.


    \subsubsection{Explain why the presence of isotopically heavy carbon (i.e. enriched in 13C) in carbonate minerals found in old sedimentary rocks may indicate the presence of life?} % (fold)
    \label{sec:explain_why_the_presence_of_isotopically_heavy_carbon_i_e_enriched_in_13c_in_carbonate_minerals_found_in_old_sedimentary_rocks_may_indicate_the_presence_of_life_}

    Biogenic substances shows a preference for \ce{^{12}C} and, therefore, \ce{^{13}C} is retained in the surface reservoir of oxidized carbon, mostly as dissolved bicarbonate~\citep{Schidlowski:1988de}. There is also short-lived radioactive nuclide  \ce{^{14}C} but it occurs only in trace amounts. Therefore, the sedimentary rocks are enriched with inorganic heavy carbon \ce{^{13}C} due to light \ce{^{12}C} is extracted by biosphere.



    \subsubsection{The earliest photosynthetic planktonic organisms were most likely not living in the oceans’ surface waters (as is the case today). Why not?} % (fold)
    \label{sec:the_earliest_photosynthetic_planktonic_organisms_were_most_likely_not_living_in_the_oceans_surface_waters_as_is_the_case_today_why_not_}

    The early Sun was fainter by roughly 30\%. It is very likely that there was the global glaciation. Therefore, it is highly doubtful that photosynthetic planktonic organisms were living in the oceans’ surface waters.


    \subsubsection{In the Late-Archean microbial communities shown on Figure 1 (numbered 1 to 4), upward fluxes of methane (CH4) and molecular hydrogen (H2) are shown. Which organisms produce CH4, which ones produce H2? Methane can be biosynthesized from H2 and CO2: write a reaction describing this process. Other microorganisms can obtain energy from the oxidation of CH4 – how do we call these microorganisms? In the absence of oxygen, sulfate can act as the electron acceptor (oxidant) for CH4 oxidation. Write a reaction describing this metabolic pathways.} % (fold)

    \begin{itemize}
      \item Methanogens produce \ce{CH4};
      \item Fermenters and some algae\citep{Lee:2010gx} produce \ce{H2};
      \item Methane can be biosynthesized: \ce{H2 + CO2 -> CH4 + H2O};
      \item Methanotrophs can obtain energy and carbon from the oxidation of \ce{CH4};
      \item In the absence of oxygen, sulfate can act as the electron acceptor (oxidant) for \ce{CH4} oxidation: \ce{CH4 + SO4^{2-} -> HCO3- + HS- + H2O}
    \end{itemize}


    \subsubsection{What are cyanobacteria? Purple S bacteria? Mesophiles?} % (fold)
    \label{sec:what_are_cyanobacteria_purple_s_bacteria_mesophiles_}

    \textbf{Cyanobacteria}. `Inventor' of oxygenic photosynthesis. They used to be called blue-green algae. Cyanobacteria consumes light across the visible range, including the green light.

    Some biologists say, using metaphorical language, that the plants are just comfortable `houses' for living of cyanobacteria.

    \textbf{Purple S bacteria}. The purple sulfur bacteria are capable of anaerobic photosynthesis. The ancestor of the purple bacteria is alpha proteobacteria.

    \textbf{Mesophile}. Organism that prefer the moderate temperature of the environment in the range from 20C to 45C.

    \textbf{Ancient bacterial mats}. It was a three-layered bacterial mat, almost the same as modern bacterial mats, with the difference that the top layer is not formed by oxygen (oxygenic) but anoxic photosynthetic bacteria. These were the ancestors of cyanobacteria, which has not learned yet how to use water as an electron donor. \textbf{The second layer} is formed by another anoxic photosynthetic bacteria - alpha proteobacteria. This pink creation and today live in bacterial mats beneath the cyanobacteria, because they consume longer wavelength light, which easily passes through the top layer of the green community. Fermenting bacteria were in the \textbf{third layer}. It fermented excess of organics fallen from the upper layers and produced the molecular hydrogen, which is used for the reduction of sulfates by sulfate reducers. As a result of their symbiosis they replenish stocks of hydrogen sulfide required by two upper layers. Methanogens also were here. They use hydrogen produced by fermenters for reduction of carbon dioxide and methane synthesis~\citep{Markov2010}.

    % subsubsection what_are_cyanobacteria_purple_s_bacteria_mesophiles_ (end)
    \subsubsection{The “Tree of Life”, or the universal phylogenetic tree, is based on vertical gene transfer. Horizontal gene transfer is also possible, however. Define vertical and horizontal gene transfer.} % (fold)
    \label{sec:the_tree_of_life_or_the_universal_phylogenetic_tree_is_based_on_vertical_gene_transfer_horizontal_gene_transfer_is_also_possible_however_define_vertical_and_horizontal_gene_transfer_}

    Vertical gene transfer in which genes are transfered during the process of reproduction ('from parents to children').

    Horizontal gene transfer refers to the transfer of genes between organisms via direct contact of cells or by a virus.

    \subsubsection{The RNA-world hypothesis requires an early habitat for life rich in phosphorus. Why?} % (fold)
    \label{sec:the_rna_world_hypothesis_requires_an_early_habitat_for_life_rich_in_phosphorus_why_}

    \cite{LeslieE:2004jz} has outlined the major steps that are required to establish an RNA world biotic system: synthesis of sugars, nucleoside synthesis, phosphorylation, formation of long single-stranded polynucleotides, separating and copying of double-stranded polynucleotides and yielding a second ribozyme molecule. Repetition of this process would lead to an exponentially growing population~\citep{Pasek:2005je}.

    Phosphate is essential molecule for RNA world biotic system because it \textbf{serves as a bridge} in large biopolymers such as RNA.


    \subsubsection{Photosynthesis requires an electron donor. Which electron donors are used in anoxygenic photosynthesis? What is the electron donor in oxygenic photosynthesis? (Identify not only the compound, but also the element in the compound donating electrons.)} % (fold)
    \label{sec:photosynthesis_requires_an_electron_donor_which_electron_donors_are_used_in_anoxygenic_photosynthesis_what_is_the_electron_donor_in_oxygenic_photosynthesis_identify_not_only_the_compound_but_also_the_element_in_the_compound_donating_electrons_}

    In the photosynthesis the electron acceptor is carbon in carbon dioxide \ce{C^4+O2}.

    During anoxygenic photosynthesis organisms use a variety of electron donors:

    \begin{itemize}
      \item hydrogen: \ce{H^0_2 -> 2 H+ + 2e^-}
      \item hydrogen sulphide: \ce{H2S^2-  -> S^6+O4 + 8e^-}
      \item sulfur: \ce{S^0  -> S^6+O4 + 6e^-}
      \item some organic compounds: \ce{C^0H2O -> C^4+O2 + 4e^-}
    \end{itemize}


    Oxygenic photosynthesis uses water \ce{H2O} as electron donor: \ce{2H2O^2- -> 2H+ + O^0_2 + 4e^-}


    \subsubsection{The advent of cyanobacteria brought about an enormous expansion of the biosphere. What capability of cyanobacteria in addition to oxygenic photosynthesis made this possible?} % (fold)
    \label{sec:the_advent_of_cyanobacteria_brought_about_an_enormous_expansion_of_the_biosphere_what_capability_of_cyanobacteria_in_addition_to_oxygenic_photosynthesis_made_this_possible_}
    The most important turning point in the development of life was the invention of oxygenic photosynthesis, whereby began to accumulate in the atmosphere and oxygen made possible of existence of higher organisms. The Great Oxidation Event occurred around  2.5-2.7 billion years ago (although some scientists believe in more earlier appearance of photosynthetic oxygen).

    The anoxic community (anoxic bacterial mats) was quite stable and could easily exist in this form for hundreds of million years. But when the blue-green revolutionaries `invented' oxygenic photosynthesis and began to produce oxygen the peaceful life came to an end. For all the ancient forms of life on earth \textbf{oxygen was a dangerous poison}. Even cyanobacteria itself was not very comfortable to live in a poisoned environment. After all, without it earth still would remain `the planet of microbes'. Fortunately for cyanobacteria, they quickly found a way to neutralize the \textbf{toxic products of its own life}.

    Another important advent is the ability to \textbf{fix the nitrogen}. Cyanobacterium Synechococcus manage to combine in its single cell photosynthesis and nitrogen fixation, separating them in time. During the day they photosynthesize, but at night, when there is no light, photosynthesis stops and the concentration of oxygen in the cyanobacterial mat decreases, they switch to nitrogen fixation~\citep{Steunou2006}.


    \textbf{Some remarks:}
    \begin{enumerate}
      \item How the transition from anoxic photosynthesis to oxygen happened? Back in 1970, theoretical model has been proposed according to which the transition was realized through an intermediate stage where nitrogen compounds served as electron donors~\citep{Olson1970}. Only in 2007 the nitrogen photosynthesis (the intermediate stage on the way to the oxygen photosynthesis) has finally been discovered. The discovery was made during the study of microbes that live in fresh water and sewage lagoons. Microbiologists from the University of Konstanz (Germany) grew bacteria in anoxic conditions in the light in a medium with a small amount of nitrite (\ce{NO2-}). After a few weeks in 10 out of 14 samples the pink color became noticeable which is the characteristic of bacteria practicing anoxic photosynthesis, and the oxidation of nitrite to nitrates was registered(\ce{NO3-})~\citep{Griffin2007}.

      \item The main problem faced by nitrogen-fixing cyanobacteria is that the key enzymes of \textbf{nitrogen fixation - nitrogenase - can not work in the presence of oxygen}, which is released during photosynthesis. Until recently, scientists believed that combination of photosynthesis and nitrogen fixation in the same cell is not possible. However, recent studies have shown that we are greatly underestimating the metabolic ability of cyanobacteria.
    \end{enumerate}

    \subsubsection{What are the functions of Rubisco and nitrogenase?} % (fold)
    \label{sec:what_are_the_functions_of_rubisco_and_nitrogenase_}

    Rubisco - carbon fixation enzyme. This enzyme preferentialy captures the light carbon \ce{^12C} from carbon dioxide \ce{CO2} during the Calvin cycle and produces molecular oxygen \ce{O2}.

    Nitrogenase - nitrogen fixation enzymes. Captures nitrogen from nitrogen gas \ce{N2} and bond it to three hydrogen atoms to form ammonia \ce{NH3}.

    These enzymes captures the inorganic C and N convert them into useful in biosphere substances. These are the key enzymes by which inorganic substances enter the biosphere.

    \subsection{Global Fluxes of Chemical Species}
    \label{sub:Global Fluxes of Chemical Species}


    \subsubsection{Atmospheric Sources and Sinks} % (fold)
    \label{sec:athmospheric_sources_and_sinks}

    \subsubsection{Carbon Dioxide} % (fold)
    \label{ssub:carbon_dioxide}

    \begin{tabularx}{\textwidth}{m{200pt}|m{200pt}}
    \textbf{Sources} & \textbf{Sinks} \\ \hline
    Combustion & Oceans\\
    Land-Use & Terrestrial biosphere \\
    \end{tabularx}

    \subsubsection{Methane} % (fold)
    \label{ssub:carbon_dioxide}

    \begin{tabularx}{\textwidth}{p{200pt}|p{200pt}}
    Fossil carbon sources, i.e coal-mining operations, and oil and natural gas production, transmission, distribution and use & \ce{OH-} radical in the troposphere\\
    Wetlands, Lakes & Photochemical removal in the stratosphere \\
    Rice paddies & Uptake by soils \\
    Animal waste and domestic sewage \\
    \end{tabularx}


    \subsubsection{Nitrous Oxide} % (fold)
    \label{sub:subsubsection_name}

    \begin{tabularx}{\textwidth}{m{200pt}|m{200pt}}
     Cultivated and tropical soils & Stratospheric photo-dissociation and stratospheric photo-oxidation\\
    Forrests & Removal by soils \\
    Acid production \\
    Combustions
    \end{tabularx}

    \subsubsection{Ozone} % (fold)
    \label{sub:ozone}

    \begin{tabularx}{\textwidth}{m{200pt}|m{200pt}}
    Production of tropospheric ozone by carbon monoxide, hydrocarbons, and oxides of nitrogen & Reduction of stratospheric ozone by chlorine- and bromine-containing chemicals \\
    \end{tabularx}

    \subsubsection{Chlorofluorocarbons} % (fold)
    \label{sub:cfcs}

    \begin{tabularx}{\textwidth}{m{200pt}|m{200pt}}
    Production of aluminium & Photolysis in the stratosphere \\
    Use in electrical equipment &  \\
    \end{tabularx}

    \subsubsection{Fluxes to and from the Atmosphere} % (fold)
    \label{sec:fluxes_to_and_from_the_atmosphere}

    \subsubsection{Carbon Dioxide} % (fold)
    \label{ssub:carbon_dioxide}

    \begin{tabularx}{\textwidth}{m{145pt}|m{145pt}|c}
    \textbf{From} & \textbf{To} & \textbf{Ref} \\ \hline
    $3.9\pm1.4$ Gt \ce{C} per year & $17.5\pm50$\% Tg \ce{C} per year & \cite{Watson:1992ty}  \\
    \end{tabularx}


    \subsubsection{Methane} % (fold)
    \label{ssub:carbon_dioxide}

    \begin{tabularx}{\textwidth}{m{145pt}|m{145pt}|c}
    381 Tg \ce{C} per year & 394 Tg \ce{C} per year  & \cite{Watson:1992ty}  \\
    \end{tabularx}

    \subsubsection{Nitrous Oxide} % (fold)
    \label{sub:subsubsection_name}

    \begin{tabularx}{\textwidth}{m{145pt}|m{145pt}|c}
    $5.18\pm16.1$ Tg \ce{N} per year & $10\pm17.5$ Tg \ce{N} per year  & \cite{Watson:1992ty}  \\
    \end{tabularx}

    \subsubsection{Ozone} % (fold)
    \label{sub:ozone}


    \begin{tabularx}{\textwidth}{m{145pt}|m{145pt}|l}
    3310 Tg \ce{O3} per year &  3170 Tg \ce{O3} per year & \cite{Lelieveld:2000to} \\
    4900 Tg \ce{O3} per year &  4300 Tg \ce{O3} per year & \cite{Bey:2001do} \\
    4895 Tg \ce{O3} per year &  4498 Tg \ce{O3} per year & \cite{Sudo:2002jl} \\
    5260 Tg \ce{O3} per year &  4750 Tg \ce{O3} per year & \cite{Horowitz:2003kg} \\
    4560 Tg \ce{O3} per year &  4290 Tg \ce{O3} per year & \cite{vonKuhlmann:2003ch} \\
    4486 Tg \ce{O3} per year &  3918 Tg \ce{O3} per year & \cite{Bauer:2004kx} \\
    4090 Tg \ce{O3} per year &  3850 Tg \ce{O3} per year & \cite{Wild:2004hb}\\
    4980 Tg \ce{O3} per year &  4420 Tg \ce{O3} per year & \cite{Stevenson:2005ev}\\

    4436 Tg \ce{O3} per year &  3890 Tg \ce{O3} per year & \cite{Folberth:2006gx}\\
    5060$\pm$570 Tg \ce{O3} per year&  4560$\pm$720 Tg \ce{O3} per year & \cite{Stevenson:2006el}\\
    & $2400\pm600$ Tg \ce{O3} per year  & \cite{Ganzeveld:2009de}  \\

    \end{tabularx}

    \subsubsection{Chlorofluorocarbons} % (fold)
    \label{sub:cfcs}
    \begin{tabularx}{\textwidth}{m{145pt}|m{145pt}|l}
    & 40$\pm$2 pptv per year  & \cite{Watson:1992ty}  \\

    \end{tabularx}

    \subsubsection{Solution for lecture example} % (fold)
    \label{sec:solution_for_lecture_example}

    \begin{equation}
      \frac{ dn}{ dt} = a - k\cdot n \text{~~~~~~~~~with initial value:~~~~~} n(0)=n_0
    \end{equation}

    \begin{equation}
      \frac{dn}{dt} = -k ( n- \frac{a}{k} )
    \end{equation}

    \begin{equation}
      \frac{dn}{n- \frac{a}{k}} = -kdt
    \end{equation}

    \begin{equation}
      \ln |n- \frac{a}{k}| = -kt + C
    \end{equation}

    \begin{equation}
      n- \frac{a}{k} = C e^{-kt}
    \end{equation}

    \begin{equation}
      n_0 = \frac{a}{k} + Ce^0 => C = n_0 - \frac{a}{k}
    \end{equation}

    \begin{equation}
      n(t) = \frac{a}{k} + \left( n_0 - \frac{a}{k} \right) e^{-kt}
    \end{equation}

    \subsection{Vascular plants}
    \label{sub:Vascular plants}


    \subsubsection{According to Berner, vascular plants retard erosion but enhance weathering – isn’t this a contradiction? Explain why it isn’t.} % (fold)
    \label{sec:according_to_berner_vascular_plants_retard_erosion_but_enhance_weathering_isn_t_this_a_contradiction_explain_why_it_isn_t_}

    It is not a contradiction because plants anchor clay-rich soil allowing the retention of water and nothing is moved away, stays at place; therefore, it is enhanced weathering but not erosion.

    \subsubsection{If the rise of plants during the Devonian accelerated the consumption of CO2, why didn’t atmospheric CO2 drop to zero?} % (fold)
    \label{sec:if_the_rise_of_plants_during_the_devonian_accelerated_the_consumption_of_co2_why_didn_t_atmospheric_co2_drop_to_zero_}

    \textbf{Regarding sinks of \ce{CO2}}: due to \ce{CO2} is the greenhouse gas lower \ce{CO2} make lower global temperatures and less river runoff which decreases the weathering rates. Also, if the decrease of \ce{CO2} is high enough, we could even have glaciation; and as a result, dead or transfered to dormant state plants and decrease of consumption.

    \textbf{Regarding sources of \ce{CO2}}: we still have weathering of ancient organic matter rocks and degasing from deep sedimentary rocks and organic matter which will tend to restore the level of \ce{CO2}.

    \subsubsection{How is the carbon isotopic composition of CaCO3 or carbonate impurities of goethite in paleosols related to the atmospheric level of CO2? [This question will require you do dig into the literature.]} % (fold)
    \label{sec:how_is_the_carbon_isotopic_composition_of_caco3_or_carbonate_impurities_of_goethite_in_paleosols_related_to_the_atmospheric_level_of_co2_this_question_will_require_you_do_dig_into_the_literature_}


    As shown by \cite{Yapp:1992ca}, \cite{Yapp:1996eh}, \cite{Yapp:1996jq} and \cite{Schroeder:1999jd} the measured mole fractures (X) and $\delta^{13}C$ values of the \ce{Fe(CO3)OH} component in goethite have linear dependence:

    \begin{equation}
      \delta^{13}C = 0.0162 \frac{1}{X} - 20.1
    \end{equation}

    with high correlation r of 0.98. \cite{Yapp:1987cv} showed that thermodynamically formulated relationship which has the dependence with atmospheric partial pressure of \ce{CO2}:

    \begin{equation}
      P_{\ce{CO2}} = \log X + 6.04 - \frac{1570}{T}
    \end{equation}

    And, therefore, knowing the average temperature of the ancient period, for instance, from oxygen isotopic composition we can estimate the atmospheric partial pressure of \ce{CO2}.

    \subsubsection{What mechanism in addition to enhanced silicate weathering explains the very low CO2 levels during the late Carboniferous?} % (fold)
    \label{sec:what_mechanism_in_addition_to_enhanced_silicate_weathering_explains_the_very_low_co2_levels_during_the_late_carboniferous_}


    As proposed by \cite{Retallack:1997fi} the drop of \ce{CO2} during Devonian-Carboniferous was due to enhanced weathering and enhanced \textbf{burial of organic matter in swamps}.




    \section{Clean Water: The Key to Human Health and Prosperity} % (fold)
    \label{sec:clean_water}

    PVC: As humans, everything we do involves water. We use it to produce our food, extract minerals, generate energy and manufacture goods. In fact, there’s no product on the market that does not have a water footprint."What is becoming clearer and clearer is that degradation of water quality is probably the most pervasive, global threat to human health and human prosperity"

    In Canada, there is a perception that there is so much fresh water that we do not need to worry about the supply. Professor Philippe Van Cappellen, ecohydrology researcher at the University of Waterloo, worries about it. He’ll tell you that it is becoming clearer and clearer that degradation of water quality is probably the most pervasive, global threat to human health and human prosperity. It is currently estimated that 90\% of cities in China are dealing with ground water contamination issues. In a country where 70\% of the drinking water is supplied by ground water, this is a very important problem. It is also a problem closer to home – areas in Canada, like Southern Ontario, Saskatchewan, and Southern Alberta, are facing ground water contamination issues.

    Traditionally ecohydrology was the study of the interactions between the water cycle and natural ecosystems. However, as we are dealing more and more with human dominated ecosystems like those created in agricultural and urban areas, ecohydrology has shifted to looking at the world as a collection of socio-ecological systems where humans are an integral part of the ecosystem. Prof. Van Cappellen’s group is trying to determine how to balance the need for clean water for humans with the need for enough water of good quality for natural ecosystems. Gaining an understanding at a fundamental level, at the lab-bench level or in the field, on how natural processes eliminate contaminants from the environment, can lead to the development of new green technologies or engineered environments for water treatment and water conservation.

    The impact of this type of research is far reaching. Essentially, in many areas of the world, development is limited by the availability of clean drinking water. If we can increase the availability of clean water, we can automatically generate economic prosperity. In terms of generating jobs and having a direct economic impact, there is a lot of potential in water technology. Canada with all its water and its large water-related research community, has the possibility to be a frontrunner in that area.

    \section{Surface water} % (fold)
    \label{sec:surface_water}

    \subsection{Rivers} % (fold)
    \label{sub:rivers}



    \section{Groundwater} % (fold)
    \label{sec:groundwater}

    \subsection{Classification} % (fold)
    \label{ssub:classification}

        \emph{Groundwater flow systems(Figure 1) can be classi- fied as local, intermediate, or regional [Toth, 1963]. Local flow systemsrechargeat a topographichigh and dischargein an adjacent topographic low. There is a high degree of connectednesswith the surface. Local flow systems typically have high rates of recharge (1-30 cm/yr) and high rates of groundwater flow (1-100 m/yr). Rates of recharge and water levels respond rapidly to individual precipitation events in local flow systemsbecausethey are sointimately connectedwith the surface. Intermediate flow systems recharge and discharge in areas that are separated by one or more topographichighs. Because of this, intermediate flow systemsare much less connected with the surface, and ratesofrecharge(0.01-1 cm/yr)andgroundwaterflow (0.1-1 m/yr) are proportionally lower. In intermediate flowsystems,waterlevelsandratesofrechargedonot respond to individual precipitation events. Regional flow systemsrecharge only at the groundwater divide, and they discharge only at the bottom of the basin. Recharge rates in regional flow systemsare very low, and flow is often sluggish.There is little connection between surface environments and regional flow sys- tems.} \cite{Lovley1995}

        \subsubsection{Deep subsurface} % (fold)
        \label{ssub:deep_subsurface}
            \emph{We suggest that the use of the term "deep sub- surface" in microbiological studies should be re- stricted to intermediate and regional flow systems. This definition is independent of total depth and instead depends entirely on the hydrologic frame- work of the system under study}.\cite{Lovley1995}


            \emph{The typesof microbial metabolism that are found in the deep subsurface are likely to be similar to those found in surface sedimen- tary environments [Pedersen, 1993]} \cite{Lovley1995}

            \emph{Because of the lack of photosynthesis the only source of 02 in the deep subsurface is 02 that enters dissolved in the recharge water.} \citep{Lovley1995}

            \emph{\ce{O2} is expected to be found in deep aquifers only when there is little or no microbial activity,such as is seen in some sedimentary basins of the American Southwest where the lack of organic carbon has allowed oxygen to persist in groundwater that is more than 10,000 years old [Winograd and Robertson, 1982].} \citep{Lovley1995}

            \emph{ Like aerobic respiration, nitrate reduc tion is expected to be a minor process for organic matter oxidation in most deep subsurface environments because nitrate inputs from groundwater are low and there are no significant mineral sourcesof nitrate.} \cite{Lovley1995}
        % subsubsection deep_subsurface (end)

        % subsection classification (end)
    % section groundwater (end)

    \section{Extreme limitted energy environments} % (fold)
    \label{sec:extreme_limitted_energy_environments}

    \subsection{Slow Reaction Rates} % (fold)
    \label{sub:reaction_rate}


    p.170: Despite the extremes in temperature, pressure, pH and other compositional factors that microorganisms in marine sediment and many other environments face, most share the ability to live under very low energy conditions (Jørgensen and Boetius, 2007; Jørgensen, 2011; Jørgensen, 2012). One of the consequences of living in such low energy systems is long turnover times compared to their surface-dwelling analogs (Thorn and Ventullo, 1988; Parkes and others, 1990; Konopka, 2000; D’Hondt and others, 2002). For example, bacteria in aquifers (Phelps and others, 1994), sedimentary rocks (Chapelle and Lovley, 1990; Phelps and others, 1994), marine (D’Hondt and others, 2004; Schippers and others, 2005; Jørgensen and D’Hondt, 2006; Lomstein and others, 2012; Røy and others, 2012) and freshwater sediments (Thorn and Ventullo, 1988) and ice cores (Price and Sowers, 2004) appear to have turnover times exceeding 1000 years. Even photosynthetic microbial communities in Antarctica appear to have biomass turnover times of up to 19,000 years (Johnston and Vestal, 1991). Taken together, a significant proportion of the world’s microorganisms seem to be adapted to not only low-energy fluxes, but extremely low metabolic activity. Although the relationships between energy supply, energy demand, and the rates of microbially catalyzed processes for most subsurface organisms are unclear (Jørgensen, 2011) -- Highlighted jun 29, 2015 \citep{LaRowe:2015dt}

    p.171: The catabolic rates vary over twelve orders of magnitude from 6.0 􏰭 10􏰬9 to 6.66 􏰭 103 nmol cm􏰬3 day􏰬1 (the reactions are typically reported per mole of a substrate being processed, although sometimes it’s given per mole produced). -- Highlighted jun 29, 2015 \citep{LaRowe:2015dt}

    \subsection{Reaction energetics} % (fold)
    \label{sub:reaction_energetics}


    p.174: Reaction energetics.—The amount of energy available from catabolic reactionsvaries as a function of the temperature, pressure and composition of the system ofinterest. Within the marine realm, this has been demonstrated in hydrothermalsystems (Shock and others, 1995; McCollom and Shock, 1997; McCollom, 2000; Shockand Holland, 2004; McCollom, 2007; LaRowe and others, 2008; Amend and others,2011; LaRowe and others, 2014), ocean sediments (Schrum and others, 2009; Wangand others, 2010; LaRowe and Amend, 2014; Teske and others, 2014) and basementrock (Bach and Edwards, 2003; Cowen, 2004; Edwards and others, 2005; Boettger andothers, 2013) -- Highlighted jun 29, 2015 \citep{LaRowe:2015dt}

    p.177: Only a portion of catabolic power is used to generate biomass. The remainder is used for any number of cellular activities including shifts in metabolic pathways, energy spilling reactions, cell motility, changes in stored polymeric carbon, osmoregulation, extracellular secretions, defense against chemical stresses and the “proofreading”, synthesis and turnover of macromolecules such as proteins and RNA (van Bodegom, 2007) -- Highlighted jun 29, 2015 \citep{LaRowe:2015dt}

    \subsection{Maintenance Energy} % (fold)
    \label{sub:maintenance_energy}

    p.177: Values of maintenance energies that have been reported in the literature, summarized below, are typically determined under relatively high-energy conditions (Hoehler and Jørgensen, 2013) -- Highlighted jun 29, 2015 \citep{LaRowe:2015dt}

    p.177: On the other end of the maintenance energy spectrum is the basal power requirement, which has been defined as “the energy flux associated with the minimal complement of functions required to sustain a metabolically active state” (Hoehler and Jørgensen, 2013) -- Highlighted jun 29, 2015 \citep{LaRowe:2015dt}

    p.177: Many studies have attempted to quantify maintenance energy/power to determine how much energy microorganisms use for purposes other than growth (see review by van Bodegom, 2007) -- Highlighted jun 29, 2015 \citep{LaRowe:2015dt}

    p.177: In addition, it has been shown that maintenance coefficients depend on growth conditions (Neijssel and Tempest, 1976; Chesbro and others, 1979; Goma and others, 1979; Beyeler and others, 1984), and therefore should be captured by a variable, not a constant.Furthermore, models that define a maintenance supply (Schulze and Lipe, 1964) or maintenance coefficient (Pirt, 1965) as the minimum substrate consumption to maintain a cell ignore the fact that the amount of energy that can be supplied by a single substrate (generally defined as the electron donor) depends strongly on the identity and concentration of the complementary electron acceptor and the temperature, pressure and composition of the environment. -- Highlighted jun 29, 2015 \citep{LaRowe:2015dt}

    p.177: In order to remedy these shortcomings, van Bodegom (2007) proposes a new formulation for a single maintenance coefficient that encompasses all energy expenditures that do not lead to growth. However, this model may be prohibitively complex for application to natural systems. -- Highlighted jun 29, 2015 \citep{LaRowe:2015dt}

    p.178: but in deep settings, metabolic rates can be at least 4 orders of magnitude lower than in surface samples (Price and Sowers, 2004) and per-cell energy fluxes appear to be several orders of magnitude lower than maintenance energies predicted from laboratory cultures (Jorgensen, 2012) -- Highlighted jun 29, 2015 \citep{LaRowe:2015dt}

    \subsection{Growth Rate and Cell yield} % (fold)

    p.182: The yield coefficients that are based on standard state Gibbs energies share the same shortcoming as those that relate grams of biomass produced per mole or gram of substrate: neither approach takes into account the diversity of natural habitats. In particular, the amount of energy available per mole of a given substrate can vary substantially from one environment to another, especially when comparing the laboratory to the deep biosphere, where most microorganisms exist under very lower energy states (Jørgensen and D’Hondt, 2006; Jørgensen and Boetius, 2007; Jørgensen, 2012) -- Highlighted jun 29, 2015 \citep{LaRowe:2015dt}

    p.182: Using a fixed value for biomass yield would predict the same amount of biomass produced by organisms catalyzing the same reaction in two very different environments, despite the large difference in energy availability in these settings. -- Highlighted jun 29, 2015 \citep{LaRowe:2015dt}

    p.183: Perhaps because so many of the studies summarized in table 6 are directed towards optimizing the industrial application of microorganisms, there is little to no focus on how environmentally relevant variables affect these yield values. -- Highlighted jun 29, 2015 \citep{LaRowe:2015dt}

    \section{Why do we need the model} % (fold)
    \label{sec:why_do_we_need_the_model}

    \emph{The resulting RTMs are ideal diagnostic tools for the study of the diagenetic dynamics, as they explicitly represent the coupling and interactions of the processes involved (e.g. Van Cappellen and Wang, 1996; Berg et al., 2003; Jourabchi et al., 2005; Arndt et al., 2006).} \cite{Arndt2013}

    \emph{RTMs allow extracting biogeochemical reaction rates from easily obtainable pore water profiles. In addition, they can complement organic matter deg- radation rates determined from direct measurements or extracted from the analysis of pore water profiles. For instance, field observa- tions primarily target shallow sediment depths and fast-decaying ma- terials, although it is well known that the most refractory compounds may degrade over much longer distances on geological timescales (Middelburg, 1989)} \cite{Arndt2013}

    \emph{RTMs offer a means to bridge such a large spec- trum of spatial and temporal scales and to reconstruct the sediment history in the context of a system that evolves over years to thou- sands or millions of years as well as over centimeters to tens of me- ters or kilometers (e.g. Zabel and Schulz, 2001; Riedinger et al., 2006; Arndt et al., 2009; Dale et al., 2009).} \cite{Arndt2013}

    \emph{The lack of mechanistic understanding of organic matter degradation is reflected in the mathematical formulation used to represent this process in RTMs. Generally, existing organic matter degradation models do not reflect the complex interplay of factors that may con- trol organic matter degradation on different scales. As a consequence, model parameters derived to fit observed pore water and sediment profiles implicitly account for the neglected factors. The implicit na- ture of model parameters thus complicates the transferability and predictive capability of existing approaches in data poor areas. Yet, the predictive capability of organic matter degradation models plays a key role for the evaluation of deep carbon cycling, the estimation of substrate fluxes to the deep biosphere (e.g. Arndt et al., 2006; Røy et al., 2012), the hindcasting and forecasting of the sediment's diagenetic history (e.g. Dale et al., 2008a; Arndt et al., 2009; Reed et al., 2011; Mogollon et al., 2012; Wehrmann et al., 2013), the predic- tion of methane gas hydrate inventories (e.g. Marquardt et al., 2010; Gu et al., 2011; Wadham et al., 2012) and oil reservoirs (e.g. Head et al., 2006) or the extrapolation to regional and global scales (e.g. Archer et al., 2002; Thullner et al., 2009; Krumins et al., 2013)} \cite{Arndt2013}

    \emph{Incorporating the complex interplay of the different factors that may control organic matter degradation and proposing a consis- tent, predictive algorithm for parameterizing their equations that can be applied to the entire spectrum of boundary conditions encountered at the seafloor represents a major challenge for future generations of RTMs. Yet, the rapidly expanding geochemical and mi- crobiological data sets collected in the framework of global monitor- ing programs, such as the Ocean Drilling Program (ODP), as well as the need for a better quantification of the past, present and future benthic carbon turnover in the Earth system calls for the prompt de- velopment of RTMs.} \cite{Arndt2013}

    % section why_do_we_need_the_model (end)


    \section{Diagenetic equation} % (fold)
    \label{sec:diagenetic_equation}

    {\color{red}\textbf{NOTE:}} This chapter is the {\color{red}copypasted} summary of \cite{Boudreau1997Diagenetic,Berner1980Early}.

    \subsection{The standard approach} % (fold)
    \label{sub:the_standard_approach}

        To obtain a meaningful concentration, it is necessary to average over a volume of the size where the concentration has ceased to change because of variations in fine- scale properties. The characteristic dimension of such a volume is not fixed, but should be larger than several grain diameters and much smaller than the scaledefined by the bulk (macroscopic) gradient in the concentration.

        \begin{equation}
        \label{eq:bulk_c}
            \hat{C(x,y,z,t)} = \frac{1}{V_{avg}} \int_{V_{avg}}C(x,y,z,t)dv
        \end{equation}

        where $\hat{C}$ - average bulk concentration \citep{Berner1980Early}. $C$ - microscopic value of the concentration at a point in the porewater or solids, and $V_{avg}$ - averaging volume. Real sampling techniques measure the bulk average or a closely related quantity.


        The averaging in Eq. \ref{eq:bulk_c} can be applied to various sediment properties, e.g. porosity, solute concentrations, solid species concentrations, etc., to obtain a set of continuous sediment properties (fields) that overlap.

        As a first step, a diagenetic model needs a well-defined reference frame. In our case, this means stating the location of the origin and orientation of the spatial axes. Normally, the Cartesian system is used. The consequence of this choice is that the reference frame moves with time if the sediment is accumulating (Berner, 1971). Moreover, \cite{Berner1980Early} states that most diagenetic changes are vertical in sediments

        % subsection the_standard_approach (end)

    \subsection{Total derivatives} % (fold)
    \label{sub:total_derivatives}

        \begin{equation}
            \left( \frac{D\hat{C}}{Dt} \right)_{layer} = \left(\frac{\partial\hat{C}}{\partial t} \right)_x + \frac{dx}{dt} \left(\frac{\partial \hat{C}}{\partial x} \right)_t
        \end{equation}


        The term dx/dt is the velocity of the observed layer as perceived in the coordinate system anchored at the sediment-water interface; thus, dx/dt is the burial velocity of the solids, w:

        \begin{equation}
            \frac{dx}{dt}=w
        \end{equation}

        Steady-state diagenesis relative to the sediment-water interface:

        \begin{equation}
        \label{eq:ss1}
            \frac{\partial \hat{C}}{\partial t} = 0
        \end{equation}

        Consequently:

        \begin{equation}
        \label{eq:ss2}
            \frac{D \hat{C}}{D t} = w \frac{\partial \hat{C}}{\partial x}
        \end{equation}

        Equations (\ref{eq:ss1}) and (\ref{eq:ss2}) mean that there are indeed time changes in concentration during steady-state diagenesis if we follow a chosen layer of sediment, but there are no changes at a depth fixed relative to the interface.

    % subsection total_derivatives (end)

    \subsection{Conservation (Balance) equations} % (fold)
    \label{sub:conservation_eqs}

        Conservation equations are known as the \"Diagenetic Equations\" through the work of \cite{Berner1980Early}. The conservation equation for a chosen species balances what concentration accumulates in the layer between $x-\Delta x/2$ and $x+\Delta x/2$ with that which enters and leaves (fluxes) or is created or destroyed in an arbitrary time period, $\Delta t$.

        \begin{figure}[htbp]
            \centering
            \includegraphics[width=0.45\textwidth]{layer}
            \caption{A small arbitrary layer of sediment used to create the 1-D conservation (diagenetic) equation from \cite{Boudreau1997Diagenetic}.}
            \label{fig:layer}
        \end{figure}

        The mean amount that fluxes across the surface perpendicular to the depth-direction at $x-\Delta x/2$ per unit time is $F(x-\Delta x/2,t)$, while that at $x+\Delta x/2$ is $F(x+\Delta x/2,t)$. If the fluxes are local, then we can express both$F(x-\Delta x/2,t)$ and $F(x+\Delta x/2,t)$ in terms of the value at x, i.e. $F(x,t)$. This is done with Taylor series expansions:

        \begin{equation}
            F(x- \frac{\Delta x}{2},t) = F(x,t) - \frac{\partial F}{\partial x} \frac{\Delta x}{2} + \frac{\partial^2 F}{\partial x^2} \frac{\Delta x^2}{4}
        \end{equation}

        and:

        \begin{equation}
            F(x+ \frac{\Delta x}{2},t) = F(x,t) + \frac{\partial F}{\partial x} \frac{\Delta x}{2} + \frac{\partial^2 F}{\partial x^2} \frac{\Delta x^2}{4}
        \end{equation}

        If $\Delta x$ is kept sufficiently small, then only the first two terms on the right-hand side.

        \begin{equation}
        \label{eq:taylor_1}
            F(x- \frac{\Delta x}{2},t) = F(x,t) - \frac{\partial F}{\partial x} \frac{\Delta x}{2}
        \end{equation}

        and

        \begin{equation}
        \label{eq:taylor_2}
            F(x+ \frac{\Delta x}{2},t) = F(x,t) + \frac{\partial F}{\partial x} \frac{\Delta x}{2}
        \end{equation}


        The total amount of species consumed or produced in the layer during $\Delta t$ by reactions is $\sum R \Delta t \Delta x$, where $\sum R$ is the net time-averaged source or sink caused by all the reactions that affect the bulk concentration. Given that the accumulated mass, $\Delta M$, in the layer in time $\Delta t$ is

        \begin{equation}
            \Delta M = \left[ \hat{C}(x,t+\Delta t) - \hat{C}(x,t) \right] \Delta x
        \end{equation}

        Then

        \begin{equation}
            \left[ \hat{C}(x,t+\Delta t) - \hat{C}(x,t) \right] \Delta x \approx \left( F(x- \frac{\Delta x}{2})-F(x+ \frac{\Delta x}{2}) \right) \Delta t + \sum R \Delta t \Delta x
        \end{equation}

        or with (\ref{eq:taylor_1}) and (\ref{eq:taylor_2}),

        \begin{equation}
            \left[ \hat{C}(x,t+\Delta t) - \hat{C}(x,t) \right] \Delta x \approx - \frac{\partial F}{\partial x} \Delta x \Delta t + \sum R \Delta t \Delta x
        \end{equation}

        Divide by $\Delta t \Delta x$ and let $\Delta t$ and $\Delta x \rightarrow 0$

        \begin{equation}
        \label{eq:balance}
            \frac{\partial \hat{C}}{\partial t} = - \frac{\partial F}{\partial x} + \sum R
        \end{equation}

        This equation is the conservation or balance equation for a generic species in one dimension (depth), which we can recognize as Berner's diagenetic equation.

        It is not possible to solve eq (\ref{eq:balance}) without identifying this bulk concentration and specifying the mathematical forms of the flux.

        The concentration may refer to a solute in the porewaters

        \begin{equation}
            \hat{C} = \varphi C
        \end{equation}

        where C is now the amount or mass per unit volume of porewater, and $\varphi$ is the porosity, which is defined as
        \begin{equation}
            \varphi = \frac{{\mbox{volume of porewater}}}{{\mbox{volume of solid sediment + porewater volume}}}
        \end{equation}

        Conservation of total fluid is obtained by setting $C = \rho_f$, which is the density of the pore fluid. if a species is part of the solids, then

        \begin{equation}
            \hat{C} = (1 - \varphi) B
        \end{equation}

        where B is the amount or mass of the species per unit volume of solids. To obtain conservation of total solids, set $B=\rho_s$, which is the intrinsic density of these solids.

        \subsection{Types of Fluxes} % (fold)
        \label{sub:types_of_fluxes}

        The specification of flux components within F involves the mathematical representation of certain transport processes.

        \begin{figure}[htbp]
            \centering
            \includegraphics[width=0.45\textwidth]{advection_no_compaction}
            \caption{Advection with no compaction. Figure shows the velocity vectors of the porewater, u, and solid sediment, w, at a chosen depth relative to the sediment-water interface.}
            \label{fig:adv_no_comp}
        \end{figure}

        \subsubsection{Advection} % (fold)
        \label{ssub:advection}

            The advective flux, $F_A$ (mass or moles moving through a unit area per unit time), for a solute in a 1-D model is simply

            \begin{equation}
                F_A = \varphi u C
            \end{equation}

            where u is the velocity of the porewater flow in a chosen reference frame. In a 1-D model, this advection can be due to burial, compaction, and/or externally impressed flow (hydrological flow).

            \begin{figure}[htbp]
                \centering
                \includegraphics[width=0.45\textwidth]{adv_with_compaction}
                \caption{Advection with compaction. Porewater advection, u, is smaller than the solid sediment advection, w, when viewed from the sediment-water interface. }
                \label{fig:adv_with_comp}
            \end{figure}


            \begin{itemize}
                \item \textit{Burial and Compaction.} The burial component of porewater advection results from the sediment-water interface moving away from porewater as sediment accumulates. Compaction is the closer packing of sediment particles, caused by the weight of the overlying solid sediment column with a concomitant expulsion of pore fluids.


                During compaction, porewater moves up toward the sediment-water interface when viewed from a sediment grain (Fig. \ref{fig:adv_with_comp} right), but in an accumulating sediment column, it usually still moves downward in a reference frame anchored to that same interface (Fig. \ref{fig:adv_with_comp} left). In the latter case, the porewater velocity downward appears to be smaller than that of the solids.

                \item \textit{Externally Impressed Flow.} Such flows are not the rule in muddy aquatic sediments because of their low permeabilities, but sandy sediments offer much greater opportunities. Externally impressed flow sometimes occurs in lacustrine and near-shore coastal sediments, where aquifers come into contact with bottom sediments.

                Pressure-driven flow is governed by Darcy’s Equation

                \begin{equation}
                    u_x = - \frac{k}{\varphi \mu} \frac{\partial p'}{\partial x}
                \end{equation}

                here k is the permeability, $\mu$ is the dynamic viscosity of the porewater, and $p'$ is the reduced pressure (the difference between the actual and average hydrostatic pressures).
            \end{itemize}

            For a species of the solids, the advective flux, $F_A$, due to burial in 1-D is simply

            \begin{equation}
                F_A = (1-\varphi)wB
            \end{equation}

            where w is the burial velocity of the solids in the chosen reference frame. Solid advection can be attributed to burial, compaction, and biological processes.

            \begin{itemize}
                \item \textit{Biological advection}. The primary example of biologically mediated solid advection is the result of so-called head-down deposit feeders. These burrowing macrofauna have their mouths at depth in the sediment and ingest solid material in a feeding region around their mouths. The ingested sediment is then carried upward through the animal’s gut to the surface, where it is defecated or deposited. The feeding void at depth is filled by collapse and compaction of the sediment, which engenders a downward advection from all depths between the void and the surface.

                The upward motion is sometimes called a biological “conveyor-belt”, and it is modelled as a “nonlocal” transport.
            \end{itemize}

        \subsection{Mathematical algorithms } % (fold)
        \label{sub:algori}

        \emph{Integration of the ordinary differential equations is performed using an implicit multistep BDF algorithm (Ascher and Petzold, 1998). This algorithm is specially effective for stiff ODEs, which are characterized by solutions with rapid variations in some of the variables (Hairer and Wanner, 2010).} \cite{Leal2015}


    \section{Thermodynamics theory} % (fold)
    \label{sec:choosing_reaction_path_model}


        % subsection why_it_works (end)


        \subsection{Gibbs energy of the reaction} % (fold)
        \label{ssub:gibbs_energy}

        The following is the summary of \cite{Denbigh1980}, \cite{Stumm1995}, \cite{Atkins2010} and \cite{Zumdahl2014} with new insights of \cite{LaRowe2011} and...

        Gibbs free energy \citep{Gibbs1873}:

        \begin{equation}
            G = H - T S
        \end{equation}
        H - is the all energy stored in the molecules (chemical bonds), S - the measure of disorder. G - energy which can do work.

        Gibbs energy of the reaction:

        \begin{equation}
        \label{eq:gibbs_free_energy_eq}
            \Delta G = \Delta H - T \Delta S
        \end{equation}

        In the spontaneous reactions $\Delta H < T \Delta S$. if $\Delta H < 0 $ - reaction is exothermic; if $\Delta H > 0 $ - reaction is endothermic;

        Free energy of the reaction at standard state conditions:

        \begin{equation}
            \Delta G^0 = \Delta H^0 - T \Delta S^0
        \end{equation}

        Standard state conditions means partial pressures of any gases involved in reactions is 0.1 MPa and the concentrations of all aqueous solutions  are 1 M under the Temperature 25C (298.15K).

        \emph{For biochemical reactions, the conditions are arbitrarily set at 25􏱼C (298 K) and 1 atm of pressure, with all the reactants and products present at 1.0 M concentration, except for water, which is present at 55.6 M, and H􏱻 at $10^{-􏱺7}$ M (pH 7.0). 􏰁$\Delta G^{\circ\prime}$􏱼􏱾 indicates standard conditions include pH 7, whereas 􏰁$\Delta G^{\circ}$􏱼 indicates standard conditions in 1.0 M H􏱻 (pH 0.0). The designation $K^{\prime}_􏱾{eq}$ also indicates a reaction mixture at pH 7} \cite{Karp2008}.

        The Gibbs energy of the reaction:

        \begin{equation}
            \ce{Reactant_I + Reactant_{II} <=> Product_I + Product_{II}}
        \end{equation}

        Can be estimated using (\ref{eq:gibbs_free_energy_eq}). Enthalpy of reaction:

        \begin{equation}
            \Delta H^0 = \sum_{i}^{products} {\nu}_i \Delta H_f^0 - \sum_{j}^{reactants} \nu_j \Delta H_f^0
        \end{equation}

        with $H_f^0$ - the standard enthalpy of formation, $\nu$ - stoichiometric coefficient. And entropy:

        \begin{equation}
            \Delta S^0 = \sum_{i}^{products} {\nu}_i \Delta S^0 - \sum_{j}^{reactants} \nu_j \Delta S^0
        \end{equation}

        where $S^0$ - the absolute standard entropy.


        Another approach is to use standard-state free energy of formation:

        \begin{equation}
            \Delta G^0 = \sum_{i}^{products} {\nu}_i \Delta G_f^0 - \sum_{j}^{reactants} \nu_j \Delta G_f^0
        \end{equation}

        where $G_f^0$ - the standard Gibbs free energy of formation.

        \subsection{Gibbs free energy at given temperature.} % (fold)
        \label{sec:gibbs_free_energy_at_given_temperature_}

        Gibbs free energy at given temperature can be estimated using Gibbs-Helmholtz equation:

        \begin{equation}
        \Delta H = \left(\frac{\partial \left(\frac{\Delta G}{T}\right)}{\partial \left(\frac{1}{T}\right)}\right)_P
        \end{equation}

        Making the assumption $\Delta H^{0}$ is constant, rearrange and integrate both sides:

        \begin{equation}
        \int_{T_1}^{T_2} d \left( \frac{\Delta G^0}{T} \right) \approx \Delta H^0(p) \int_{T_1}^{T_2} d \left(\frac{1}{T}\right)
        \end{equation}

        where $T_1$ - temperature at the standard state, $H^0$ - is standard enthalpy, $G^0$ - standard Gibbs free energy. The Gibbs free energy at temperature $T_2$ can be estimated:

        \begin{equation}
        \Delta G_f^0 (T_2) \approx T_2 \cdot \left[ \frac{\Delta G_f^0(T_1)}{T_1} + \Delta H_f^0(p) \left(\frac{1}{T_2}-\frac{1}{T_1}\right) \right]
        \end{equation}


        \subsection{NOSC} % (fold)
        \label{sub:nosc}

        Good summary of the extensive thermodynamic database that exists for a broad range of aqueous, crystalline, liquid and gaseous organic compounds was done by \cite{LaRowe2011}. Based on this, they represented empirical relation between nominal oxidation state of carbon (NOSC) and standard molal Gibbs energy of oxidation half reaction of organic compounds:

         \begin{equation}
             \Delta G_{Cox}^0 = 60.3 - 28.5 \cdot NOSC
         \end{equation}

         which for general oxidation half reaction of organic matter reaction written by \cite{LaRowe2011} as following:

         \begin{equation}
         \begin{split}
             \ce{C_aH_bN_cO_dP_eS_{f}^{Z} + (3a +4e -d) H2O ->} \\
             \ce{aHCO3- + cNH4+ + eHPO4^{2-} + fHS- + (5a + b - 4c - 2d + 7e - f)H+ + } \\
              \ce{(-Z + 4a + b - 3c - 2d + 5e - 2f)e^-}
         \end{split}
         \end{equation}

         where $Z$ corresponds to the net charge of the organic compound, and the subscripts $ a, b, c, d, e$ and $f$ refer to the stoichiometric numbers of the elements C, H, N, O, P and S.

        Values of Gibbs free energy of formation and Gibbs energies of oxidation half reactions are tabulated in many texts \citep{Stumm1995,Rittmann2001,Lide2010}.

        \emph{The average nominal oxidation state of the carbon (NOSC) represents a simple proxy that can be used to scale the bonding in organic compounds to their energetic content. The NOSC has the advantage that it does not require structural information in order to estimate the energetic potential of complex, natural organic matter.} \cite{Arndt2013}

        \emph{If the half reaction value is not tab- ulated, the calculation is straightforward using values for the standard free energy of formation for all the species in the half reaction.} \cite{VanBriesen2002}


        % subsection gibbs_energy (end)

        \subsection{Chemical equilibrium} % (fold)
        \label{sub:chemical_equilibrium}
        Chemical potential of any specie is:

        \begin{equation}
            \mu_i = \mu_i^0 + R T \ln{a_i}
        \end{equation}

        where $a_i$ - is the chemical potential in terms of activity. Substitute it into equation based on mass action law:

        \begin{equation}
            \Delta G = \sum_i \nu_i \mu_i
        \end{equation}

        Then Gibbs free energy of a reaction:
        \begin{equation}
            \Delta G = \sum_i \nu_i \mu_i^0 + R T \sum_i \nu_i \ln{a_i}
        \end{equation}

        or:

        \begin{equation}
            \Delta G = \Delta G^0 + R T \ln{\prod_i{a_i^{\nu_i}}}
        \end{equation}

        where:

        \begin{equation}
            Q = \prod_i{a_i^{\nu_i}} = \prod_i{(\gamma_i C_i)^{\nu_i}}
        \end{equation}

        $Q$ - is reaction quotient, $\gamma_i$ - activity coefficient.

        At chemical equilibrium $\Delta G = 0$, then:

        \begin{equation}
            \Delta G^0 = - R T \ln{K_{eq}}
        \end{equation}

        In this case quotient $Q$ became $K_{eq}$. We should remember quotient is represented in activities not in concentrations.


        % subsubsection chemical_equilibrium (end)

        \subsection{Transferring of 1 $\bar{e}$} % (fold)
        \label{sub:rate_of_transfering_of_1_e}

        \emph{Comparing the energetics of these two processes on a per mole basis would lead to the erroneous conclusion that iron reduction would never occur until all sulfate is ex- hausted (see below). This normalization is also useful because, as will be discussed later, the impact of reaction energetics on reaction rates is related to the energy yield per electron catalyzed. } \cite{LaRowe2011}

        {\color{red}\textbf{TODO here:}
        \begin{itemize}
            \item Table of reaction rates for different TEA and TED to compare Gibbs energetics with rate. Maybe they related somehow? Do we have these values?
        \end{itemize}}

        % subsection rate_of_transfering_of_1_bar (end)

\section{Enzymes and Metabolism} % (fold)
    \label{sec:enzymes_and_metabolism}
    \subsection{Enzymes} % (fold)
    \label{sub:enzymes}

    % subsection enzymes (end)

    \subsubsection{Enzymes as catalysts} % (fold)
    \label{sub:enzymes_as_catalysts}

    Copy-paste from \cite{Karp2008}.

    Enzymes are the mediators of metabolism, responsible for virtually every reaction that occurs in a cell.

    Even though enzymes are proteins, many of them are conjugated proteins; that is, they contain nonprotein components, called cofactors, which may be inorganic (metals) or organic (coenzymes). Vitamins and their derivatives often function as coenzymes. When present, cofactors are important participants in the functioning of the enzyme, often carrying out activities for which amino acids are not suited.

    As is true of all catalysts, enzymes exhibit the following properties:

    \begin{itemize}
        \item they are required only in small amounts;
        \item they are not altered irreversibly during the course of the reaction, and therefore each enzyme molecule can participate repeatedly in individual reactions;
        \item they have no effect on the thermodynamics of the reaction.
    \end{itemize}

    Catalysts employed by organic chemists in the lab, such as acid, metallic platinum, and magnesium, generally accelerate reactions a hundred to a thousand times over the noncatalyzed rate. In contrast, enzymes typically increase the velocity of a reaction $10^8$ to $10^{13}$ fold or greater.

    The reactants bound by an enzyme are called substrates.

    In addition to their high level of activity and specificity, enzymes act as metabolic traffic directors in the sense that enzyme catalyzed reactions are very orderly the only products formed are the appropriate ones. This is very important be- cause the formation of chemical by-products would rapidly take its toll on the life of a fragile cell.

    Finally, unlike other catalysts, the activity of enzymes can be regulated to meet the particular needs of the cell at a particular time. As will be evident in this chapter and the remainder of the text, the enzymes of a cell are truly a wondrous collection of miniature molecular machines.

    Enzyme task - to overcome the activation energy barrier. Even ATP, whose hydrolysis is so favorable, is stable in a cell until broken down in a controlled enzymatic reaction. If this weren’t the case, ATP would have little biological use.

    Chemical transformations require that certain covalent bonds be broken within the reactants. For this to occur, the reactants must contain sufficient kinetic energy (energy of motion) that they overcome a barrier, called the activation energy ($E_A$) shown on the fig. \ref{fig:Ea}.

    \begin{figure}[tb]
        \begin{center}
            \includegraphics[width=\textwidth]{activation_energy}
        \end{center}
        \caption{Activation energy}
        \label{fig:Ea}
    \end{figure}

    That part of the enzyme molecule that is directly involved in binding the substrate is termed the \textbf{active site}. The active site and the substrate(s) have complementary shapes, enabling them to bind together with a high degree of precision. The binding of substrate to enzyme is accomplished by the same types of noncovalent interactions (ionic bonds, hydrogen bonds, hydrophobic interactions) that determine the structure of the protein itself.

    \begin{wrapfigure}{r}{0.5\textwidth}
        \begin{center}
            \includegraphics[width=0.48\textwidth]{enzyme_mechanism}
        \end{center}
        \vspace{-20pt}
        \caption{Three mechanisms by which enzymes accelerate reactions.}
        \label{fig:enz_mechanism}
    \end{wrapfigure}

    Three mechanisms by which enzymes accelerate reactions (fig. \ref{fig:enz_mechanism}):
    \begin{itemize}
        \item maintaining precise substrate orientation;
        \item changing substrate reactivity by altering its electrostatic configuration;
        \item exerting physical stress on bonds in the substrate to be broken.
    \end{itemize}

    In 1913, Leonor Michaelis and Maud Menten reported on the mathematical relationship between substrate concentration and the velocity of enzyme reactions as measured by the amount of product formed (or substrate consumed) in a given amount of time:

    \begin{equation}
        V = V_{max} \frac{[S]}{[S]+K_M}
    \end{equation}


    \begin{figure}[tb]
        \begin{center}
            \includegraphics[width=0.98\textwidth]{enz_velocity}
        \end{center}
        \vspace{-20pt}
        \caption{The relationship between the rate(velocity)of an enzyme-catalyzed reaction and the substrate concentration.}
        \label{fig:enz_velocity}
    \end{figure}

    To generate a hyperbola curve (fig. \ref{fig:enz_velocity}), the initial velocity is determined for a series of incubation mixtures that contain the same amount of enzyme but an increasing concentration of substrate. It is evident from this curve that the initial reaction velocity varies markedly with substrate concentration. At low substrate concentrations, enzyme molecules are subjected to relatively few collisions with substrate in a given amount of time. Consequently, the enzyme has “idle time”; that is, substrate molecules are rate limiting. At high substrate concentrations, enzymes are colliding with substrate molecules more rapidly than they can be converted to product. Thus, at high substrate concentrations, individual enzyme molecules are working at their maximal capacity; that is, enzyme molecules are rate limiting. Thus, as a greater and greater concentration of substrate is present in the reaction mixture, the enzyme approaches a state of saturation. The initial velocity at this theoretical saturation point is termed the maximal velocity (Vmax).

    To generate a hyperbolic curve such as that in fig. \ref{fig:enz_velocity} and make an accurate determination of the values for Vmax and KM, a considerable number of points must be plotted. An eas- ier and more accurate description is gained by plotting the reciprocals of the velocity and substrate concentration against one another, as formulated by Hans Lineweaver and Dean Burk. When this is done, the hyperbola becomes a straight line (Figure \ref{fig:Lineweaver-Burk}) whose x intercept is equal to $􏰂1/K_M$, y intercept is equal to $1/V_{max}$, and slope is equal to $K_M/V_{max}$. The values of $K_M$ and $V_{max}$ are, therefore, readily determined by extrapolation of the line drawn from a relatively few points.

    \begin{figure}[tb]
        \vspace{-20pt}
        \begin{center}
            \includegraphics[width=0.98\textwidth]{Lineweaver-Burk}
        \end{center}
        \vspace{-20pt}
        \caption{A Lineweaver-Burk plot of the reciprocals of velocity and substrate concentration.}
        \label{fig:Lineweaver-Burk}
    \end{figure}

    The $K_M$ for most enzymes ranges between $10􏰂^-{1}$ and $10^{-􏰂7}$, with a typical value about $10^{-􏰂4}$ M.

    Other factors that strongly influence enzyme kinetics are the pH and temperature of the incubation medium. Each enzyme has an optimal pH and temperature at which it operates at maximal activity.

    \textbf{Enzyme inhibitors} are molecules that are able to bind to an enzyme and decrease its activity. The cell depends on inhibitors to regulate the activity of many of its enzymes. Enzyme inhibitors can be divided into two types: reversible or irreversible. \textbf{Irreversible inhibitors} are those that bind very tightly to an enzyme, often by forming a covalent bond to one of its amino acid residues. \textbf{Reversible inhibitors}, on the other hand, bind only loosely to an enzyme, and thus are readily displaced. \textbf{Competitive inhibitors} are reversible inhibitors that compete with a substrate for access to the active site of an enzyme. In \textbf{noncompetitive inhibition}, the substrate and in- hibitor do not compete for the same binding site; generally, the inhibitor acts at a site other than the enzyme’s active site.


    % subsection enzymes_as_catalysts (end)

    \subsubsection{Enzyme production - the limiting factor} % (fold)
    \label{sec:enzyme_production_the_limiting_factor}

    {\color{red}{Maybe. But probably the bacteria already have everything it needs. Maybe consider just shift from 1 enzyme to 1 another which takes time.}}

    Enzymes remains active even 72h after the death of the bacteria that produced them \citep{Kiersztyn2012}.

    \begin{itemize}
        \item fixing carbon from co2?
        \item limited production due to limited carbon
    \end{itemize}

    \subsection{Metabolism} % (fold)
    \label{sub:metabolism}

    Increasing evidence suggests that the enzymes of a metabolic pathway are often physically linked to one another, a feature that allows the product of one enzyme to be delivered directly as a substrate to the active site of the next enzyme in the reaction sequence.

    \textbf{Catabolic pathways} lead to the disassembly of complex molecules to form simpler products. Catabolic pathways serve two functions: they make available the raw materials from which other molecules can be synthesized, and they provide chemical energy required for the many activities of a cell. As will be discussed at length, energy released by catabolic pathways is stored temporarily in two forms: as high-energy phosphates (primarily ATP) and as high-energy electrons (primarily in NADPH).

    \textbf{Anabolic pathways} lead to the synthesis of more complex compounds from simpler starting materials. Anabolic pathways are energy-requiring and utilize chemical energy released by the exergonic catabolic pathways.

    Once macromolecules have been hydrolyzed into their components - amino acids, sugars, and fatty acids - the cell can reutilize the building blocks directly to (1) form other macromolecules of the same class (stage I), (2) convert them into different compounds to make other products, or (3) degrade them further (stages II and III, fig. \ref{fig:metabol_stages}) and extract a measure of their free-energy content.

    \begin{figure}[tb]
        \begin{center}
            \includegraphics[width=0.98\textwidth]{metabol_stages}
        \end{center}
        \caption{Three stages of metabolism}
        \label{fig:metabol_stages}
    \end{figure}

    Even though substances begin as macromolecules having a very different structure, they are converted by catabolic pathways to the same low-molecular-weight metabolites. For this reason, catabolic pathways are said to be convergent.

    Remarkably, the chemical reactions and metabolic pathways described in this chapter are found in virtually every living cell, from the simplest bacterium to the most complex plant or animal. It is evident that these pathways appeared very early in, and have been retained throughout, the course of biological evolution.

    \subsubsection{Utilization of energy} % (fold)
    \label{ssub:utilization_of_energy}

    when a pair of electrons is shared by two different atoms, the electrons are attracted more strongly to one of the two atoms of the polarized bond. In a C—H bond, the carbon atom has the strongest pull on the electrons; thus it can be said that the carbon atom is in a reduced state. In contrast, if a carbon atom is bonded to a more electronegative atom, as in a C—O or C—N bond, the electrons are pulled away from the carbon atom, which is thus in an oxidized state. Since carbon has four outer-shell electrons it can share with other atoms, it can exist in a variety of oxidation states. This is illustrated by the carbon atom in a series of one-carbon molecules (Figure 3.23) rang- ing from the fully reduced state in methane (CH4) to the fully oxidized state in carbon dioxide (CO2).

    The oxidation state of the carbon atoms in an organic molecule provides a measure of the molecule’s free-energy content.

    The compounds that we use as chemical fuels to run our furnaces and automobiles are highly reduced organic compounds, such as natural gas (CH ) and petroleum derivatives. Energy is released when these molecules are burned in the presence of oxygen, converting the carbons to more oxidized states, as in carbon dioxide and carbon monoxide gases. \textbf{The degree of reduction of a compound is also a measure of its ability to perform chemical work within the cell. }

    The free energy released by the complete oxidation of glucose is very large 686 kcal/mol, and up to about 36 molecules of ATP are formed per molecule of glucose oxidized under conditions that exist within most cells (free energy required to convert ADP to ATP is 7.3 kcal/mol).

    There are basically two stages in the catabolism of glucose, and they are virtually identical in all aerobic organisms. The first stage, \textbf{glycolysis}, occurs in the soluble phase of the cytoplasm (the cytosol) and leads to the formation of pyruvate. The second stage is the \textbf{tricarboxylic acid (or TCA) cycle}, which occurs within the mitochondria of eukaryotic cells and the cytosol of prokaryotes and leads to the final oxidation of the carbon atoms to carbon dioxide. Most of the chemical energy of glucose is stored in the form of high-energy electrons, which are removed as substrate molecules are oxidized during both glycolysis and the TCA cycle.

    An important feature of glycolysis is that it can generate a limited number of ATP molecules in the absence of oxygen. Thus, gly- colysis can be considered an anaerobic pathway to ATP pro- duction, indicating that it can proceed in the absence of molecular oxygen to continue to provide ATP. Two molecules of ATP are produced by substrate-level phosphorylation.

    Pyruvate, the end product of glycolysis, is a key compound because it stands at the junction between anaerobic (oxygen-independent) and aerobic (oxygen-dependent) path- ways. In the absence of molecular oxygen, pyruvate is sub- jected to fermentation. When oxygen is available, pyruvate is further catabolized by aerobic respiration.

    \subsubsection{Fermentation} % (fold)
    \label{ssub:fermentation}

    Glycolytic reactions occur at rapid rates so that a cell is able to produce a signifi- cant amount of ATP using this pathway. In fact, a number of cells, including yeast cells, tumor cells, and muscle cells, rely heavily on glycolysis as a means of ATP formation. There is, however, a problem that must be confronted by such cells. One of the products of the oxidation of glyceraldehyde 3-phosphate is NADH. The formation of NADH occurs at the expense of one of the reactants, NAD􏰄, which is in short sup- ply in cells. Since NAD􏰄 is a required reactant in this important step of glycolysis, it must be regenerated from NADH. However, without oxygen, NADH cannot be oxi- dized to NAD􏰄 by means of the electron-transport chain be- cause oxygen is the final electron acceptor of the chain. Cells are able, however, to regenerate NAD􏰄+ by \textbf{fermentation}.  For most organisms, which depend on O2, fermentation is a stopgap measure to regenerate NAD􏰄 when O2 levels are low, so that glycolysis can continue and ATP production can be maintained.

    \begin{figure}[!htb]
        \begin{center}
            \includegraphics[width=0.7\textwidth]{fermentation_karp}
        \end{center}
        \caption{Fermentation. Different pathways depending on the oxygen concentration. The main thing that cells need to regenerate the NAD+ to continue glycolysis.}
        \label{fig:fermentation_karp}
    \end{figure}

    Under these conditions, skeletal muscle cells regenerate NAD􏰄 by converting pyruvate to lactate. When oxygen once again becomes available in suffi- cient amounts, the lactate can be converted back to pyruvate for continued oxidation. Yeast cells have faced the challenges of anaerobic life with a different metabolic solution: they convert pyruvate to ethanol, as illustrated in Figure \ref{fig:fermentation_karp}.

    When a mole of glucose is completely oxidized, 686 kcal are released. In contrast, only 57 kcal are released when this same amount of glucose is converted to ethanol under standard conditions, and only 47 kcal are released when it is converted to lactate. In any case, only two molecules of ATP are formed per glucose oxidized by glycolysis and fermentation; over 90 percent of the energy is simply discarded in the fermentation product (as evidenced by the flammability of ethyl alcohol)

    During the early stages of life on Earth, when oxygen had not yet appeared, glycolysis and fermentation were probably the primary metabolic pathways by which energy was ex- tracted from sugars by primitive prokaryotic cells. Following the appearance of cyanobacteria, the oxygen of the atmosphere rose dramatically, and the evolution of an aerobic metabolic strategy became possible.

    \subsubsection{Minimal biosynthetic requirements} % (fold)
    \label{ssub:low_energy_environment}

    \paragraph{Reducing power.}

    Synthesis of fats requires the reduction of metabolites, which is accomplished by the transfer of high-energy electrons from NADPH, a compound similar in structure to NADH, but containing an additional phosphate group. A cell’s reservoir of NADPH represents its \textbf{reducing power}, which is an important measure of the cell’s usable energy con- tent. The oxidized form of NADPH, NADP􏰄, is formed from NAD􏰄+. The transfer of free energy in the form of these electrons (from NADPH) raises the acceptor to a more reduced, more energetic state.

    {\color{red}When energy is abundant, the production of NADPH is favored, providing a supply of electrons needed for biosynthesis of new macromolecules that are essential for growth. When energy resources are scarce, however, most of the high-energy electrons of NADH are “cashed in” for ATP, and only enough NADPH is generated to meet the cell’s minimal biosynthetic requirements.}

    \subsubsection{Metabolic regulation} % (fold)
    \label{ssub:metabolic_regulation}

    The amount of ATP present in a cell at a given moment is sur- prisingly small. A bacterial cell, for example, contains approxi- mately one million molecules of ATP, whose half-life is very brief (on the order of a second or two). With so limited a sup- ply, it is evident that ATP is not a molecule in which a large to- tal amount of free energy is stored. The energy reserves of a cell are stored instead as polysaccharides and fats. When the levels of ATP start to fall, reactions are set in motion to increase ATP formation at the expense of the energy-rich storage forms. Similarly, when ATP levels are high, reactions that would nor- mally lead to ATP production are inhibited. Cells regulate these important energy-releasing reactions by controlling certain key enzymes in a number of metabolic pathways.

    In the case of enzymes, regulation of cat- alytic activity is centered on the modification of the structure of the active site. Two common mechanisms for altering the shape of an enzyme’s active site are covalent modification and allosteric modulation, both of which play key roles in regulat- ing glucose oxidation. Metabolism is also regulated by controlling the concentrations of enzymes.

    Altering enzyme activity:

    \begin{itemize}
        \item \textbf{By covalent modification.} Fischer and Krebs prepared a crude extract of muscle cells and found that inactive enzyme molecules in the extract could be converted to active ones simply by adding ATP to the test tube. Subsequent research has shown that covalent modifica- tion of enzymes, as illustrated by the addition (or removal) of phosphates, is a general mechanism for changing the activity of enzymes. Enzymes that transfer phosphate groups to other proteins are called protein kinases and regulate such diverse activities as hormone action, cell division, and gene expression.

        \item \textbf{By Allosteric modulation.} Allosteric modulation is a mechanism whereby the activity of an enzyme is either inhibited or stimulated by a compound that binds to a site, called the \textbf{allosteric site}, that is spatially distinct from the enzyme’s active site.
    \end{itemize}

    {\color{red} One of the primary mechanisms cells use to shut down anabolic assembly lines is a type of allosteric modulation called \textbf{feedback inhibition}, in which the enzyme catalyzing the first committed step in a metabolic pathway is temporar- ily inactivated when the concentration of the end product of that pathway—an amino acid, for example—reaches a certain level.} Two substrates, A and B, are converted to the end product E. As the concentration of the product E rises, it binds to the allosteric site of enzyme BC, causing a conformational change in the active site that decreases the en- zyme’s activity.

    \paragraph{Separating Catabolic and Anabolic Pathways.}

    Most cells are able to synthe- size glucose from pyruvate at the same time they are oxidizing glucose as their major source of chemical energy. Thus, a cell could not shut down glucose synthesis and crank up glucose breakdown because the same enzymes would be active in both directions. The particular enzymes of glycolysis and gluconeogenesis described above are key enzymes in their respective pathways. These enzymes are regulated, in part, by AMP and ATP. Keep in mind that the concentration of AMP within cells is in- versely related to the concentration of ATP; when ATP levels are low, AMP levels are high, and vice versa. Consequently, el- evated AMP concentrations signal to a cell that its ATP fuel reserves are becoming depleted.

    \textbf{Synopsis}:

    \begin{itemize}
        \item All spontaneous (exergonic) energy transformations proceed from a state of higher free energy to a state of lower free energy; the 􏰀G must be negative. Reactions having equilibrium constants greater than one have negative 􏰀G􏰃􏰆 values.

        \item The hydrolysis of ATP is a highly favorable reaction (􏰀G􏰃􏰆 􏰁 􏰂7.3 kcal/mol) and can be used to drive reactions that would otherwise be unfavorable. The concentrations of reactants and products can be main- tained at relatively constant nonequilibrium (steady state) values within a cell because materials are continually flowing into the cell from the external medium and waste products are continually being removed.

        \item Enzymes act by lowering the activation energy (EA)—the kinetic energy required by reactants to undergo reaction.

        \item Metabolism is the collection of biochemical reactions that occur within the cell. Metabolic pathways are divided into two broad types: catabolic pathways, in which com- pounds are disassembled and energy is released, and anabolic path- ways that lead to the synthesis of more complex compounds using energy stored in the cell.

        \item Enzymes are proteins that vastly accelerate the rate of specific chemical reactions by binding to the reactant(s) and increasing the likelihood that they will be converted to products.

        \item The state of reduction of an organic molecule, as measured by the number of hydrogens per carbon atom, provides a rough measure of the molecule’s energy content.

        \item The NADHs are produced by oxidation of an aldehyde to a carboxylic acid with the accompanying transfer of a hy- dride ion (a proton and two electrons) from the substrate to NAD􏰄. In the presence of O2, most cells oxidize the NADH by means of an electron-transport chain, forming ATP by aerobic respiration. In the absence of O2, NAD􏰄 is regenerated by fermentation in which the high-energy electrons of NADH are used to reduce pyruvate. NAD􏰄 regeneration is required for glycolysis to continue.

        \item An enzyme’s activity is commonly regulated by two mechanisms: covalent modification and allosteric modulation.

        \item Glucose is degraded by glycolysis and synthesized by gluconeogenesis. While most of the enzymes are shared by the two pathways, three key en- zymes are unique to each pathway, which allows the cell to regulate the two pathways independently and to overcome what would oth- erwise be irreversible reactions.

    \end{itemize}

    \subsection{Cell maintenance} % (fold)
    \label{sub:cell_maintenance}

    \emph{The above equations have also been modified to include the effects of cell maintenance and cell death.} \citep{Blanch1981}


        \subsection{Derivation of the non-competitive inhibition term.} % (fold)
        \label{ssub:derivation_of_the_inhibition_term_}

        \begin{figure}[htbp]
            \centering
            \includegraphics[width=0.65\textwidth]{enz_noncomp_scheme}
            \caption{Schematic representation of reaction mechanism of non-competitive inhibition.}
            \label{fig:enz_noncomp_scheme}
        \end{figure}

        During non-competitive inhibition the inhibitor binds to the allosteric site of the enzyme and distort the active site:

        \begin{equation}
            K_I = \frac{[E][I]}{[EI]}
        \end{equation}

        The only difference with competitive type inhibition that the inhibitor can bind not only to the enzyme but also to the enzyme-substrate complex due to it has the separate allosteric site. In general case the substrate binded to the active site can affect the the dissociation constant:

        \begin{equation}
        \label{eq:diss_const}
            K^{\prime}_I = \frac{[ES][I]}{[ESI]}
        \end{equation}

        The rate of the formation of product:

        \begin{equation}
            rate = k_b [ES]
        \end{equation}

        Mass balance of the enzyme in the system:

        \begin{equation}
        \label{eq:mass_bal}
            [E]_0 = [E] + [ES] + [EI] + [ESI]
        \end{equation}

        Where the concentration of the enzyme $[E]$ in steady-state:

        \begin{equation}
            [E] = \frac{K_m [ES]}{[S]}
        \end{equation}

        \begin{equation}
            [EI] = \frac{K_m[ES][I]}{[S]K_I}
        \end{equation}

        From the Eq. (\ref{eq:diss_const}):

        \begin{equation}
            [ESI] = \frac{[ES][I]}{K^{\prime}_I }
        \end{equation}

        Substitute into Eq. (\ref{eq:mass_bal}):

        \begin{equation}
            [E]_0 = \frac{K_m [ES]}{[S]}+ [ES] + \frac{K_m[ES][I]}{[S]K_I} + \frac{[ES][I]}{K^{\prime}_I }
        \end{equation}

        Solve it for $[ES]$:

        \begin{equation}
            [E]_0 = [ES] \frac{K_m}{[S]} \left( 1 + \frac{[I]}{K_I} \right) + [ES] \left( 1 + \frac{[I]}{K^{\prime}_I} \right)
        \end{equation}

        Let:

        \begin{equation}
            \alpha = \left( 1 + \frac{[I]}{K_I} \right)
        \end{equation}

        And:

        \begin{equation}
            \alpha^{\prime} = \left( 1 + \frac{[I]}{K^{\prime}_I} \right)
        \end{equation}

        Then:
        \begin{equation}
            [E]_0 = [ES] \frac{K_m}{[S]} \alpha + [ES] \alpha^{\prime} = [ES] \left( \frac{K_m}{[S]} \alpha + \alpha^{\prime} \right)
        \end{equation}

        And:

        \begin{equation}
            [ES] = \frac{[E]_0}{ \frac{K_m}{[S]} \alpha + \alpha^{\prime}} = \frac{[E]_0 [S]}{K_m \alpha + [S] \alpha^{\prime}}
        \end{equation}

        Plug into the reaction rate equation:

        \begin{equation}
            rate =  \frac{k_b [E]_0 [S]}{K_m \alpha + [S] \alpha^{\prime}} = \frac{V_{max} [S]}{K_m \alpha + [S] \alpha^{\prime}}
        \end{equation}

        if $ K^{\prime}_I = K_I$:

        \begin{equation}
            rate = \frac{V_{max} [S]}{K_m \alpha + [S] \alpha} = \frac{V_{max} [S]}{K_m + [S]} \cdot \frac{1}{\alpha } = \frac{V_{max} [S]}{K_m + [S]} \cdot \frac{K_I}{K_I+[I]}
        \end{equation}



    \section{Reaction paths} % (fold)
    \label{sec:bioenergetic_model_of_low_energy_environment}

    \emph{More importantly,acetate is a central intermediate in the metabolism for organic matter in  many sedimentary environments.} \citep{Lovley1995}

    \emph{the types of organic compounds that can be oxidized to carbon dioxide are different for different terminal electron acceptors.} \citep{Lovley1995}

    \emph{Some aerobic microorganisms can also catalyze the oxidation of inorganic compound such as ammonium, Fe(II), Mn(II), and S0[Ehrlich,1990]}. \citep{Lovley1995}

     \emph{However, some important reactions that clearly take place with O2 as TEA have not been documented under denitrifying conditions. For example, the current literature indicates that benzene,a prevalent groundwater contaminant, can not be oxidized with nitrate as the electron acceptor, whereas microorganisms readily oxidize it to carbon dioxide when 02 is available[Anidetal., 1993;Barbaroetal., 1992;Flyvbjerg et al., 1993; Hutchins et al. 1991].} \cite{Lovley1995}

     \begin{figure}[htbp]
         \centering
         \includegraphics[width=1\textwidth]{Lovley_original}
         \caption{Original pic from \cite{Lovley1995}}
     \end{figure}


    \begin{figure}[tb]
        \begin{center}
            \includegraphics[width=12.5cm]{Ferm001}
        \end{center}
        \caption{Simplified reactions paths}
        \label{fig:fermentation_concept_slide}
    \end{figure}

    % section bioenergetic_model_of_low_energy_environment (end)

    \subsection{Competing environment} % (fold)
    \label{sub:competing_environment}

        Suppose we have environment with glucose and different TEA. Then, corresponding Gibbs Energies of reactions are represented in Table \ref{tab:redox_react}:

        \begin{equation}
            \Delta G_i = \Delta G_i^0 + R T \ln{Q_i}
        \end{equation}

        {\color{red}\textbf{Important ideas here:}
        \begin{itemize}
            \item Maybe they don't compete. Maybe they try to live as beneficial as possible. Maybe microorganisms try to maximize the usable energy of all reactions??? In this case it is possible to model it as optimization problem and find the best possible outcome for microcosm.

            This thoughts were delivered by \cite{Bethke2011}.

            \item Another explanation of the same rates in the experiment of \cite{Bethke2011}. They concluded that: \emph{Iron reducers and sulfate reducers, as shown in table 3, conserve in their ATP pools four or five times more energy than methanogens, per mole of electron donor consumed. Acidophilic iron reducers, as noted above, may capture energy even more effectively. The more energy a microbe traps, the larger a growth yield it can achieve.}


             Therefore, it could be that it is more difficult for bacteria to produces more and more ATP if it has already a lot??

        \end{itemize}}

    % subsection competing_environment (end)

    \subsection{Choosing the reaction path} % (fold)
    \label{sub:reaction_path}

        Choose the reaction path in fig. \ref{fig:fermentation_concept_slide} based on what?:
        {\color{red}
        \begin{equation}
            \Delta G \textbf{? or  } \Delta G^0 \cdot R \cdot F_{in} \textbf{? or } F_{T} \textbf{? or } [B]^{anox}
        \end{equation}
        }
    % section choosing_reaction_path_model (end)


    \subsection{Why it works} % (fold)
    \label{ssub:why_it_works}

        \emph{Geochemical analyses of subsurface environments typically treat microorganisms merely as black boxes that may help facilitate certain reactions that are thermodynamically favorable.} from \cite{Lovley1995}

        \emph{From this perspective it is unnecessary to have a detailed understanding of the functioning of the microorganisms because the reactions that the microorganisms catalyze can be predicted on the basis of purely non-biological models such as equilibrium thermodynamics.} from \cite{Lovley1995}

    % subsection reaction_path (end)

    \section{Reaction rates of bacterial activity} % (fold)
    \label{sec:reaction_rates}

    \emph{Organic matter is composed of a wide variety of organic substances, in-cluding protein, lignin, cellulose, chitin, plus all of the other products of biosynthesis and the “geopolymerization” process. It would be truly astounding if the decay of all these compounds were to fall neatly into a small finite number of reactive types. More likely, the sensitivity of the measurement methods forcibly divides the true distribution of organic matter types into the distinct categories (Middelburg, 1989).} \citep{Boudreau1997Diagenetic}.

    \emph{Aris (1968, 1989) and Ho and Aris (1987) have advanced the use of the Gamma distribution as a good all-purpose distribution for reactive continua}. \citep{Boudreau1997Diagenetic}




    \subsection{Hydrolysis} % (fold)
    \label{sec:hydrolysis}

    \emph{Magnitude of KM for Enzymes. In general KM for the respiratory enzymes, those associate with sugar metabolism, is lower than KM for the hydrolytic enzymes, those associate with primary substrate attack}.\citep{Humphrey1972}

    \emph{p.18: Enzymatic Action on Large Polymeric Surfaces. An interesting category of enzyme reactions is that represented by attack of enzyme on large polymeric surfaces.} \citep{Humphrey1972}:

    \begin{equation}
        v = \frac{v_{max} [S]}{K_M + [S]} \prod_i \frac{K_I}{K_I+I_i}
    \end{equation}

    Hydrolysis is represented by Michaelis - Menten kinetics \citep{Michaelis1913,Dale2006,Thullner2007}:

    \begin{equation}
        R_{hydr} = k_{cat} \cdot [E] \cdot \frac{[POM]}{[POM]+k_m^{POM}}
    \end{equation}

    where $k_{cat}$ - the turnover number, is the maximum number of substrate molecules converted to product per enzyme molecule per second; $[E]$ - enzyme concentration; $[POM]$ - particular organic matter (macromolecule) concentration; $k_m^{POM}$ - half-saturation constant.

    {\color{red}Todo:

    \begin{itemize}
        \item Hydrolysis in case of solid POM.
    \end{itemize}}

    % subsection hydrolysis (end)




    \subsection{Respiration} % (fold)
    \label{sec:respiration}

    To describe the respiration we use the dual-Monod equation which describes the possibility of reaction being limited by both: electron donor or acceptor \citep{Jin2005,Thullner2007}:

    \begin{equation}
    \label{respiration_eq}
        R_{resp} = k_{resp} \cdot [B] \cdot \frac{[TED]}{[TED]+k_m^{TED}} \cdot \frac{[TEA]}{[TEA]+k_m^{TEA}}
    \end{equation}

    where $k_{resp}$ - is the rate constant; $[B]$ - biomass, $[TED]$ - concentration of terminal electron donor(TED); $k_m^{TED}$ - half-saturation constant of TED; $[TEA]$ - concentration of terminal electron acceptor(TEA); $k_m^{TEA}$ - half-saturation constant of TEA.

    \subsubsection{Aerobic respiration} % (fold)
    \label{ssub:aerobic_respiration}

        \emph{The pathways for organic matter oxidation by most anaerobic microbial processes are much different than those for aerobic respiration and denitrification. Whereas single organisms can, by themselves, completely oxidize a wide variety of organic compounds to carbon dioxide during aerobic respiration and denitrification, in most types of anaerobic respiration the microorganisms are very limited in the types of organic compounds they can completely oxidize} \cite{Lovley1995}

        \emph{It has been shownby Bender and Heggie [1984]that 90-95\% of the degraded organic carbon is consumed by bacteria using oxygenwhile the other dissolved or solid electron acceptors (NO3-, MnO2, Fe203, SO4) serve to oxidize less than 5\% of the metabolized organic carbon fraction.} \cite{Rabouille1991}

        \emph{Recently, Jahnke and Jackson [1988] proposed that the sedimentcould control the oxygen concentrationof deep waters (below 3000 m) because bacterial degradation of organic matter by oxygen is much more e•cient in the sedimentsthan in the free water at these depths} \cite{Rabouille1991}

    \subsubsection{Denitrification} % (fold)
        \label{ssub:denitrification}


        \emph{For review of denitrification metabolism see Tiedje[1988]... In some instances, O2 and nitrate may be reduced simultaneously Tiedje[1988]} \cite{Lovley1995}

        \emph{ Like aerobic respiration, nitrate reduc tion is expected to be a minor process for organic matter oxidation in most deep subsurface environments because nitrate inputs from groundwater are low and there are no significant mineral sourcesof nitrate.} \cite{Lovley1995}


        % subsubsection denitrification (end)

    \subsubsection{Iron respiration} % (fold)
    \label{ssub:iron_respiration}

        \emph{For example, microorganisms that use Fe(III) as an electron acceptor can only oxidize simple organic acids, long chain fatty acids, and monoaromatic compounds to carbondioxide [Lovley, 1991, 1994]} \cite{Lovley1995}

        \emph{Thus in nitrate-depleted, anoxic environments a significant amount of the organic matter released from hydrolysis of complex organic matter is first metabolized by fermentative microorganisms(Figure 3). Fermentative microorganisms convert large organic compounds to smaller organics} \cite{Lovley1995}

        \emph{Fe(III)-reducing microorganisms can also directly oxidize monoaromatic compounds and long-chainfatty acids that are released from complex organic matter (Figure 3).} \cite{Lovley1995}
        % subsubsection iron_respiration (end)

    \subsubsection{Sulfate reduction} % (fold)
    \label{ssub:sulfate_reduction}
        \emph{Sulfate may also be used as an electron acceptor for the oxidation of organic compounds in sedimentary environments (see Widdel [1988] for a detailed review of this metabolism)} \cite{Lovley1995}

        \emph{The reactions carried out by sufate reducers(Figures2 and 3) are similar to those for Fe(III) reduction,with the exception that sulfate reducers substitute sulfate for Fe(III). The sulfate is reduced to sulfide.} \cite{Lovley1995}
    % subsubsection sulfate_reduction (end)

    \subsubsection{Reduction of other metals} % (fold)
    \label{ssub:reduction_of_other_metals}
        \emph{The reduction of metals other than Fe(III) and Mn(IV) is generally not significant in terms of organic matter oxidation because of the relatively low concentrations of these electron acceptors.However, the reduction of these metals has the potential to greatly influence metal geochemistry. Metal and metalloid reductions carried out by microorganismsincludeU(VI) to U(IV); Se(VI) to Seø, 1972;Reeburgh,1983],and a similar success ion of Cr(VI) to Cr(III), Mo(VI) to Mo(V), and Au(III) to Au(O).}
    % subsubsection reduction_of_other_metals (end)

    \subsection{Methane} % (fold)
    \label{sub:methane}

    % subsection methane (end)

    \subsubsection{Methanogenesis} % (fold)
    \label{sub:methanogenesis}


        \emph{Microorganisms may also convert organic matter to a mixture of carbon dioxide and methane. However, the methane-producing microorganisms are strictly limited in the types of compounds that they can metabolize[Oremland,1988]} \cite{Lovley1995}

        \emph{The principal types of methane production are the dismuataion of acetate to methane and carbon dioxide and the reduction of carbon dioxide to methane with \ce{H2} serving as the electron donor.} \cite{Lovley1995}

        \emph{Methane producing microorganisms cannot use fatty acids with more than 2 carbons and cannot methabolize aromatic compounds. Therefore, under methanogenic conditions, another group of organisms, the proton reducing acetogenic bacteria, are also required} \cite{Lovley1995}

    \subsubsection{Oxidation of Methane} % (fold)
    \label{ssub:oxidation_of_methane}

    \cite{Haroon2013,Raghoebarsing2006,Ettwig2010,Boetius2000,Beal2009}



    % subsection respiration (end)

    \subsection{Fermentation} % (fold)
    \label{sec:fermentation}

        \begin{figure}[tb]
            \begin{center}
                \includegraphics[width=12.5cm]{fermentation}
            \end{center}
            \caption{Picture of oxidative phosphorylation and fermentation from wiki}
            \label{fig:fermentation_wiki}
        \end{figure}


        {\color{red}Fermentation occurs for two reasons according to \cite{Konhauser2007}:
        \begin{itemize}
            \item In the absence of external electron acceptors, glycolysis/fermentation provides the means by which cells can generate ATP for biosynthesis. In essence, it allows them to perform internally balanced redox reactions regardless of the external environment.

            \item During one of the steps in glycolysis, NAD+ is reduced to NADH. This step could potentially inhibit further growth because cells have a finite amount of NAD+ available, and if it were all reduced to NADH, then the oxidation of glucose would stop. This limitation is resolved within the cell by oxidizing NADH back into NAD+ through reactions involving the reduction of pyruvate to any of a variety of fermentation byproducts.
        \end{itemize}
        }

        Fig. \ref{fig:fermentation_wiki} shows path of oxidative phosphorylation and fermentation. Fermentation happens to produce NAD+ as well as if no TEA is available.


        Fig. \ref{fig:fermentation_concept_slide} is the summary of review paper of \cite{Lovley1995} and \cite{Apel2011}. Macro molecule of organic matter(MMOM) hydrolyzed to long chain fatty acids (LCFA), aromatics, sugars and amino-acids(AA). Aerobes and denitrifiers can completely oxidize all of hydrolyzed during aerobic respiration and denitrification. Meanwhile, Mn, sulfate and iron reducers can respire only aromatics, LCFA, short chain fatty acids (SCFA) and acetate, but can not respire the sugars and amino acids. \emph{Therefore, in the nitrate depleted anoxic environments organic matter released from hydrolysis of complex organic matter is first metabolized by fermentative microorganism to smaller organics and only then respired by Mn, Sulfate and Iron reducing bacteria.}

        \emph{Organic matter oxidation is coupled to the sequential utilization of terminal electron acceptors (TEAs), typically in the order of O2, NO3-, Mn(VI), Fe(III) and SO42− followed by methanogenesis and/or fermentation} \cite{Arndt2013}

        \emph{Yet, the microorganisms responsible for the hydrolytic degradation of organic macromolecules and the fermentative pathways in anoxic sediments remain to be identified} \cite{Arndt2013}

        \emph{Thus in nitrate-depleted, anoxic environments a significant amount of the organic matter released from hydrolysis of complex organic matter is first metabolized by fermentative microorganisms(Figure 3). Fermentative microorganisms convert large organic compounds to smaller organics} \cite{Lovley1995}

        \emph{the predominant fermentation products are acetate and H2, but other simple fatty acids such as propionate and butyrate are also produced[LovleyandKlug, 1986;LovleyandPhillips, 1989; SOrensen et al., 1981]} \cite{Lovley1995}

        \emph{Because fermentative microorganisms do not completely oxidize organic matter to carbon dioxide, other types of microorganisms operate in conjunction with the fermentative microorganisms in order to bring about a complete oxidation of organic matter in anoxic, nitrate-depleted sediments(Figure 3)} \cite{Lovley1995}

    {\color{red}Todo:
    \begin{itemize}
        \item Check activators and inhibitors.

        Do we need to model them? Can we model them?
    \end{itemize}}



    % subsection fermentation (end)

    \subsection{Growth and Decay rates} % (fold)
    \label{sub:growth_and_decay}

    \emph{Monod's original investigations on the relationship between cell growth rate and carbon source concentration led to the relationship.}:

    \begin{equation}
        r_X = \mu X = \mu_{max} \frac{S}{K+S}\cdot X
    \end{equation}

    And may other (as well as Mulriple Subsrrate Models and "growth rate enhancing" substrates) could be found in \cite{Blanch1981}


    Standard model for biomass \citep{Thullner2007,Boudreau1999,Dale2010,Stolpovsky2011}:

    \begin{equation}
        \frac{\partial [B]}{\partial t} = Y \cdot R_{resp} - \mu_{dec} \cdot [B]
    \end{equation}


    with yield coefficient :

    \begin{itemize}
        \item Theoretical approach by \cite{Rittmann2001} with $R^2 = 0.9$:

        \begin{equation}
            1
        \end{equation}

        \item Theoretical approach by \cite{Heijnen1992} with $R^2 = 0.9$:

        \begin{equation}
            2
        \end{equation}

        limitations! 6 carbons?

        \item Empirical relation by \cite{Roden2011} with $R^2 = 0.949$:

        \begin{equation}
            Y = 0.28 + 0.0211 \cdot (-\Delta G^{'})
        \end{equation}
        where $G^{'}$ is catabolic free energy release.
    \end{itemize}

    \emph{In general, a 2- to 10-fold change in reactant or product concentrations had little effect (10\%) on estimated dG values, as is typical of such calculations \citep{Thauer1977}. Testing showed that the concentration of dissolved \ce{H2} as an end product of fermentation (ethanol, lactate, or glucose) or an intermediate during syntrophic growth (lactate, propionate, or butyrate fermentation coupled to methanogenesis) was the only case that had a dramatic effect on estimated dG} \citep{Roden2011}


    {\color{red}\textbf{Things TODO here:}

    \begin{itemize}
        \item Try to make theoretical approach for cell yield $Y$ versus current Gibbs energy $G^{'}$ based on works of \cite{Heijnen1992} and \cite{Rittmann2001}.

        \item Result of the actual theoretical yield we can compare with \cite{Roden2011} empirical relation or other DB of cell yield.

        \item Read article: \cite{Xiao2008}

    \end{itemize}}

    Also, we may include dormant and active cells \cite{Stolpovsky2011}.

    \emph{For cell growth to occur, an electron-donor substrate is oxidized, and electrons are shuttled to the electron acceptor to generate energy or to the carbon and nitrogen sources to reduce these elements to the oxidation state necessary for incorporation into cells} \citep{VanBriesen2002}

    \subsection{Inhibition of Growth Models} % (fold)
    \label{sub:inhibition_models}

    \emph{Inhibition Models. There are many ways in which an inhibitor may cause a reduction in the cell's metabolic activity. Table 2.3, from Edwards (1970) summarizes these \citep{Blanch1981}:}

    \begin{enumerate}
        \item Modify chemical potential of substrates,intermediates or products;
        \item alter cell permeabilily;
        \item alter enzyme activity;
        \item dissociate one or more enzyme aggregates;
        \item affect enzyme synthesis;
        \item influence functional activity of cell.
    \end{enumerate}

    \emph{ Most models of inhibition, either by substrate or product, can be related to inhibition, of one or more enzymes involved in cellular metabolism. Yano, et a1 (1966), considered the formation of multiple inactive enzyme-substrate,complexes} \citep{Blanch1981}:

    \begin{equation}
        \mu = \mu m \frac{1}{\left[1+K_S/S+\sum (S/K_i)\right]^i}
    \end{equation}

    \emph{This equation was used by Boon and Laudelout (1962) to fit the inhibition of Nitrobacrer by nitrite, and by Edwards in comparing various data on substrate inhibition} \citep{Blanch1981}:

    \begin{equation}
        \mu = \mu m \frac{S}{(K_S + S)(1+S/K_i)}
    \end{equation}


    \emph{When ionic substrates are involved, a model based on Debye-Huckel Theory for the enzymatic reaction rate can be written analogously for microbial growth:} \citep{Blanch1981}:

    \begin{equation}
        \mu = \mu m \frac{S-I\cdot exp(1.17)}{S+K_S(1+I/K_i)}
    \end{equation}

    \emph{One (and many others in \cite{Blanch1981}) model which has the form of the above equation is}:

    \begin{equation}
        \mu = \mu m \frac{S \cdot exp(-S/K_i)}{K_S+S}
    \end{equation}

    \subsection{Lag Phase. Response to Step and Pulse Changes.} % (fold)
    \label{sub:lag_phase_response_to_step_and_pulse_changes_}

    \emph{When either batch or continuous cultures are perturbed, it is apparent from the complexity of the microbial response that a simple unstructured model is not an adequate descriptor of the system. Models, such as the Monod equation, predict an instantaneous response to changes in the external environment (temperature, pH or substrate concentration).Early experimental continuous culture studies (Mateles, Ryu and Yasuda (1965); Aiba, er al. (1967); Gilley and Bungay (1967); Storer and Gaudy (1969)) showed that the specific growth rate did not change instantaneously with changes in substrate concentration or dilution rate. Mateles, et al. (1965) showed that a very rapid increase in specific growth rate was possible following an increase in dilution rate, and this was followed by a more gradual increase as the cells adjusted to the new dilution rate. Young, Bruley and Bungay (1970) discuss experimental data indicating the existence of a lag in the growth rate response of the cell to increases in substrate concentration. } \citep{Blanch1981}

    \emph{This transfer function approach was further extended by Young and Bungay (1973) in considering their response of S. cerevisiae to step changes in substrate concentration, pH, temperature and dilution rate. Giiley and Bungay (1967) and Zines and Rogers (1970) also used pulse and frequency response techniques to determine the form of the transfer function relating specific growth rate and substrate concentrationHarrison and Topiwala (1974) review much of the literature on the 1.ransient response of microbial cultures. No clear kinetic model which described the specific growth rate emerges, and often the cell and product yield coefficients are found to be time dependent under transient conditions. An example of this is given by Regan, et a/.(1971). and by Brookes and Meers (1973). Metalova, er a/.(1972) show an increase in penicillin yield by pulse addition of substrate}. \citep{Blanch1981}

    \emph{p.208: Borzani and Hiss (1979) developed a model to describe their results from quasisteady state operation, based on adsorption of substrate by the cell and transport of substrate through the cell membrane. This gives rise to a second order differential equation describing cell concentration, which shows the appropriate oscillatory behavior.There has been little reported for the other extreme in forced systems, relaxed steady state operation, where the input cycling is so rapid the organism is unable to follow it and enters a "relaxed" steady state. Both this and the quasi-steady state operation, as well as the intermediate regimes, have been analyzed in detail for various types of chemical reactions. Much of the relevant literature is summarized by Bailey (1977). The intermediate regime, however, has been examined for a variety of organisms (Sundstrom, et al. (1 976); Megee (1971) Brooks and Meers (1973).} \citep{Blanch1981}

    \emph{p.208: From the preceding examples, the behavior of microbial cells in situations where the external environment is varying is difficult to predict and most simple models of cell are inadequate} \citep{Blanch1981}

    \emph{In some instances it may be possible to identify key metabolic sequences and model the kinetics of the feedback control loops operating within the cell, in order to adequately predict the cell behavior. This is a complex problem and the coupling of physical transport of nutrients and intracellular kinetics leads to exceedingly complex mathematical models. Some insights into the general formula- tion of models describing metabolism can be gained from the review of Heinrich, Rapoport and Rapoport (1977). This then leads us to the formulation of models describing the growth of a single cell, based on nutrient transport, a simplification of the variety of cellular reactions.which utilize these nutrients, and postulates on the mechanism of cell division. Such models must, of necessity, be computer simulations and involve a large number of unknown reaction rate constants. They do, however, recognize the real complexity of the cellular growth process. Shuler, Leung and Dick (1979) summarize many of these type of models, and develop their own description for the growth of a bacterial cell.} \citep{Blanch1981}

    \subsection{Other factors in rates} % (fold)
    \label{sub:other_factors}

    We may also include other terms:

    \begin{itemize}
        \item Thermodynamic factor \citep{Jin2005, Dale2006, Dale2008, LaRowe2011, Regnier2011, Arndt2013}:

        \begin{equation}
            F_T = \frac{1}{e^{ \frac{\Delta G_r + F \Delta \Psi }{RT} } + 1}
        \end{equation}
        \item Temperature effect \citep{Middelburg1996,Davidson2006,Burdige2011,Arndt2013}:

        \begin{equation}
            F_{t} = e^{ -(E_a / RT)}
        \end{equation}

        NOTE: \emph{it is a semi-empirical formulation that has been derived for elementary reactions (e.g. Benson, 1976). In addition, apparent values of A and Ea are generally calculated from rate measurements, although the Arrhenius equation relates the reaction rate constant, k, and not the rate to temperature.} \citep{Arndt2013}


        \emph{The reaction rate constant in Eq. (3.4) depends on temperature. This dependence can be modelled via the Arrhenius equation (Lasaga, 1998) as \citep{Leal2015}}:

        \begin{equation}
            k = k^{\circ} exp\left[  - \frac{Ea}{R} \left( \frac{1}{T} - \frac{1}{298.15} \right)\right]
        \end{equation}



    \end{itemize}

    \subsection{Equilibrium vs Kinetic approaches} % (fold)
        \label{sub:equilibrium_vs_kinetic_approaches}

     \emph{Frequently geochemical investigations of a system assume chemical equilibrium conditions. Calculating the solubilities of minerals and gases in aqueous solutions at different temperatures, pressures, salinities, and other conditions require only equilibrium calculations (Anderson and Crerar, 1993). Sometimes, however, one might be interested in the time scales over which such pro- cesses occur, and equilibrium calculations will not provide this.} \citep{Leal2015}

     \emph{calculating the evolution of the system compo- sition demands rate laws of the reactions. In addition, due to its time-dependence, chemical kinetics consists of solving a system of ordinary differential equations, while chemical equilibrium requires only the solution of algebraic equations} \citep{Leal2015}

     \emph{Another complexity in geochemical kinetics is the broad difference of the speeds of the aqueous, gaseous and mineral reactions (Lasaga, 1998). Commonly, aqueous reactions proceed substan- tially faster than mineral reactions, with the former sometimes achieving equilibrium in microseconds, while the latter requiring several days to many years. Therefore, this can result in an ineffi- cient numerical integration of the ordinary differential equations, requiring tiny time steps to ensure accuracy and stability.} \cite{Leal2015}

     \emph{To address this problem, we consider the geochemical system to be in partial equilibrium (Helgeson, 1968; Helgeson et al., 1969, 1970). A system in partial equilibrium means that it is in equilibrium with respect to some reactions and out of equilibrium with respect to others. For example, since aqueous and gaseous reactions are often considerably faster than mineral reactions, it seems plausible to assume they are in equilibrium at all times. As the mineral reactions proceed kinetically, the aqueous and gas- eous reactions are constantly perturbed and then instantaneously re-equilibrated.} \cite{Leal2015}

     \emph{The partial equilibrium assumption simplifies the problem by replacing stiff differential equations by algebraic ones...As a result, the governing equations become a system of non-linear differential–algebraic equations, with the constraint that mass of the chemical elements in the system must be conserved and charge balance of an electro-lyte solution attained.} \cite{Leal2015}

     \emph{The first work on computational reaction path modelling in geochemistry can be tracked to Helgeson (1968) and Helgeson et al. (1969). They presented a modelling example of the hydrolysis of K-feldspar where the partial equilibrium assumption was adopted by considering the aqueous reactions in equilibrium.} \cite{Leal2015}

     \emph{The following is a list of computer codes commonly used for geochemical kinetics modelling: EQ6 (Wolery and Daveler, 1992), PHREEQC (Parkhurst and Appelo, 1999, 2013), MINTEQA2 (Allison and Kevin, 1991), CHESS (van der Lee and Windt, 2002), SOLMINEQ.88 (Kharaka et al., 1988), and The Geochemist’s Work- bench (Bethke, 2007). They calculate the evolution of systems as minerals kinetically dissolve or precipitate. In addition, The Geo- chemist’s Workbench, as described in Bethke (2007), provides sup- port for modelling redox reactions controlled by kinetics.} \cite{Leal2015}


     \emph{As discussed in Leal et al. (2013, 2014), these geochemical pack-ages adopt a stoichiometric approach for aqueous speciation calcu- lations. Their databases contain only the equilibrium constants of the reactions, which are required for the solution of the system of mass action equations. The chemical potentials of the species, on the other hand, are not available, which are needed to calculate the Gibbs free energy of the system. Therefore, determining the stable equilibrium phase assemblage of the system is a difficult task, since, given two or more states, it is not possible to determine which} \cite{Leal2015}

     \emph{Recently, Mironenko and Zolotov (2011) developed a computer code for modelling equilibrium-kinetics of water–rock interac-tions. Instead of using a stoichiometric scheme for chemical equi-librium calculations, they used the algorithm of de Capitani and Brown (1987), which minimises the Gibbs free energy of the sys-tem using a convex simplex approach. The chemical equilibrium algorithm of Leal et al. (2014) used here, however, is capable of minimising non-convex objective functions using a trust-region interior-point method.} \cite{Leal2015}


     \emph{The reactions in geochemical systems proceed with different speeds. Their time scales can differ from each other by several orders of magnitude, ranging from microseconds to years (Lasaga, 1998). Langmuir (1996) presents a list containing some common geochemical reactions and their respective half-times.} \cite{Leal2015}

     \emph{Thus, it is reasonable to consider that the aqueous solutes are in equilibrium at all times during the process, while cal-cite is kinetically reacting, and thus out of equilibrium with them. This assumption has also been adopted by Lichtner (1985), Steefel and Cappellen (1990) and Steefel and Lasaga (1994).} \cite{Leal2015}

     \emph{ The partial equilibrium assumption eliminates the dependence of the calculations on the time scales of the fast reactions. Because only the slow reactions are assumed to be controlled by kinetics, while the fast reactions are controlled by equilibrium, the rate laws of the latter are no longer necessary. Their equilibrium conditions are governed by algebraic constraints instead of differential ones.} \cite{Leal2015}

     \emph{The reaction rates given by Eqs. (4.3) and (4.4) were determined far from equilibrium. Near equilibrium, these rates should continu- ously decrease to zero in order to reproduce the eventual equilib- rium state of the mixture of calcite and CO2 saturated water. To account for this behaviour, the following rate equations were adopted instead:} \cite{Leal2015}:

     \begin{equation}
     \label{eq:near_eq}
         r = k \cdot a_i \left( 1 - \frac{Q_1}{K_1} \right)
     \end{equation}


    \emph{In addition, the introduced saturation factors in the reaction rates (\ref{eq:near_eq}) are not necessarily correct, and some adjustments could be applied:}

    \begin{equation}
        r = k \cdot a_i\left[ \left( 1 - \frac{Q_1}{K_1} \right)^\xi \right]^\nu
    \end{equation}

    \subsection{Parameters} % (fold)
    \label{sec:parameters}

    \emph{The parameters k Ea etc for several minerals can be found in Palandri and Kharaka (2004).} \cite{Leal2015}

    \emph{The standard chemical potentials l?i of the species were obtained using the equations of state of Helgeson and Kirkham (1974a), Helgeson et al. (1978), Tanger and Helgeson (1988), Shock and Helgeson (1988) and Shock et al. (1992). For this, we used the parameters of the database file slop98.dat from the software SUPCRT92 (Johnson et al., 1992) and the equation of state of Wagner and Pruss (2002) to calculate the density of water and its temperature and pressure derivatives. The equilibrium con- stants Km of the mineral reactions (3.6) were calculated using the standard chemical potentials of the participating species.} \cite{Leal2015}

    \emph{The following activity and fugacity coefficient models were adopted in our calculations:
1. the HKF extended Debye–Hückel activity coefficient model for solvent water and ionic species, Helgeson and Kirkham (1974a,b, 1976), Helgeson et al. (1981);
2. the Setschenow activity coefficient model for neutral aqueous species other than CO2(aq);
3. the activity coefficient model of Duan and Sun (2003) for CO2(aq);
4. the fugacity coefficient models of Spycher et al. (2003) for CO2(g) and H2O(g)} \cite{Leal2015}

    \begin{tabular}{|c|c|c|c|c|}
        \hline
        Parameter & Symbol & Value & Units & Source \\
        \hline
        Glucose? & $k_{resp}$ & $10^{-6}$ & $\frac{mmol}{mg \cdot s}$ & \cite{Ingvorsen1984} \\

        \hline
        Glucose? & $k_m^{TED}$ & $0.07$ & $mmol$ & \cite{Ingvorsen1984} \\

        \hline
        glucose? & $Y$ & $4.3$ & $\frac{mg}{mmol}$ & \cite{Ingvorsen1984} \\

        \hline
        Reactivity of dead biomass & $k_D$ & $10$ & $yr^{-1}$ & \cite{Westrich1984,Dale2010} \\

        \hline
    \end{tabular}

    \begin{table}[h]
    \label{Hydrolisis rates}
    \begin{tabular}{|c|c|c|c|}
    \hline
    Value & Units & Additional info & Source \\
     \hline
     $4.27 \cdot 10^{-11}$ & $s^{-1}$ & Most reactive & \cite{Small2008} \\

     \hline
     $4.27 \cdot 10^{-12}$ & $s^{-1}$ & Less reactive & \cite{Small2008} \\

     \hline

     $5.6 \cdot 10^{-4}$ & $s^{-1}$ & inhibition by Cellobiose & \cite{Kinetics2004} \\

     \hline

     $0.735 \cdot 10^{-4}$ & $L \cdot g \cdot C^{-1} \cdot h^{-1}$ & Second order: $k[C][X]$ & \cite{Holwerda2013} \\

    \hline
    \end{tabular}
    \captionof{table}{Hydrolysis rates constants for cellulose}
    \end{table}

    \section{Equilibrium reactions} % (fold)
    \label{sec:equilibrium_reactions}

    \subsection{Constants} % (fold)
    \label{sub:constants}

    % subsection constants (end)
    Millero (1995) has established a set of K versus S and T relations for the equilibrium constants. \citep{Boudreau1997Diagenetic}



    \section{Stability}

        \subsection{Monostability} % (fold)
        \label{sub:monostability}

        There are different types of responses to perturbation of the system. Some of the systems are monostable. This means that response depends directly on level of stimulus and when stimulus removed - response returns to prior level.

        % subsection monostability (end)

        \subsection{Bistability} % (fold)
        \label{sub:at_the_biochemical_level_how_do_bistability_arise_}

        \emph{Another type of stability arises in cell division and apoptosis. When a cell decide that it's going to divide or not or it's going to die or not} (Works for eukaryotes not sure about prokaryotes). Bistable systems are systems that can switch between two stable steady-states but cannot rest in intermediate states \citep{Ferrell2002}. Two stable steady-states are low and high state.

        There are many different signaling circuits which could produce bistability: double negative feedback (mutual activation or mutual inhibition) \citep{Gardner2000}  or single positive/negative feedback loop \citep{Becskei2001}. These types can produce bistability, but they do not guarantee one \citep{Ferrell2002}. Therefore, we need quantitative analyses.
        \begin{itemize}
            \item Persistence.
            Bistability is property of biological processes and require persistence.
            \item All-or-none.
            Bistability are not graded. The principle is all-or-none.
        \end{itemize}

        % subsection at_the_biochemical_level_how_do_bistability_arise_ (end)

        \subsection{Multistability} % (fold)
        \label{sub:multistability}

        More recent study \citep{Thomson2009} developed a new insight into the process of phosphorylation and represent them as multistable system.

        % subsection multistability (end)

    \section{Quantum theory} % (fold)
    \label{sec:quantum_physics_notes}

    \subsection{Quantization of energy} % (fold)
    \label{sub:quantization_of_energy}

    Experiment with the objects getting cold (fig. \ref{fig:object_cold}). This implies that only certain levels of energy are allowed and others are not.

    \begin{figure}[tb]
        \begin{center}
            \includegraphics[width=\textwidth]{object_cold}
        \end{center}
        \caption{Objects getting cold}
        \label{fig:object_cold}
    \end{figure}

    The energy is quantized by Planck's constant $h$. Planck suggested that radiated energy can come only in quantized packets of size:
    \begin{equation}
         E = h\nu
    \end{equation}

    where $E$ - energy, $h$ - Planck's constant $6.626\cdot 10^{-34}$ $[J \cdot s]$, $\nu$ - frequency of radiation ($s^{-1}$). Other representations of the same idea:

    \begin{itemize}
        \item via wavelength, $ \lambda $ $[m]$, using relation $\nu\lambda = c$:

        \begin{equation}
            E = h \frac{c}{\lambda}
        \end{equation}

        \item via wavenumber $\bar{\nu}$ $[m^{-1}]$:

        \begin{equation}
            E = hc \bar{\nu}
        \end{equation}
    \end{itemize}

    \subsection{1d Schrodinger equation} % (fold)
    \label{sub:1d_schrodinger_equation}

    The 1d Schrodinger equation:

    \begin{equation}
        -\frac{\hbar^2}{2m} \frac{\partial^2}{\partial x^2} \Psi_n(x) + V(x) \Psi_n(x) = \varepsilon_n \Psi_n(x)
    \end{equation}

    where 1st term is kinetic energy, 2nd - potential and 3rd - allowed total energy (allowed levels of energy). Solving the Schrodinger equation for a given (i) potential V and (ii) set of boundary conditions yields a set of wave functions, $\Psi_n$, and a set of associated energies, $\varepsilon_n$, that are said to be “allowed”. The integer index n specifies the state.

    $|\Psi(x)|^2dx$ is the \textbf{probability} that the system is located between x and dx. \emph{In the absence of an experiment, there is no objective reality.}

    \subsection{Hydrogen chlorine canon} % (fold)
    \label{sub:hydrogen_chlorine_canon}

    What if we mix \ce{Cl2} with \ce{H2}? Will we sponteniously get \ce{HCl}?

    \begin{equation}
        \ce{Cl2 + H2 -> 2HCl}
    \end{equation}

    where \ce{Cl2} bond = 242 kJ/mol; \ce{H2} bond = 436 kJ/mol; \ce{HCl} bond = 431 kJ/mol. Breaking 2 bonds (\ce{H2} and \ce{Cl2}) casts and making 2 bonds releases (when you make a bond it releases energy):

    \begin{equation}
    \label{H_HCL}
        \Delta H = (436 + 242) - 2 \cdot 431 = -184 \si{\kilo\joule\per\mole}
    \end{equation}

    \begin{figure}[tb]
        \begin{center}
            \includegraphics[width=\textwidth]{reaction_HCL}
        \end{center}
        \caption{Reaction in hydrogen chlorine canon}
        \label{fig:reaction_HCL}
    \end{figure}

    \begin{wrapfigure}{r}{0.5\textwidth}
        \begin{center}
            \includegraphics[width=0.48\textwidth]{wavelenths}
        \end{center}
        % \vspace{-15pt}
        \caption{Wavelength image from Universe by Freedman and Kaufmann.}
        \label{fig:wavelenths}
    \end{wrapfigure}

    Therefore, to break the bond of \ce{Cl2} molecule to initiate the reaction. We can use laser light to initiate the reaction.

    Q: Which color of laser will be sufficient to break the bond (red, green or blue)?

    A: We can convert from kJ/mol to kJ/bond \ce{Cl2} broken by dividing by Avogadro's number:

    \begin{equation}
        \frac{246\cdot 10^3}{6.022\cdot 10^{23}} = 4.085 \cdot 10^{-19} \si{\joule}
    \end{equation}

    We can then rearrange and solve for the wavelength, $\lambda$:

    \begin{equation}
        \lambda = \frac{hc}{E}
    \end{equation}

    \begin{equation}
        \lambda = \frac{6.626 \cdot 10^{-34} \cdot 2.998 \cdot 10^8}{4.085 \cdot 10^{-19}}
    \end{equation}

    The answer is that the wavelength of laser should be shorter than 486.3 nano meters in order to to initiate the reaction in the hydrogen chlorine canon which is blue light.

    % \begin{wrapfigure}{r}{0.3\textwidth}
    %     % \vspace{-20pt}
    %     \begin{center}
    %         \includegraphics[width=0.28\textwidth]{tube_HCL}
    %     \end{center}
    %     \label{fig:tube_HCL}
    % \end{wrapfigure}

    Q: What happens once the reaction is started?

    \begin{itemize}
        \item According to the eq. \ref{H_HCL} the reaction releases 184 kJ/mol in heat.
        \item The temperature will go up (assume an ideal diatomic gas at high temp):

        \begin{equation}
            \Delta U = C_V \Delta T
        \end{equation}

        \begin{equation}
            \Delta T = \frac{\Delta U}{C_V} = \frac{\Delta U}{ \frac{7}{2} R } = 6323 \si{\kelvin}
        \end{equation}

        Heat capacity for diatomic ideal gas $C_V$ equals 7/2 of R (universal gas constant).

        \item At constant volume, the pressure will go up (Amonton’s Law):

        \begin{equation}
            \frac{P_2}{P_1} = \frac{T_2}{T_1}
        \end{equation}

        \begin{equation}
            P_2 = 1 \frac{300+6323}{300} = 22
        \end{equation}

        The pressure in the tube will be increased to 22 atm and will force the canon to shot.

    \end{itemize}

        \subsection{How is energy stored within an atom. Atomic energy levels} % (fold)
        \label{sub:atomic_energy_levels}

        Atomic energy levels are the allowed energies in an atomic system. Connecting macroscopic thermodynamics to a molecular understanding requires that we understand how energy is distributed on a microscopic scale. How is energy stored within an atom:

        \begin{itemize}
            \item \textbf{Electrons.} Electronic energy. Changes in the kinetic and potential energy of one or more electrons associated with the nucleus.

            One of the simplest systems chemically is the hydrogen atom. So the hydrogen atom is one electron surrounding one proton. The Schrodinger equation that describes the motion of an electron about a proton can be solved exactly, analytically, and from that solution, we learn the following. There are infinite number of quantized energy levels. And, in terms of terminology, we refer to the lowest allowed energy level as the ground state of a system.

            There is no simple formula for the electronic energy levels of atoms having more than one electron. Instead, we can refer to tabulated data from Moore table.

            \item \textbf{Nuclear motion.} Translational energy. The atom can move (translate) in space — kinetic energy only.

            In addition to electronic energy, atoms have translational energy. In 1D, motion is along the x dimension and the particle is constrained to the interval 0 ≤ x ≤ a. The allowed states are non-degenerate:

            \begin{equation}
                \varepsilon_n = \frac{n^2h^2}{8ma^2}
            \end{equation}

            where n = 1, 2, 3, etc.


            In 3D states can be degenerate. E.g., if a = c then the two different states b (nx = 1, ny = 1, nz = 2) and (nx = 2, ny = 1, nz = 1) have the same energy:

            \begin{equation}
                \varepsilon_{n_xn_yn_z} = \frac{h^2}{8m} \left( \frac{n_x^2}{a^2} + \frac{n_y^2}{b^2} + \frac{n_z^2}{c^2}\right)
            \end{equation}
        \end{itemize}



        \begin{figure}[tb]
            \begin{center}
                \includegraphics[width=\textwidth]{eel_H}
            \end{center}
            \caption{Electronic energy levels of hydrogen atom}
            \label{fig:eel_H}
        \end{figure}

        \subsection{How is energy stored in a Molecule?} % (fold)
        \label{sub:how_is_energy_stored_in_a_molecule_}

        The energy stored in a molecule:

        \begin{itemize}
            \item \textbf{Electronic energy.} Changes in the kinetic and potential energy of one or more electrons associated with the molecule. Same as many-electron atoms.

            \item \textbf{Translational energy}. The molecule can move (translate) in space. Same particle-in-a-box solutions as for atoms.

            \item \textbf{Rotational energy}. The entire molecule can rotate in space. Schrodinger equation: Rigid-rotator.

            \item \textbf{Vibrational energy}. The nuclei can move relative to one another in space. Schrodinger equation: Quantum-mechanical harmonic oscillator.
        \end{itemize}

    \subsection{Wave-particle duality} % (fold)
        \label{sub:wave_particle_duality}

        Classical theory predicts that increasing of intensity of light should lead to more energetic photoelectrons according to:

        \begin{equation}
            \vec E (\vec r,t) = \vec{E_0} \cdot e^{i(\vec k \cdot \vec r - \omega t)}
        \end{equation}

        Therefore, it was expected in the experiment of \cite{Lenard1902} (Fig. \ref{fig:Lenard_exp}) that light of any frequency should be able to kick electrons out of the metal plate. But experiment showed exactly the opposite:

        \begin{figure}[tb]
            \begin{center}
                \includegraphics[width=\textwidth]{Lenard_exp}
            \end{center}
            \caption{Photoelectric experiment of \cite{Lenard1902}}
            \label{fig:Lenard_exp}
        \end{figure}

        \begin{enumerate}
            \item The energy of electrons did not depend on the intensity.
            \item No photoelectrons were produced if the frequency was smaller that a critical value.
        \end{enumerate}

        \cite{Einstein1905} suggested explanation of the photoelectric effect as the following : \emph{'According to the concept that the incident light consists of energy quanta...however, one can conceive of the ejection of electrons by light in the following way. Energy quanta penetrate into the surface layer of the body, and their energy is transformed, at least in part, into kinetic energy of electrons. The simplest way to imagine this is that a light quantum delivers its entire energy to a single electron: we shall assume that this is what happens...' }:

        According to \cite{Einstein1905} the energy of the photon with the wave-length, $\lambda$ :
        \begin{equation}
        \label{eq:particle_electron}
            E = \frac{hc}{\lambda} = \hbar \omega
        \end{equation}

        where $h = 2 \pi \hbar = 6.626 \cdot 10^{-34}$ $J \cdot s$ - the Planck constant.

        A. Einstein received the Nobel Prize in Physics \emph{"for his services to Theoretical Physics, and especially for his discovery of the law of the photoelectric effect"}.

        On the other hand, interesting accident happened in the Bell lab in 1927. \cite{Davisson1927} observed the diffraction of electrons and concluded that electrons behaves as waves: \emph{"The most striking characteristic of these beams is a one to one correspondence ...which the strongest of them bear to the Laue beams that would be found issuing from the same crystal if the incident beam were a beam of x-rays. Certain others appear to be analogues ... of optical diffraction beams from plane reflection gratings. Because of these similarities ... a description ... in terms of an equivalent wave radiation ... is not only possible, but most simple and natural. This involves the association of a wavelength with the incident electron beam, and this wavelength turns out to be in acceptable agreement with the value $h/mv$ ……, Planck's action constant divided by the momentum of the electron"}.

        Davisson and Thomson received the Nobel Prize in 1937 \emph{"for their experimental discovery of the diffraction of electrons by crystals"}.

        Such of electron Beauvoir was predicted by Prince de Broglie in 1925 in his PhD thesis \citep{Haslett1972}. De Broglie received Nobel Prize in 1929 \emph{"for his discovery of the wave nature of electrons"}:

        \begin{equation}
        \label{eq:wave_electron}
             \lambda = \frac{h}{mv}
         \end{equation}

        \begin{figure}[tb]
        \centering
            \begin{center}
                \includegraphics[width=\textwidth]{Davisson_exp}
            \end{center}
            \caption{What do \cite{Davisson1927} observed in the experiment.}
            \label{fig:Davisson1927_exp}
        \end{figure}

        \begin{figure}[tb]
        \centering
            \begin{center}
                \includegraphics[width=\textwidth]{Davisson_exp2}
            \end{center}
            \caption{What do \cite{Davisson1927} actually saw.}
            \label{fig:Davisson1927_exp2}
        \end{figure}

        Summary:

        \begin{enumerate}
            \item There is clear experimental evidence that light behaves (sometimes) as a beam of particles carrying energy quanta (fig. \ref{fig:Lenard_exp}).

            \item On the other hand, there is also evidence that electrons (sometimes) behave as waves. (fig. \ref{fig:Davisson1927_exp2})

        \end{enumerate}

        % subsection wave_particle_duality (end)

        \subsection{Schrodinger Equation "derivation" from solution} % (fold)
        \label{sub:schrodinger_equation_derivation}


        The wave function is:

        \begin{equation}
        \label{eq:wave}
          \Psi = e^{i(kx-\omega t)}
        \end{equation}

        The total energy is kinetic energy plus potential energy:

        \begin{equation}
            E = KE + PE
        \end{equation}

        We ussualy write this as:

        \begin{equation}
            E = \frac{1}{2}mv^2+V
        \end{equation}

        where $V$ is potential energy depending where we happened to be (the gravitational field or electric field etc). We can also rewrite it as:

        \begin{equation}
        \label{eq:energy}
           E = \frac{p^2}{2m}+V
        \end{equation}

        Multiply by $\Psi$:
        \begin{equation}
        \label{eq:energy_psi}
            E \Psi= \frac{p^2\Psi}{2m}+V\Psi
        \end{equation}

        Due to $p=mv$ and therefore:
        \begin{equation}
          \frac{1}{2}mv^2 =  \frac{m^2v^2}{2m} = \frac{p^2}{2m}
        \end{equation}

        Lets differentiate wave function (eq. \ref{eq:wave}) with respect to $x$:

        \begin{equation}
            \frac{d\Psi}{dx}=ike^{i(kx-\omega t)}=ik\Psi
        \end{equation}

        Second derivative of the wave function with respect to $x$:

        \begin{equation}
        \label{eq:second_rerive_schr}
            \frac{d^2\Psi}{dx^2} = (ik)^2\Psi
        \end{equation}


        We know what:

        \begin{equation}
            k=\frac{2\pi}{\lambda}
        \end{equation}

        Then:

        \begin{equation}
        \label{eq:lambda}
            \lambda = \frac{2\pi}{k}
        \end{equation}

        At the same time momentum is:
        \begin{equation}
        \label{eq:momentum}
            p=\frac{h}{\lambda}
        \end{equation}

        Substitute eq. (\ref{eq:lambda}) into eq. (\ref{eq:momentum}) we receive:
        \begin{equation}
            p=\frac{hk}{2\pi}=\hbar k
        \end{equation}

        Therefore:

        \begin{equation}
        \label{eq:k}
            k=\frac{p}{\hbar}
        \end{equation}

        Then substitute eq. (\ref{eq:k}) into eq. (\ref{eq:second_rerive_schr}):
        \begin{equation}
            \frac{d^2\Psi}{dx^2} = -1(\frac{p^2}{\hbar^2})\Psi
        \end{equation}

        And rearrange:

        \begin{equation}
        \label{eq:second_deriv_schrodinger}
            -\hbar^2 \frac{d^2\Psi}{dx^2} = p^2 \Psi
        \end{equation}


        Substitute eq. (\ref{eq:second_deriv_schrodinger}) into eq. (\ref{eq:energy_psi}) we receive \textbf{Time Independent Schrodinger Equation}:

        \begin{equation}
        \label{eq:TISE}
            E \Psi= \frac{-\hbar^2}{2m} \frac{d^2\Psi}{dx^2} +V\Psi
        \end{equation}

        Lets differentiate wave function (eq. \ref{eq:wave}) with respect to $t$:

        \begin{equation}
        \label{eq:wave_time_deriv}
            \frac{d\Psi}{dt} = -i\omega\Psi
        \end{equation}

        But we also know what:
        \begin{equation}
            E = \hbar\omega
        \end{equation}

        Multiply by $\Psi$:

        \begin{equation}
        \label{eq:total_energy}
            E\Psi = \hbar\omega\Psi
        \end{equation}

        Lets divide eq. (\ref{eq:total_energy}) by $\hbar$ and multiply by $-i$:

        \begin{equation}
            \frac{-i}{\hbar}E\Psi = -i\omega\Psi
        \end{equation}

        Looking into eq. (\ref{eq:wave_time_deriv}) we see that:

        \begin{equation}
            \frac{d\Psi}{dt} = - \frac{i}{\hbar}E\Psi
        \end{equation}

        Rearranging:

        \begin{equation}
            E\Psi = - \frac{\hbar}{i} \frac{d\Psi}{dt} = i\hbar \frac{d\Psi}{dt}
        \end{equation}

        due to:
        \begin{equation}
            i=\frac{1}{-i}
        \end{equation}

        \begin{equation}
            E\Psi = i\hbar \frac{d\Psi}{dt}
        \end{equation}

        Substitute into TISE (eq. \ref{eq:TISE}):

        \begin{equation}
            i\hbar \frac{d\Psi}{dt} = - \frac{\hbar^2}{2m} \frac{d^2\Psi}{dx^2} +V\Psi
        \end{equation}

        We receive Time Dependent Schrodinger Equation (TDSE):

        \begin{equation}
            i\hbar \frac{d}{dt} \Psi(r,t)= - \frac{\hbar^2}{2m} \nabla^2 \Psi(r,t) + V(r)\Psi(r,t)
        \end{equation}

        Or:

        \begin{equation}
            i\hbar \frac{d}{dt} \Psi = \hat H \Psi
        \end{equation}

        It is good to remember that Hamiltonian $\hat H$ is:

        \begin{equation}
            \hat H = \hat T + \hat V
        \end{equation}

        where $\hat V = V(r,t)$ is potential energy operator and:

        \begin{equation}
            \hat T = \frac{\hat p^2}{2m} = - \frac{\hbar}{2m} \nabla^2
        \end{equation}

        where $\hat T$ is kinetic energy operator with momentum operator $\hat p = -i \hbar \nabla$.
        % subsection schrodinger_equation_derivation (end)
    % section quantum_physics_notes (end)

    \section{Statistical Mechanics} % (fold)
    \label{sec:statistical_mechanics}

    Energy of the molecule:

    \begin{equation}
        E = \frac{3}{2}k_B t_K
    \end{equation}

    Redefine $k_B$ as converter from `human units':

    \begin{equation}
        T = k_B t_K
    \end{equation}

    Carnot's entropy:
    \begin{equation}
        S_{carnot} = - k_B * S_{inf} = - k_B \sum_{i=1}^n P_i \cdot \log P_i
    \end{equation}

    Informational entropy is dimensionless:

    \begin{equation}
    \label{eq:inf_entrop}
        S = \frac{1}{k_B}S_{carnot}  = - \sum_{i=1}^n P_i \cdot \log P_i
    \end{equation}

    \subsection{Temperature} % (fold)
    \label{sub:temp}

    Temperature is not a fundamental quantity, but is derived as the amount of energy required to add an incremental amount of entropy to a system..

    Average Energy of the system:

    \begin{equation}
    \label{eq:mean_E}
        \sum_{i=1}^n P(i,E) \cdot E_i = \mean{E}
    \end{equation}

    Change in the energy:

    \begin{equation}
        \Delta E = \frac{\partial E}{\partial S} \Delta S = T \Delta S
    \end{equation}

    due to $\partial E / \partial S = T$. Or could be written in Carnot's entropy where the Boltzmann constant is canceled out:

    \begin{equation}
        dE = TdS = t_K k_B \frac{1}{k_B} dS_{carnot} = t_K d S_{carnot}
    \end{equation}

    Lecture 3:

    \subsection{Lagrangian multipliers} % (fold)
    \label{sub:lagran}

    Maximazing of eq. \ref{eq:inf_entrop} is the optimization problem. The Lagrangian multipliers is the best technique to solve the problem:

    \begin{equation}
        (S-\lambda_0 1) =  - k_B \sum_{i=1}^n P_i \cdot \log P_i - \lambda_0 \sum_{i=1}^n P_i
    \end{equation}

    Therefore, we need to maximize:

    \begin{equation}
        0 = -k_B \sum_{i=1}^n (\log P_i + 1) \delta P_i - \lambda_0 \sum_{i=1}^n \delta P_i
    \end{equation}

    Then:

    \begin{equation}
        -k_B \log P_i - k_B - \lambda_0 = 0
    \end{equation}

    Therefore:

    \begin{equation}
        P_i = exp(-1 - \frac{\lambda_0}{k_B})
    \end{equation}
    % subsection lagran (end)


    \subsection{Distribution of energy representing the maximum entropy in a system at equilibrium} % (fold)
    \label{sec:distribution_of_energy_representing_the_maximum_entropy_in_a_system_at_equilibrium}

    Average energy in the system (eq. \ref{eq:mean_E}), informational entropy (eq. \ref{eq:inf_entrop}) with statistical constrain:

    \begin{equation}
        \sum_{k=1}^n P_i = 1
    \end{equation}



   \section{Principles of Biochemistry} % (fold)
   \label{sec:principles_of_biochemistry}

   \subsection{Unique properties of carbon that make it the backbone of many of the molecules of life} % (fold)
   \label{ssub:why_carbon_have_so_many_different_}

   Major components of living organisms: H, C, N, O, Na, K, Ca, P, S, Cl.

   Trace amounts: Mg, V, Cr, Mn, Fe, Co, Ni, Cu, Zn, Se, Mo, I.

   Human body: 95\% O, C, H, N.

   Earth's crust: Ca, Fe, Al, Si.

   About C: The structural diversity leads to functional diversity.

   \begin{figure}[!h]
       \centering
       \includegraphics[width=0.65\textwidth]{carbon_orb}
       \caption{Carbon hybrid orbitals - the key to the diversity of carbon bonding}
       \label{fig:label}
   \end{figure}

   \subsection{Protein structure} % (fold)
   \label{sub:protein_structure}

   Basic building of proteins: 20 common natural amino acids.

   Histodine - often found on the active site of the enzymes (pK 6).

   Usually, inside the cell the conditions are reducing and outside is oxidizing.

   \subsection{Bioenergetics} % (fold)
   \label{sub:bioenergetics}

   Glycolysis happens in 10 steps: 1-5 is the investment step to disstabilize the glucose. 6-10 the payoff part.

   The coupling of catabolic and anabolic reactions can happen if:

   \begin{itemize}
       \item sum of $\Delta G$ is negative;
       \item they should share some the common reactants;
       \item the cell should contain the enzyme which can catalyse both reactions.
   \end{itemize}

   \begin{figure}[htbp]
       \centering
       \includegraphics[width=0.65\textwidth]{energy_path_human}
       \caption{Energetic pathways in human body}
       \label{fig:label}
   \end{figure}

    Cell use NAD+ in glycolysis and it can be regenerated via:

    \begin{itemize}
        \item in the oxidative phosphorylation;
        \item by reduction of pyruvate in Lactate or ethanol fermentation.
    \end{itemize}

    Lactate fermentation: muscle operate under anaerobic conditions during workout (not enough oxygen). Human cells utilize lactate fermentation under anaerobic conditions to regenerate NAD+ to support continued glycolysis.

    Ethanol fermentation: reverse version of this fermentation happens in liver.

    During fermentation there is no production of ATP only the regeneration of NAD+.

    Cori cycle: Red blood cells transfer the oxygen in the blood and they do not have organelles such as mitochondria, they use glycolysis(+2ATP) and produce pyruvate and converted to lactate via fermentation. Lactate acidify the hemoglobin and transfered to the liver. In the liver lactate converted back to glucose (high energy cost: -6 ATP). NADH transfers electrons to iron (via cytochrome b5) in red blood cells to regenerate NAD+ needed to drive glycolysis forward and To maintain hemoglobin-bound iron in its reduced (Fe2+) state.

    1 ml of blood - 5e6 of red blood cells - 1 rbc consists of 270e6 of hemoglobin.

    \subsection{e.coli bioenergetics} % (fold)
    \label{sub:bacterial_bioen}

    \begin{figure}[htbp]
        \centering
        \includegraphics[width=0.95\textwidth]{pathways_bacteria}
        \caption{Multiple pathways in aerobic and anaerobic conditions in bacterial activity}
        \label{fig:e.coli_path}
    \end{figure}

    \begin{figure}[htbp]
        \centering
        \includegraphics[width=0.95\textwidth]{linear_vs_branched_catabolism}
        \caption{caption}
        \label{fig:linear_vs_branched_catabolism}
    \end{figure}

    Lactate fermentation - regeneration of NAD+. (no ATP during fermentation only at glycolysis).

    Ethanol fermentation - regeneration of NAD+. (no ATP during fermentation only at glycolysis).

    Acetate fermentation - production of ATP. (no NAD+ regeneration during fermentation).

    So under anaerobic condition, the modulation of the activity of the different fermentation pathways in E. coli will lead to the production of a viable number of molecules of ATP per molecule of glucose consumed. So the yield of these ATP fermentation pathways will be valuable and will be controlled by the needs of E. coli in NAD+ or in energy.

    There are a number of fermentation pathways using
    different precursors.
    And in some cases, the precursors are a product of other fermentation pathways.
    Bacteria do not store a large quantity of fuel molecules.
    So they will depend on the nutrients that happen
    to be available in their surroundings.
    Diverse fermentation pathways have evolved in different bacteria
    too allow the use of a large array of carbon sources.


    \section{Writing in the Sciences} % (fold)
    \label{sec:writing_in_the_sciences}

    \subsection{Principles of effective writing} % (fold)
    \label{sub:principles_of_effective_writing}

    \textbf{What makes good writing?}

    \begin{enumerate}
        \item Good writing communicates an idea clearly and effectively (Takes having something to say and clear thinking);
        \item Good writing is elegant and stylish (Takes time, revision, and a good editor).
    \end{enumerate}

    \textbf{Things to do to become a better writer:}

    \begin{itemize}
        \item Read, pay attention, and imitate;
        \item Write in a journal;
        \item Talk about your research before trying to write about it;
        \item Write to engage your readers—try not to bore them!;
        \item Accept that writing is hard for everyone;
        \item Revise. Nobody gets it perfect on the first try;
        \item Learn how to cut ruthlessly. Never become too attached to your words;
        \item Find a good editor;
        \item Take risks;
    \end{itemize}

    \textbf{To write good sentence:}

    \begin{itemize}
        \item Is this sentence easy to understand?
        Is this sentence enjoyable and interesting to read?
        Is this sentence readable?
        Is it written to inform or to obscure?
    \end{itemize}

    \textbf{Principles of effective writing:}

    \begin{enumerate}
        \item Cut unnecessary words and phrases; learn to part with your words;
        \item Use the active voice (subject + verb + object);
        \item Write with verbs: use strong verbs, avoid turning verbs into nouns, and don’t bury the main verb.
    \end{enumerate}

    \subsection{Cutting} % (fold)
    \label{sub:cutting}

    \begin{enumerate}
        \item Dead weight words and phrases:
        \begin{itemize}
            \item As it is well known
            \item As it has been shown
            \item It can be regarded that
            \item It should be emphasized that
        \end{itemize}
        \item Empty words and phrases:
        \begin{itemize}
            \item basic tenets of
            \item methodologic
            \item important
        \end{itemize}
        \item Long words or phrases that could be short:
        \item Unnecessary jargon and acronyms
        \item Repetitive words or phrases:
        \begin{itemize}
            \item studies/examples
            \item illustrate/demonstrate
            \item challenges/difficulties
            \item successful solutions
        \end{itemize}
        \item Adverbs: very, really, quite, basically, generally, etc.
        \item Long words and phrases that could be short:
        \begin{itemize}
            \item A majority of = most;
            \item A number of = many;
            \item Are of the same opinion = agree;
            \item Less frequently occurring = rare;
            \item All three of the = the three;
        \end{itemize}
        \item Eliminate negatives: did not win = lost;
        \item Eliminate superfluous uses of `there are/there is';
        \item Omit needless prepositions
    \end{enumerate}

    \subsection{Use the Active voice} % (fold)
    \label{sub:use_active_voice}

    \begin{itemize}
        \item Use active voice instead of passive;
        \item It is OK to use `We' and `I';
        \item Passive voice for Methods section;

    \end{itemize}

    % subsection use_active_voice (end)

    \section{Important notes}

        \emph{Respiring microbes live by trapping some of the energy liberated when they catalyze the transfer of electrons from a reduced species such as
        acetate or dihydrogen to an oxidized species like dioxygen, nitrate, ferric iron, or
        sulfate. They use the energy they trap to carry out life functions such as cell mainte-nance, and to create biomass} from \citep{Bethke2011}

        \emph{The initially reduced species consumed by a respirer is the electron donor, and
        the oxidized species is the acceptor. As they transfer electrons, microbes oxidize the
        donor species and reduce the acceptor, directly regulating the redox state and hence
        the quality and chemical properties of the groundwater in which they live....} from \citep{Bethke2011}

        \emph{The idea that microbial activity in aquifers is distributed according to an energetic
        hierarchy of electron accepting processes—a thermodynamic ladder—was formalized
        by Champ and others (1979)...
        } from \citep{Bethke2011}

    \subsection{Geological Correlations} % (fold)
    \label{sub:geological_core}

    Correlations in long time scale:
    \begin{itemize}
        \item Oxygen $\delta \ce{O}$ with temperature;
        \item Salinity with weathering/run off;
    \end{itemize}
    % subsection geological_core (end)


    As a rule, the climate of so remote eras is judged by the isotopic composition of carbon, oxygen, silicon, sulfur, and other elements in the sedimentary rocks of the crust. These data often are unclear. And Sometimes assumption about Proterozoic temperatures are based on the presence of fossils in rocks of various bacteria, but these data are not considered to be reliable and usually cause a lot of objections \citep{Markov2010}.

    \section{Benjamin} % (fold)
    \label{sec:benjamin}

Redox cycle.
Mn less important in soils.

Types of Mn oxidation:

\begin{itemize}
    \item \ce{Mn^{2+}} oxidation by o2 is slow.
    \item Bacteria and fungy produce super oxyde \ce{o2-} and oxidize the Mn and Fe.
    \item Photoreduction. Mn reduced and ligand oxidized.
\end{itemize}

Conductivity affects the redox reactions. LIght excite the electron and shift them up. Excited electron gives reduced state. Explain why? Wherefore only in the upper layers of water have reduction of Mn.

\subsection{Capturing intermediates states of reactions} % (fold)
\label{sub:capturing_intermediates_states_of_reactions}
\begin{itemize}
    \item Pump probe spectroscopy. Excite with laser and wait then measure. give pico and nano seconds.
\end{itemize}

\subsection{Model} % (fold)
\label{sub:model}

light absorbtion. \ce{Mn2+} goes to Mn3+ inti nanosheets.

How reductions happens

reactive oxygen should be within the lattice.

Webb et all (2005)

Great oxidation effect. When oxygen started on earth.

Water-oxidation catalysis by manganise (Nature)

MnO2

Ca2+ catalise the reactions of oxydation of Mn.

\subsubsection{Questions} % (fold)
\label{ssub:questions}
\begin{itemize}
    \item Could you explain why conductivity of oxides affects the redox reactions?

    \item Why exited electrons shift to more reduced state?
\end{itemize}
% subsubsection questions (end)

\section{Conversion of units} % (fold)
\subsection{Energy requirements in biotechnology literature per c-mol of bacteria to per mol} % (fold)
\label{sub:energy_requirements_in_biotechnology_literature_c_mol_of_bacteria}

% subsection energy_requirements_in_biotechnology_literature_c_mol_of_bacteria (end)
\label{sec:conversion_of_units}
For your information, maybe you need it, I obtained the conversion factor from $\frac{kJ}{c-mol \cdot h}$ to $\frac{kJ}{mol \cdot h}$ by this conversion factor(cv):

$\frac{kJ}{c-mol \cdot h} \cdot cv=\frac{kJ}{mol \cdot h}$

where:

$cv=Mwt(\ce{CH_{1.8}O_{0.5}N_{0.2}}) \cdot m_1\cdot N_a$


due to:

$[B] = \frac{N}{N_a}$

$N = \frac{m_{all}}{m_1}$

$m_{all} = \frac{c-mol}{Mwt(CH1.8O0.5N0.2)}$


where:

$[B]$ - concentration of bacteria;
N - number of cells
$m_{all}$ - weight of bacteria;
$m_1$ - weight of 1 cell;
$N_a$ - Avogadro's number;
$Mwt(\ce{CH_{1.8}O_{0.5}N_{0.2}})$ - is molar weight of \ce{CH_{1.8}O_{0.5}N_{0.2}} which is 24.6118.


I am not really sure about my conversion but if we just multiply by 5 then we say that our bacteria consist only of 5 carbons which is not true, I think. Right? I think bacteria consist of 10s of thousands of carbons (depending on the weight of the cell)

Lets try to walk through together and see what we've got:

Concentration of Bacteria:

+ $[B] = \frac{N}{N_a}$

where N- number of cells and Na - avagadro.

+  $[B] = \frac{m_{all}}{m_1 \cdot N_a}$

due to $N = \frac{m_{all}}{m_1}$

+ $[B] = \frac{mol-c}{m_1 \cdot N_a \cdot Mwt}$

due to $m_{all} = \frac{mol-c}{Mwt}$

rearrange:


$[B] \cdot m_1 \cdot N_a \cdot Mwt = mol-c$


Substitute in

$\frac{kJ}{c-mol \cdot h}$


we receive:


$\frac{kJ}{ [B] \cdot m_1 \cdot N_a \cdot Mwt \cdot h}$

to leave only B in the denominator we need to multiply by:

$m_1 \cdot N_a \cdot Mwt$

Therefore:

$\frac{kJ}{c-mol \cdot h} m_1 \cdot N_a \cdot Mwt = \frac{kJ}{mol \cdot h}$

What do you think?


\section{Reacion path model unused part} % (fold)
\label{sec:reacion_path_model_unused_part}


    \subsection{Artificial approach based on $exp(-\Delta_r G_i/RT)$} % (fold)
    \label{ssub:dg_c}

    Artificial mathematical model where exponent raised to a power of positive value of Gibbs free energy(minus in front of fraction) relative to $RT$ yields very large number.

    \begin{equation}
        F^i_{opt} = \frac{exp(-\frac{\Delta_r G_i + F \Delta \Psi}{RT})}{\sum\limits_{n}^{j} exp(-\frac{\Delta_r G_j + F \Delta \Psi}{RT})}
    \end{equation}

    Even small relative change in Gibbs free energy between different reaction give large relative difference for $F^i_{opt}$.

    % subsubsection approach_based_on_statistical_distribution (end)

    \subsection{Approach based on Transition state theory (TST), Electron Transfer Theory (ET) and inhibition mechanics (under development)} % (fold)
    \label{ssub:approach_based_on_the_electron_transfer_theory_and_inhibition_mechanics}


    The first-order rate constant k can be written as:

    \begin{equation}
        k = A \cdot exp \left( -\frac{E_a}{RT}\right)
    \end{equation}

    or according to TST and ET theory \citep{Marcus1985}:

    \begin{equation}
    \label{eq:marcus}
        k = \kappa (r)\cdot \nu \cdot exp\left(- \frac{\Delta_r G_i^\ddagger}{RT}\right)
    \end{equation}

    with dependence on the Gibbs free energy of reaction:

    \begin{equation}
        \Delta_r G_i^\ddagger = \frac{\lambda}{4}\left[ 1 + \frac{\Delta_r G_i}{\lambda} \right]^2
    \end{equation}

    Meanwhile, maybe inhibitors increase the activation energy ($E_a$) of the inhibited reaction. {\color{red} Under development.}

        % section building_the_model (end)
    \subsection{Supporting materials} % (fold)
    \label{sec:how_it_works}

    \begin{figure}[htbp]
        \centering
        \includegraphics[width=0.7\textwidth]{energy_prod}
        \caption{Bacteria. Pictures from \cite{Goodsell2009}}
        \label{fig:energy_prod}
    \end{figure}

    To explain the model in details I would like to start describing the energy production in the cell (this is summary from \cite{Karp2008} and \cite{Goodsell2009}):

\begin{itshape}

    \begin{enumerate}

        \item The first steps of energy production occur outside the cell and called hydrolysis. Hydrolysis break down OM into manageable pieces like glucose that are transported into the cell.

        \item After transferring of the glucose into the cell it starts with glycolysis (A region on the picture) and break it into two pieces. This process is energetic and used to create 2 molecules of ATP. Many organisms stop at this point, using the ATP for energy and discarding the pieces as alcohol.

        \item However, some bacterias add additional steps to gain even more energy out of the OM molecules. The pieces are completely broken down to carbon dioxide in the citric acid cycle (B region on the picture). As the molecules are broken down, high-energy electrons are captured on several carrier molecules such as NAD+ and FAD.

        \item In the final step, termed respiration, these electrons flow through a series of NADH complex, found in the cytoplasmic membrane (C on the picture). And then these electrons are placed by enzymes on TEA (D).

    \end{enumerate}

\end{itshape}

    The main message here is the amount of TEA in the surrounding environment is not the main factor during uptake of carbohydrates and glycolysis. The last steps 3-4 is determined by presence of TEA which bacteria will use during the oxidative phosphorylation. If not then bacteria goes the fermentation pathway.

    Also, there is a difference between solid and solute TEA presented in the environment: if TEA is solute when rates could be modeled as Michaeilis-Menten kinetics \citep{Michaelis1913}, if TEA is solid, then via  Langmuir adsorption isotherm \citep{Bonneville2006}.

    \paragraph{Inhibition.}

    Generally, there are 2 types of inhibition:
    \begin{itemize}
        \item Competitive inhibition
        \begin{itemize}
            \item An inhibitor competes with the substrate for binding to the active site.
            \item Competitive inhibition increases the amount of substrate needed to achieve maximum rate of catalysis.
            \item Competitive inhibition does NOT change the maximum possible rate of the enzyme's catalysis.
            \item You can overcome competitive inhibition by providing more substrate.
        \end{itemize}
        \item Non-competitive inhibition (our case).
        \begin{itemize}
            \item An inhibitor binds to an allosteric site on the enzyme to deactivate it.
            \item The substrate still have access the active site, but the enzyme is no longer able to catalyze the reaction as long as the inhibitor remains bound.
            \item Non-competitive inhibition decreases the maximum possible rate of the enzyme's catalysis.
            \item Non-competitive inhibition does NOT change the amount of substrate needed to achieve the maximum rate of catalysis.
            \item You can't overcome non-competitive inhibition by adding more substrate.
        \end{itemize}
    \end{itemize}


    \subsection{Divide and Conquer} % (fold)
    \label{sub:divide_and_conquer}


    \emph{The general form of metabolism of cells is shown on the fig. \ref{fig:metabol_stages}. Remarkably, the chemical reactions and metabolic pathways described are found in virtually every living cell, from the simplest bacterium to the most complex plant or animal.} \citep{Karp2008}.

    Figure \ref{fig:metabol_stages}: \emph{Once macromolecules have been hydrolyzed into their components - amino acids, sugars, and fatty acids - the cell can reutilize the building blocks directly to: (1) form other macromolecules of the same class (stage I); (2) convert them into different compounds to make other products; or (3) degrade them further (stages II and III) and extract a measure of their free-energy content.} \citep{Karp2008}.

    \begin{figure}[tb]
        \begin{center}
            \includegraphics[width=0.98\textwidth]{metabol_stages}
        \end{center}
        \caption{Three stages of aerobic metabolism. Picture from \cite{Karp2008}}
        \label{fig:metabol_stages}
    \end{figure}

    Figure \ref{fig:fermentation_karp}: \emph{Pyruvate, the end product of glycolysis, is a key compound because it stands at the junction between anaerobic (oxygen-independent) and aerobic (oxygen-dependent) pathways. In the absence of molecular oxygen, pyruvate is subjected to fermentation. When oxygen is available, pyruvate is further catabolized by aerobic respiration.} \citep{Karp2008}.

    Figure \ref{fig:fermentation_karp}: \emph{When oxygen once again becomes available in sufficient amounts, the lactate can be converted back to pyruvate for continued oxidation.} \citep{Karp2008}.

    \begin{figure}[!htb]
        \begin{center}
            \includegraphics[width=0.7\textwidth]{fermentation_karp}
        \end{center}
        \caption{Fermentation. Different pathways depending on the oxygen concentration. Main message: cells need to regenerate the NAD+ to continue glycolysis. Picture from \cite{Karp2008}.}
        \label{fig:fermentation_karp}
    \end{figure}
% section reacion_path_model_unused_part (end)

\subsection{Discussion} % (fold)
\label{sec:discussion}

The optimization approach:

\begin{equation}
    \begin{cases}
        & min \left\{ \sum\limits_{n}^{i} F^i_{opt} \cdot \Delta_r G_i \cdot R_i \cdot n_e \right\} \\
        & \sum\limits_{n}^{i} F^i_{opt} \leq 1
    \end{cases}
\end{equation}

gives the most logical and experimentally  supported results. Meanwhile, this method is highly unstable, very difficult in the implementation and has high demand in computational resources. Meanwhile,the artificial approach:

\begin{equation}
    F^i_{opt} = \frac{exp(-\frac{\Delta_r G_i + F \Delta \Psi}{RT})}{\sum\limits_{n}^{j} exp(-\frac{\Delta_r G_j + F \Delta \Psi}{RT})}
\end{equation}

give quite similar results. There is no theoretical explanation of the term. It should be noted that potentially any number  could be raised into the power of Gibbs free energy and not only the Euler's $e$. For instance, number $\kappa$:

\begin{equation}
    F^i_{opt} = \frac{\kappa^{-\frac{\Delta_r G_i + F \Delta \Psi}{RT}}}{\sum\limits_{n}^{j} \kappa^{-\frac{\Delta_r G_j + F \Delta \Psi}{RT}}}
\end{equation}

This number will impose how the less favorable TEA overlaps previous one (fig. \ref{fig:dg_exp_Rates_TEA_ph7_overlap} and fig. \ref{fig:dg_exp_Rates_TEA_ph7_overlap_10}, $\kappa$ = 1.4 and $\kappa$ = 10 respectively).

    % \begin{figure}[!htb]
    %     \centering
    %     \includegraphics[width=0.65\textwidth]{dg_exp/kappa1/figure4}
    %     \caption{Rates of TEA consumption with $\kappa$ = 1.4.}
    %     \label{fig:dg_exp_Rates_TEA_ph7_overlap}
    % \end{figure}

    % \begin{figure}[!htb]
    %     \centering
    %     \includegraphics[width=0.65\textwidth]{dg_exp/kappa10/figure4}
    %     \caption{Rates of TEA consumption with  $\kappa$ = 10.}
    %     \label{fig:dg_exp_Rates_TEA_ph7_overlap_10}
    % \end{figure}

Choosing different TEA by the bacteria is non-competitive inhibition as stated by \cite{Roden2008}. During non-competitive inhibition enzymes are completely deactivated by binding of inhibitor to allosteric site and distorting the binding site \citep{Karp2008}. This shuts down the reaction completely. It means that there should be the sequence of TEA consumption and, therefore, the artificial approach has best chances.

\clearpage

\section{Future research Ideas} % (fold)
\label{sec:future_research_ideas}


    \subsection{Possible plan of research} % (fold)
    \label{sec:possible_plan_of_research}
    \begin{itemize}
        \item Develop reaction path model
        \item Improving the lake model with Gibbs calculations, Bacteria strain and thermodynamically based inhibiting term
        \item CO2 and CH4 fluxes from the rivers and soils;
        \item Add Eh calculation based on Gibbs Energy and make SVD, PCA, frequency analyses of measurements and modeled data. Relate to Eh measurements
        \item Possible application: Great Lakes due to anthropogenic radioactivity.
    \end{itemize}

    \subsubsection{Develop reaction path model} % (fold)
    \label{sub:develop_reaction_path_model}
        Description in the paper future paper.
    % subsubsection develop_reaction_path_model (end)

    \subsubsection{Improving the lake model with Gibbs calculations, Bacteria strain and thermodynamically based inhibiting term} % (fold)
    \label{sub:improving_the_lake_model_with_gibbs_calculations}

    \subsection{CO2 and CH4 fluxes from the rivers and soils.} % (fold)
    \label{sub:co2_and_ch4_fluxes_from_the_rivers_and_soils_}

    p.346: Only about half of anthropogenic CO2 emissions accumulate in the atmosphere2; and by  -- Highlighted jun 25, 2015. \citep{Wehrli:2013fv}

    p.346: Climate scientists estimate the strength of carbon sinks on land by running global circulation models of the atmosphere, and by p.346: identifying regional sinks using monitoring stations that measure the global distribution and variability of atmospheric CO concen-2 trations. This top-down approach is limited by the spatial resolution of both the models and the data.By contrast, ecosystem scientists approach the problem from the bottom up: they measure the CO2 uptake and release rates of different natural and agricultural vegetation systems. The large spatial and seasonal variability of photosynthesis and respiration poses a significant challenge in scaling up these local observations -- Highlighted jun 25, 2015 \citep{Wehrli:2013fv}.

    p.346: identifying regional sinks using monitoring stations that measure the global distribution and variability of atmospheric CO2 concentrations. This top-down approach is limited by the spatial resolution of both the models and the data.By contrast, ecosystem scientists approach the problem from the bottom up: they measure the CO2 uptake and release rates of different natural and agricultural vegetation systems. The large spatial and seasonal variability of photosynthesis and respiration poses a significant challenge in scaling up these local observations -- Highlighted jun 25, 2015 \citep{Wehrli:2013fv}.

    p.346: rivers and lakes are supersaturated with CO2 — that is, the measured concentration in river water usually exceeds the equilibrium value for the exchange between CO2 present as a gas in the atmosphere and that present as a dissolved substance in water4. -- Highlighted jun 25, 2015 \citep{Wehrli:2013fv}

    p.346: This global analysis reveals an annual CO2 flux of 1.8 petagrams (Pg; 1 petagram is 109 tonnes) from rivers to the atmosphere, and 0.32 Pg from lakes and reservoirs.p.347: excluded wetlands, for which coarser estimates are available. -- Highlighted jun 25, 2015 \citep{Wehrli:2013fv}.

    p.347: Where is all this carbon coming from? There are three main possibilities: soil respiration, which increases inorganic carbon concentrations in groundwater; soil erosion, which transports organic-rich particles into streams; and the entry of a substantial amount of dead biomass into water courses in forested and wetland systems7 -- Highlighted jun 25, 2015 \citep{Wehrli:2013fv}

    p.347: Because tropical regions are seriously under-represented in global data sets, additional studies of carbon concentrations in the predicted hotspot areas in the tropics are urgently needed. Efforts to constrain data for global emissions of methane — a potent greenhouse gas — from inland waters are also a high priority -- Highlighted jun 25, 2015 \citep{Wehrli:2013fv}.


    \subsubsection{Add Eh calculation based on Gibbs Energy and make SVD, PCA, frequency analyses of measurements and modeled data. Relate to Eh measurements} % (fold)
    \label{sub:make_svd_pca_and_try_to_relate_to_eh_measurements}

    % subsubsection make_svd_pca_and_try_to_relate_to_eh_measurements (end)


    \subsubsection{Possible application: Great Lakes due to anthropogenic radioactivity.} % (fold)

    \label{sub:model_the_lake}
                \begin{figure}[htbp]
                \centering
                \includegraphics[width=0.95\textwidth]{sources_of_rad_contamination}
                \caption{Sources of nuclear contamination in Great Lakes. Figure taken from \citep{Joshi1991} }
                \label{fig:sources_of_rad_contamination}
            \end{figure}

            Add contaminating species such as \ce{^{131}I}, \ce{^{132}I}, \ce{^{132}Te}, \ce{^{134}Cs}, and \ce{^{137}Cs}.


            We could try to couple lake model with sediments to WSPEEDI-II:

            \emph{The computer-based nuclear emergency response system, Worldwide Version of System for Prediction of Environmental Emergency Dose Information (WSPEEDI-II) was used to reproduce the event which had occurred in the atmospheric environment during the period from 15 to 17 March 2011 in Fukushima Prefec- ture, Japan (Fig. 2). WSPEEDI-II includes the combination of models, a non-hydrostatic atmospheric dynamic model (MM5, Grell et al., 1994) and Lagrangian particle dispersion model (GEARN, Terada and Chino, 2008). MM5 predicts three-dimensional fields on wind, precipitation, diffusion coefficients, etc. based on atmo- spheric dynamic equations with appropriate spatial and temporal resolution, by using domain nesting method. GEARN calculates the advection and diffusion of radioactive plumes, dry and wet depo- sition onto the ground surface, and air dose rate from radionuclides in the air by the submersion model and on the ground surface (ground-shine). GEARN can predict the atmospheric dispersion for two domain simultaneously based on the meteorological fields of each domain by MM5 by considering in- and outflow between the domains. The performance of this model system was evaluated by its application to the field tracer experiment over Europe, ETEX (Furuno et al., 2004) and Chernobyl nuclear accident (Terada et al., 2004; Terada and Chino, 2005, 2008). Further information of WSPEEDI-II is available in Terada et al. (2004) and Terada and Chino (2005).}

    % subsubsection model_the_lake (end)

    % subsection possible_plan_of_research (end)

    \subsection{Philippe Skype ideas} % (fold)
    \label{sec:philippe_skype_ideas}

        Philippe:

        If you focus on phosporus there is no direct link necessarelly with carbon degradation and the energy which comes out of the carbon oxidation. And what would be the advantage of taking bioenergetic view if there is no direct bioenergetics on the phosporus itself... I am just thinking about.. because you are looking on the full scale of lake modeling where you would have photosynthesis on the top etc... I think one of the ideas would be intersting is having of this focus on northern regions. Kind of a more subarctic and arctic climate because maybe there is an initiative taking place from canada and in scandinavia, I am also in contact with Pierre Rene? They are trying to set something up what would deal with Syberian part of Russia, so focus on these cold regions and to implement some climate change scenarios. When you say for instance coupling in to the atmosperic model, 1 of the things what we could look is that will happen if yearly average temperature is increasing that mostly going to increase period of photosynthesis, etc. Essentially, increasing the delivery of OC to the sediments or respiration and ultimately more CO2 and methane production. Then, the focus maybe could be lets say CO2 and CH4. So you still stuck with carbon but start specifically start lookin at including bioenergetic yilds actualy controls methane production. One of the things that will be remain an issue is the nature of OM itself. Maybe we can some how link it, I know Chris has a lot of interest in it, to look at this hydrolysis using hydrolysis (asis?) of natural of organic matter and see how hydrolisable it is. People have done that , I know that Bore Orgenson have done that...They were looking at this.


        And thats the other thing. For instance, if you look at the sediments, and you say that sulfate reduction is taking place deeper then a lot of people say that the rate will be lower because you left with more refractory organic matter. Thts why it is important to llok at the ways to separate what is intrinsically due to the fact that you have less energy and it is apparantly less efficient way of living vs the fact that because of that you are deep in the sediments and juicy organic metter is already done. You need to think about this once I have my glucose produced then I can deal with what happening at the cell level, but the other thing is to look more at this hydrolysis process. And how we can get progress there on that level.

        Maybe you should keep focus on carbon for the moment. Because the list of the elements is very large. and in terms of nothern regions carbon is defenetly the element of interest now. and maybe later on the could think about how we can include P.


        Then about anthropogenic




        \section{Pupa Gilbert. Biominerals. Nacre nano-structure as a T proxy. (physics), UW-Madison} % (fold)
        \label{sec:pupa_gilbert_nacre_nanostructure_as_a_t_proxy_}

        Dependence of thickness of biomimnerals in mollusks on the temperature where they grow:

        Tablet thickness = 0.322 + 0.012 T





    \bibliographystyle{plainnat}
    \bibliography{bibliograph}
\end{document}
